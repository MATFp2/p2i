\documentclass{article}
\usepackage[utf8]{inputenc}
\title{{\bf Programiranje 2\\ \emph{Napredni ulaz/izlaz. Vežbanje.}}}

\usepackage{preambula}

\def\d{{\fontencoding{T1}\selectfont\dj}}
\def\D{{\fontencoding{T1}\selectfont\DJ}}

\begin{document}

\maketitle

\section{Zadaci sa časa}
\begin{z} Napisati program koji filtrira datoteku imena.txt tako sto na standardni izlaz ispisuje samo imena i prezimena. Maksimalna duzina imena, kao i prezimena je 50 karaktera.
\begin{Verbatim}%
  [formatcom=\color{primeri}]
\textbf{Primer1: }
imena.txt:
1. Pera Peric 7.5
2. Mika Mikic 8.0
3. Milos Savic 10.0

Pera Peric
Mika Mikic
Milos Savic
\end{Verbatim}  
\end{z}

\begin{z} Programu se kroz argumente komandne linije zadaje proizvoljan broj razlomaka u formatu a/b.
Na standardni izlaz ispisati vrednosti ovih razlomaka zapisane na dve decimale. U slučaju greške na standardni izlaz za greške ispisati -1.
\begin{Verbatim}%
  [formatcom=\color{primeri}]
\textbf{Primer1:            Primer 2:            Primer 3:}
./a.out 3/4 2/3     ./a.out              ./a.out 2/3 34 1/2

0.75 0.67                                -1
\end{Verbatim}  
\end{z}

\begin{z} 
Napisati program koji za uneto n, formira fajlove 1.txt, 2.txt, ..., n.txt i popunjava ih na sledeci nacin: \\
U 1.txt smesta cele brojeve [0,n] sa korakom 1 (0,1,...,n) \\
U 2.txt smesta cele brojeve [0,n] sa korakom 2 (0,2,4,...) \\
...\\
U n.txt smesta cele brojeve [0,n] sa korakom n (0 i n) \\
Maksimalna dužina imena fajla je 50 karaktera. Vrednost promenljive n mora biti u opsegu [0,100]. U slučaju greške na standardni izlaz za greške ispisati -1.
\begin{Verbatim}%
  [formatcom=\color{primeri}]
\textbf{Primer1:              Primer 2:}
5                     10

1.txt: 0 1 2 3 4 5    1.txt: 0 1 2 3 4 5 6 7 8 9 10
2.txt: 0 2 4          2.txt: 0 2 4 6 8 10
3.txt: 0 3            3.txt: 0 3 6 9
4.txt: 0 4            4.txt: 0 4 8
5.txt: 0 5            5.txt: 0 5 10
                      6.txt: 0 6
                      7.txt: 0 7
                      8.txt: 0 8
                      9.txt: 0 9
                      10.txt: 0 10
\end{Verbatim}  
\end{z}

\begin{z} 
Napisati program koji u datoteci čije se ime zadaje kao argument komandne linije, pronalazi i na standardni izlaz ispisuje sve linije u kojima se zadata reč pojavljuje n puta. Tražena rec i broj pojavljivanja n se zadaju sa standardnog ulaza. Maksimalna dužina reči je 20 karaktera. U slučaju greške, na standardni izlaz za greške ispisati -1.
\begin{Verbatim}%
  [formatcom=\color{primeri}]
\textbf{Primer1:        }
./a.out ulaz.txt

ulaz.txt
ovo je linija koja rec kisa sadrzi samo jednom
danas je padala kisa a sutra kisa nece padati
kisa je padala kisa kisa padala
kisa kisa kisa
ova recenica ne sadrzi trazenu rec

\textbf{Standardni ulaz:}
3
kisa

\textbf{Standardni izlaz:}
kisa je padala kisa kisa padala
kisa kisa kisa

\end{Verbatim}  
\end{z}

\newpage

\section{Domaći zadaci}

\begin{z}
Zadaci 1 i 2 sa prošlogodišnjeg kolokvijuma (sve tri grupe)
\end{z}

\begin{z}
Na standarnom ulazu se navode nazivi dve datoteke (ulazna i izlazna) i opcije. 
U datoteci \v cije se ime navodi kao prvo nalaze se podaci o razlomcima: 
   u prvom redu se nalazi broj razlomaka, a u svakom slede\' cem redu brojilac i imenilac jednog razlomka. 
   Potrebno je kreirati strukturu koja opisuje razlomak i u\v citati niz razlomaka
   iz datoteke, a potom:
   \begin{description}
      \item[a)] ukoliko je navedena opcija {\tt x}, upisati u drugu datoteku  
         recipro\v cni razlomak za svaki razlomak iz niza (npr. za 2/3 treba upisati 3/2)
      \item[b)] ukoliko je navedena opcija {\tt y}, upisati u datoteku 
         realnu vrednost (ispisati samo ne--nula decimale) recipro\v cnog razlomka svakog razlomka iz niza 
(npr. za 2/3 treba upisati 1.5)  
   \end{description}
   Mo\v zemo pretpostaviti da se u datoteci sa podacima o razlomcima nalazi najvi\v se 100 razlomaka.

  Prilikom pokretanja programa se, pored naziva ulazne i izlazne
  datoteke, navode i opcije {\tt -x} i {\tt -y}. Mogu\'ce je navesti jednu ili
  obe opcije. Maksimalna du\v zina naziva datoteka je 20 karaktera.
  
  Mogu\' ci na\v cini pokretanja:
  \begin{Verbatim}
   ulaz.txt izlaz.txt -x
   ulaz.txt izlaz.txt -y
   ulaz.txt izlaz.txt -yx
   ulaz.txt izlaz.txt -xy
  \end{Verbatim}
\end{z}

\begin{z}
  Za svaki automobil poznati su marka, model i cena. Iz datoteke \v cije
  se ime zadaje sa standardnog ulaza u\v citava se broj automobila a potom
  i podaci za svaki automobil. Program treba da:
  \begin{description}
  \item[a)] izra\v cuna prose\v cnu cenu po marki kola
  \item[b)] za maksimalnu cenu koju je kupac spreman da plati, a koja se zadaje
  kao argument komandne linije, da ispi\v se automobile u tom cenovnom
  rangu zajednu sa prose\v cnom cenom odgovarajuce marke.
  \end{description}
  
  Mo\v zemo pretpostaviti da se model i marka sastoje od jedne re\v ci i 
  da svaka od njih sadr\v zi najvi\v se 30 karaktera kao i da se u datoteci
  nalaze podaci za najvi\v se 100 automobila. Maksimalna du\v zina naziva datoteka je 20 karaktera.
\end{z}

\end{document}

\begin{z}
Ovdde ide tekst zadatka.
\end{z}
\begin{Verbatim}%
  [formatcom=\color{primeri}]
\textbf{Primer1:                 Primer 2:}
nije bf             vlvlvl
nesto dalje         zzzz
\end{Verbatim}
