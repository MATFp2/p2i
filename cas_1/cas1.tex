\documentclass{article}
\usepackage[utf8]{inputenc}
\title{{\bf Programiranje 2\\ \emph{Datoteke, argumenti komandne linije, pseudo-slučajni brojevi}}}

\usepackage{preambula}

\def\d{{\fontencoding{T1}\selectfont\dj}}
\def\D{{\fontencoding{T1}\selectfont\DJ}}

\begin{document}

\maketitle

\section{Zadaci sa časa}

\begin{z}
Napisati program koji prepisuje datoteku \verb|ulaz.txt| u datoteku
\verb|izlaz.txt| i to:
\begin{description}
\item[a)] karakter po karakter
\item[b)] liniju po liniju
\end{description}
U slu\v caju gre\v ske na standardni izlaz za greške ispisati {\tt -1}. Pretpostaviti da je maksimalna dužina linije 80 karaktera.
\end{z}
\begin{Verbatim}%
  [formatcom=\color{primeri}]
\textbf{Primer1:                 Primer 2:                Primer 3:}
ulaz.txt:                ulaz.txt:                ulaz.txt:
danas je lep dan         Cena soka je 30          
i ja zelim da            Cena vina je 150
postanem programer       Cena limunade je 200     izlaz.txt:
                         Cena sendvica je 120
izlaz.txt:
danas je lep dan         izlaz.txt:
i ja zelim da            Cena soka je 30  
postanem programer       Cena vina je 150
                         Cena limunade je 200
                         Cena sendvica je 120
\end{Verbatim}

\begin{z}
Napisati program koji na standarni izlaz ispisuje broj linija u
tekstualnom fajlu sa imenom \verb|knjiga.txt|. U slu\v caju gre\v ske
na standardni izlaz za greške ispisati -1.  Pretpostaviti da je maksimalna dužina linije 80 karaktera.
\end{z}
\begin{Verbatim}%
  [formatcom=\color{primeri}]
\textbf{Primer1:                 Primer 2:                Primer 3:}
ulaz.txt:                ulaz.txt:                ulaz.txt:
danas je lep dan         Cena soka je 30          
i ja zelim da            Cena vina je 150
postanem programer       Cena limunade je 200     
                         Cena sendvica je 120

3                        4                        0
\end{Verbatim}

\begin{z}
Sa standarnog ulaza u\v citavaju se imena dve tekstualne datoteke i
jedan karakter.  Napisati program koji prepisuje datoteku \v cije se
ime navodi kao prva niska u datoteku \v cije ime se navodi kao
druga niska. Ukoliko je ucitan karakter \verb|u|, program prilikom
prepisivanja treba da zamenjuje sva mala slova velikim, a ukoliko je
u\v citan karakter \verb|l| sva velika slova se zamenjuju malim. U
slu\v caju greške na standardni izlaz za greške ispisati -1.  Maksimalna du\v
zina naziva datoteka je 20 karaktera.
\end{z}
\begin{Verbatim}%
  [formatcom=\color{primeri}]
\textbf{Primer1:                       Primer 2:                        Primer 3:}
ulaz.txt izlaz.txt u           prva.dat druga.dat  l            primer.c prazna.txt V       
ulaz.txt:                      prva.dat:                        primer.c:
danas je lep dan               Cena soka je 30                  #include <stdio.h>
i Ja zelim                     Cena vina je 150                 int main()
da postanem programer          Cena limunade je 200             \{
                               Cena sendvica je 120             \}
                               
izlaz.txt:                     druga.dat:                       -1                               
DANAS JE LEP DAN               cena soka je 30
I JA ZELIM                     cena vina je 150
DA POSTANEM PROGRAMER          cena limunade je 200
                               cena sendvica je 120
\end{Verbatim}                         

\begin{z}
Prvi red datoteke \verb|matrica.txt| sadr\v zi 2 cela broja manja od
50 koji predstavljaju redom broj vrsta i broj kolona realne matrice
A. Svaki slede\'ci red sadr\v zi po jednu vrstu matrice. Napisati
program koji pronalazi sve elemente matrice A koji su jednaki zbiru
svih svojih susednih elemenata i \v stampa ih u obliku
\begin{verbatim}
(broj vrste, broj kolone, vrednost elementa).
\end{verbatim}
 U slučaju greške na standardni izlaz za greške ispisati {\tt -1}.
Pretpostaviti da je sadr\v zaj datoteke ispravan. 
\end{z}
\begin{Verbatim}%
  [formatcom=\color{primeri}]
\textbf{Primer1:                       }
3 4
 1  2  3  4
 7  2 15 -3
-1  3  1  3

(1, 0, 7)
(1, 2, 15)
\end{Verbatim}

\begin{z}
 U datoteci \v cije se ime navodi na standarnom ulazu
  programa nalazi se broj \verb|n|, a zatim i \verb|n| re\v ci (du\v zine
  najvi\v se 50 karaktera). Napisati program koji u\v citava ovaj niz
  i
  \begin{enumerate}
  \item ispisuje ga na standardni izlaz,
  \item iz njega uklanja sve duplikate i u datoteku \verb|rez.txt|
    ispisuje transformisani niz
  \end{enumerate}
  U slučaju greške na standardni izlaz za greške ispisati {\tt -1}. Maksimalna du\v zina naziva datoteka je 20 karaktera. Maksimalan broj reči je 256.
\end{z}


\begin{Verbatim}%
  [formatcom=\color{primeri}] 
\textbf{Primer 1:                                       |        Primer 2:}
dat1.txt                                        |        dat2.txt
dat1.txt: 12 jha14 hahaha deda mraz deda        |        dat2.txt:  14
          mraz deda deda jase konj konj konj    |        so secer supa so ljuto secer kiselo slatko
                                                |        ljuto
jha14 hahaha deda mraz deda mraz deda           |        paprika, ljuta paprika, ljuto dete
deda jase konj konj konj                        | 
                                                |        so secer supa so ljuto secer kiselo slatko 
rez.txt: jha14 hahaha deda mraz jase konj       |        ljuto paprika, ljuta paprika, ljuto dete  
                                                |
                                                |        rez.txt: so secer supa ljuto kiselo slatko
                                                |        paprika, ljuta dete
---------------------------------------------------------------------------------------------------

\textbf{Primer 3:                                                                  |}
dat3.txt                                                                   |      
dat.txt: 17 Buducnost televizije su ultra HD, odnosno 4K                   |       
uredaji koji imaju ogromnu dijagonalu ekrana i znacajno vise piksela       |      
                                                                           |
Buducnost televizije su ultra HD, odnosno 4K                               |
uredaji koji imaju ogromnu dijagonalu ekrana i znacajno vise piksela       |
                                                                           |
rez.txt: Buducnost televizije su ultra HD, odnosno 4K                      |
uredaji koji imaju ogromnu dijagonalu ekrana i znacajno vise piksela       |
\end{Verbatim}

\begin{z}
U datoteci \v cije se ime navodi na standarnom ulazu
  programa nalazi se broj $n$, a zatim i $n$ re\v ci (du\v zine
  najvi\v se 50 karaktera). Napisati program koji u\v citava reči 
  i
  \begin{enumerate}
  \item ispisuje ih na standardni izlaz,
  \item u datoteku \verb|rez.txt| upisuje sve re\v ci koje sadr\v ze
    prvu re\v c i podvlaku.
  \end{enumerate}
  U slučaju greške na standardni izlaz za greške ispisati {\tt -1}.  Maksimalna du\v zina naziva datoteka je 20 karaktera. \\
NAPOMENA: Nije potrebno koristiti niz za čuvanje svih učitanih reči. 
\end{z}

\begin{Verbatim}%
  [formatcom=\color{primeri}]  
\textbf{Primer 1:                                     |        Primer 2:}
dat1.txt                                        |      dat2.txt
dat1.txt: 7 rec Opet _rec Reci rec_enica        |      dat2.txt:  11 Sunce sija iznad grada  
          DVa recica_                           |      Sunce_Moje Jedan Dva Su_nce Sve Sunce123_123 suncanica.
                                                |        
rec Opet _rec Reci rec_enica                    |      Sunce sija iznad grada
DVa recica_                                     |      Sunce_Moje Jedan Dva Su_nce Sve Sunce123_123 suncanica.
                                                |        
rez.txt: _rec rec_enica recica_                 |      rez.txt: Sunce_Moje Sunce123_123
                                                |
 --------------------------------------------------------------------------------------------------

\textbf{Primer 3:                                                                  }
dat3.txt                                                                   |      
dat.txt: 18 Na danasnji dan roden je poznati engleski pisac Carls Dikens,  |       
a umro reformator srpskog jezika Vuk Stefanovic Karadzic.                  |     
                                                                           |
Na danasnji dan roden je poznati engleski pisac Carls Dikens,              |
a umro reformator srpskog jezika Vuk Stefanovic Karadzic.                  |
                                                                           |
rez.txt:                                                                   |
                                                                           |
\end{Verbatim}

\begin{z}
Napisati program koji ispisuje broj navedenih argumenata komadne linije, a zatim i same argumente sa rednim brojevima.
\end{z}
\begin{Verbatim}%
  [formatcom=\color{primeri}]
\textbf{Primer 1:                  Primer 2:                  Primer 3:}
./a.out danas 63           ./a.out                    ./a.out -abc -f input.txt

3                          1                          4
1. ./a.out                 1. ./a.out                 1. ./a.out
2. danas                                              2. -abc
3. 63                                                 3. -f
                                                      4. input.txt
\end{Verbatim}

\begin{z}
Ako su celi brojevi \verb|a| i \verb|b| argumenti komandne linije na standardni izlaz ispisati sve brojeve koji pripadaju intervalu $[a,b]$. 
U slučaju greške na standardni izlaz za greške ispisati {\tt -1}.  
\end{z}
\begin{Verbatim}%
  [formatcom=\color{primeri}]
\textbf{Primer 1:                  Primer 2:                      Primer 3:           Primer 4:}
./a.out 34                 ./a.out 12 20                  ./a.out 30 8        ./a.out -4 -1

-1                         12 13 14 15 16 17 18 19 20     -1                  -4 -3 -2 -1
\end{Verbatim}

\begin{z}
Uobi\v cajena praksa na UNIX sistemima je da se argumenti komandne
linije dele na opcije i argumente u u\v zem smislu. Opcije po\v cinju
znakom ’-’ nakon \v cega obi\v cno sledi jedan ili vi\v se karaktera
koji ozna\v cavaju koja je opcija u pitanju. Ovim se naj\v ce\v s\' ce
upravlja funkcionisanjem programa i neke mogu\' cnosti se uklju\v cuju
ili isklju\v cuju. Argumenti na\v c\v s\' ce predstavljaju opisne
informacije poput na primer imena datoteka. Napisati program koji
ispisuje sve opcije koje su navedene u komandnoj liniji.
\end{z}
\begin{Verbatim}%
  [formatcom=\color{primeri}]
\textbf{Primer 1:                                       Primer 2:              Primer 3:}
./a.out -abc input.txt -d -Fg output            ./a.out                ./a.out ulaz.txt  

a b c d F g
\end{Verbatim}

\begin{z}
Napisati program koji poredi dva fajla i ispisuje redni broj linija u
kojima se fajlovi razlikuju.  Imena fajlova se zadaju kao argumenti
komandne linije.  Pretpostaviti da je maksimalna du\v zina
reda u datoteci 200 karaktera.  Linije brojati počevši od {\tt 1}. U slučaju greške na standardni izlaz za greške ispisati {\tt -1}. 
\end{z}
\begin{Verbatim}%
  [formatcom=\color{primeri}]
\textbf{Primer 1:                         Primer 2:                               Primer 3:}                    
./a.out ulaz.txt izlaz.txt        ./a.out primer1.dat primer2.dat         ./a.out prva.dat

ulaz.txt:                         primer1.dat:                            greska
danas vezbamo                     danas vezbamo 
programiranje                     analizu
ovo je primer kad su              ovo je primer kad
datoteke iste                     su datoteke razlicite

izlaz.txt:                        primer2.dat:
danas vezbamo                     danas vezbamo
programiranje                     programiranje
ovo je primer kad su              ovo je primer kad su
datoteke iste                     datoteke razlicite
                                  
                                  2 3 4
-----------------------------------------------------------------------------------------------------
\textbf{Primer 4:}
./a.out prva.dat druga.dat

prva.dat:                           druga.dat:
ovo je primer                       ovaj primer
kada su                             kada su        
datoteke                            datoteke
razlicite duzine                    razlicite
                                    duzine
                                    i kada treba ispisati broj
                                    tih redova

1 4 5 6 7
\end{Verbatim}

\begin{z}
Napraviti funkciju koja generi\v slu\v cajan realan broj od 0 i 1. Na standardni izlaz ispisati rezultat izvršavanja napisane funkcije.
\end{z}

\begin{z}
Parametri komandne linije su {\tt n, a, b} ($a < b$). Na standardni izlaz ispisati n slučajnih brojeva koji
su izme\d u {\tt a} i {\tt b}. U slučaju greške na standardni izlaz za greške ispisati {\tt -1}. 
\end{z}


\newpage

\section{Domaći zadaci}
\begin{z}
Sastaviti program koji sa standarnog ulaza u\v citava imena dve
datoteke (ulazna i izlazna datoteka) i iz ulazne datoteke kopira u
izlaznu svaki drugi karakter polaze\'ci od prvog pro\v citanog
karaktera. U slučaju greške na standardni izlaz za greške ispisati {\tt -1}. 
 Maksimalna du\v zina naziva datoteka je 20 karaktera.
\end{z}
\begin{Verbatim}%
  [formatcom=\color{primeri}]
\textbf{Primer1:                 Primer 2:                  Primer 3:}
ulaz.txt izlaz.txt       prva.dat druga.dat         primer.c prazna.txt          
ulaz.txt:                prva.dat:                  primer.c:
danas je lep dan         Cena soka je 30            #include <stdio.h>
i ja zelim               Cena vina je 150           int main()
da postanem programer    Cena limunade je 200       \{
                         Cena sendvica je 120       \}

izlaz.txt:               druga.dat:                 prazna.txt:
aa elpdnij ei            eask e3                    icue<ti.>itmi(
apsae rgae               eavn e10Cn iuaej 0
                         easnvc e10 
\end{Verbatim}

\begin{z}
Sastaviti program koji sa standardnog ulaza prima ime datoteke koju treba
otvoriti. Ispisati (na standardnom
izlazu) koja cifra (me\d u svim ciframa koje se pojavljuju u datoteci)
ima najve\' ci broj pojavljivanja. Ukoliko nema cifara u datoteci na standardni izlaz ispisati {\tt -1}.
Maksimalna du\v zina naziva datoteka je 20 karaktera.
\end{z}
\begin{Verbatim}%
  [formatcom=\color{primeri}]
\textbf{Primer1:                       Primer 2:                        Primer 3:}
ulaz.txt                       prva.dat                         primer.c        
ulaz.txt:                      prva.dat:                        primer.c:
danas je lep dan               Cena soka je 30                  #include <stdio.h>
i Ja zelim                     Cena vina je 150                 int main()
da postanem programer          Cena limunade je 200             \{
                               Cena sendvica je 120             \}
                               
-1                             0                                -1
\end{Verbatim}  

\begin{z}
Sa standarnog ulaza se u\v citava prirodan broj k i ime datoteke u kojoj se prvo nalazi
   prirodan broj n a zatim i n celih brojeva. Napisati program koji prebrojava
   koliko k-tocifrenih brojeva postoji u datoteci. U slu\v caju gre\v ske
na standardni izlaz za greške ispisati {\tt -1}. Pretpostaviti da je sadr\v zaj datoteke
   ispravan. Maksimalna du\v zina naziva datoteka je 20 karaktera.
\end{z}
\begin{Verbatim}%
  [formatcom=\color{primeri}]
\textbf{Primer1:                       Primer 2:                        Primer 3:}
3 ulaz.txt                       1 prva.dat                         5 primer.c        
ulaz.txt:                        prva.dat:                          primer.c:
10                               4                                  3
1 9 20 400 708                   1                                  4 5 50000
-2 -520 1000                     20
403 20000                        9
                                 -8
                                 
4                                3                                  1
\end{Verbatim} 

\begin{z}
Napisati program koji za dve datoteke čija se imena zadaju kao dve niske na standardnom ulazu, radi slede\'ce: za cifru u prvoj datoteci, u drugu
datoteku se upisuje 0, za slovo se upisuje 1, a za sve ostale karaktere se upisuje 2. U slu\v caju gre\v ske
na standardni izlaz za greške ispisati {\tt -1}. Maksimalna du\v zina naziva datoteka je 20 karaktera.
\end{z}
\begin{Verbatim}%
  [formatcom=\color{primeri}]
\textbf{Primer1:                       }
prva.dat druga.dat                   
prva.dat:                        
Cena soka je 30                  
Cena vina je 150                 
Cena limunade je 200             
Cena sendvica je 120             
                               
druga.dat:
11112111121120021111211112112000211112111111112112000211112111111112112000
\end{Verbatim}

\begin{z} Ako je data tekstualna datoteka \verb|plain.txt| napraviti tekstualnu
datoteku \verb|sifra.txt| tako \v sto se svako slovo zamenjuje svojim
prethodnikom (cikli\v cno) suprotne velicine \verb|’b’| sa \verb|’A’|,
\verb|’B’| sa \verb|’a’|, \verb|’a’| sa \verb|’Z’|, \verb|’A’| sa
\verb|’z’|, itd. Podrazumevati da se na sistemu koristi tabela
karaktera ASCII. U slu\v caju gre\v ske na standardni izlaz za greške ispisati {\tt -1}.
\end{z}

\begin{z} Sa standarnog ulaza se u\v citava ime tekstualne datoteke i prirodan
broj k. Podrazumeva se da zadata datoteka sadr\v zi samo slova i
beline i da je svaka re\v c iz datoteke du\v zine najvi\v se
100. Program treba da u\v citava re\v ci iz datoteke, da svaku re\v c
rotira za k mesta i da tako dobijenu re\v c upi\v se u datoteku \v
cije je ime \verb|rotirano.txt|. Maksimalna du\v zina naziva datoteka je 20 karaktera. U slu\v caju gre\v ske
na standardni izlaz za greške ispisati {\tt -1}.
\end{z}

\begin{z}
Napisati program koji u datoteku \verb|izlaz.txt| prepisuje sve re\v{c}i iz datoteke
\verb|ulaz.txt| \v{c}iji je zbir ascii kodova slova strogo ve\'{c}i od 1000. Re\v ci su
odvojene prazninama i nisu du\v ze od  200 karaktera. U slu\v caju gre\v ske
na standardni izlaz za greške ispisati {\tt -1}.
\end{z}
\begin{Verbatim}%
  [formatcom=\color{primeri}]
\textbf{Primer 1:                                               Primer 2:}
ulaz.txt:                                               ulaz.txt:
Sa standardnog ulaza unosi se neoznacen                 konstruisanje test-primera sa
ceo broj. Formirati novi broj koji se dobija            i dugackim recima kao prestolonaslednik
izbacivanjem svake druge cifre iz polaznog broja.       brojevima1234567890

izlaz.txt:                                              izlaz.txt:
standardnog izbacivanjem                                konstruisanje test-primera
                                                        prestolonaslednik
                                                        brojevima1234567890
______________________________________________________________________________________________________________

\textbf{Primer 3:                                                Primer 4:}
ulaz.txt:                                                ulaz.txt:
ima jos dugackih reci: predskazanje,                     i sada jedan kratak primer
potom                                                    p1: 1234567890
nelogicnosti, zanemarivati, odugovlaciti, a ima          p2: ABCDEFGHIJ
i i malih reci koje su kratke                            p3: abcdefghij
predosecaj

izlaz.txt:                                               izlaz.txt:
predskazanje, nelogicnosti,                              abcdefghij
zanemarivati, odugovlaciti,
predosecaj
\end{Verbatim} 

\begin{z}
U datoteci \verb|razno.txt| nalazi se tekst. U datoteku
\verb|palindromi.txt| prepisati sve re\v ci iz datoteke
\verb|razno.txt| koje su palindromi. Re\v c je palindrom ako se \v
cita isto sa leve i desne strane. Za re\v c smatramo niz karaktera
koji se nalazi izme\d u belina i koji nije du\v zi od 200
karaktera. Maksimalan broj re\v ci nije poznat. U slu\v caju gre\v ske
na standardni izlaz za greške ispisati {\tt -1}.
\end{z}
\begin{Verbatim}%
  [formatcom=\color{primeri}]
\textbf{Primer 1:                                               Primer 2:}
razno.txt:                                              razno.txt:
Ana i melem su primeri palindroma.                      jabuka neven pomorandza kuk

palindromi.txt:                                         palindromi.txt:
Ana i melem                                             neven kuk
______________________________________________________________________________________________________________

\textbf{Primer 3:                                               Primer 4:}
razno.txt:                                              razno.txt:
Kajak voda teret PoTop                                  Oko kapAk pero radar caj

palindromi.txt:                                         palindromi.txt:
Kajak teret PoTop                                       Oko kapAk radar
\end{Verbatim}

\begin{z}
   Imena dve datoteke se zadaje na standarnom ulazu.
   U prvoj datoteci 
   navedena je rec {\tt r} i niz linija. Napisati
   program koji u drugu datoteku  
   upisuje sve linije
   u kojima se re\v c {\tt r} pojavljuje bar {\tt n} puta, gde je
   n prirodan broj koji se unosi sa standardnog ulaza. Ispis
   treba da bude u formatu {\tt broj\_pojavljivanja: linija}. Linije brojati
   po\v cev\v si od {\tt 1}. Maksimalna du\v zina naziva datoteka je 20 karaktera.
\end{z}

\begin{z}
Kao argumenti komandne linje zadate su dimenzije matrice \verb|A|
(\verb|m| i \verb|n|).  Element matrice se naziva sedlo ako je
istovremeno najmanji u svojoj vrsti, a najve\'ci u svojoj
koloni. Ispisati indekse i vrednosti onih elemenata matrice koji su
sedlo. Pretpostaviti da je maksimalna dimenzija matrice $50\times 50$.
U slu\v caju gre\v ske na standardni izlaz za greške ispisati {\tt -1}.
\end{z}
\begin{Verbatim}%
  [formatcom=\color{primeri}]
\textbf{Primer 1:                    Primer 2:                   Primer 3:          Primer 4:}
./a.out 2 3                  ./a.out 3 3                 ./a.out 3          ./a.out 200 3

1 2 3                        10  3 20                    greska             greska
0 5 6                        15  5 100
                             30 -1 200
0 0 1
                             1 1 5
\end{Verbatim}

\end{document}

\begin{z}
Ovdde ide tekst zadatka.
\end{z}
\begin{Verbatim}%
  [formatcom=\color{primeri}]
\textbf{Primer1:                 Primer 2:}
nije bf             vlvlvl
nesto dalje         zzzz
\end{Verbatim}
