\chapter{Testovi i ispiti 2019/2020}

%------------------------------------------------------------------------
\section{Programiranje 2, ispit, junski rok 2019/2020}
%------------------------------------------------------------------------

\begin{enumerate}

\item U datoteci \texttt{fajlovi.txt} nalaze se, u svakom redu, ime fajla, dvotačka, a zatim i sadržaj fajla sa tim imenom. Potrebno je napraviti fajl sa tim imenom, a zatim tekst koji ide nakon dvotačke upisati u taj fajl, za svaki red u datoteci fajlovi.txt. \\
\textbf{NAPOMENA}: \textit{Fajlovi se mogu ponavljati, u tom slučaju nadovezati tekst. Pretpostaviti da imena fajlova neće biti duža od 15 karaktera, kao i da dužina linije u fajlu neće biti duža od 200 karaktera. }
\begin{verbatim}
Primer 1:                                     Primer 2:                               
fajlovi.txt:                                  fajlovi.txt:
fajl1.txt:Ovo je tekst fajla 1                datoteka.txt:Ovo je neka datoteka           
fajl2.txt:Ovo je tekst fajla 2                rezultati.txt:Ovo su neki rezultati
fajl3.txt:Ovo je tekst fajla 3 
   
fajl1.txt:                                    datoteka.txt:
Ovo je tekst fajla 1                          Ovo je neka datoteka

fajl2.txt:                                    rezultati.txt:
Ovo je tekst fajla 2                          Ovo su neki rezultati

fajl3.txt:
Ovo je tekst fajla 3
-----------------------------------------------------------------------------------
Primer 3 (prazan fajl):                       Primer 4 (ponavljanje):
fajlovi.txt:                                  fajlovi.txt:
zoran.txt:Tekst                               fajl1.txt:Ovo je tekst
fajl.txt:                                     fajl2.txt:fajl 2
                                              fajl1.txt:fajla 1  
                                              
zoran.txt:                                    fajl1.txt:
Tekst                                         Ovo je tekstfajla 1

fajl.txt:                                     fajl2.txt:
                                              fajl 2

\end{verbatim}
\item Napisati program koji sa standardnog ulaza učitava indekse studenata, njihova imena i prezimena (svaki student u jednom redu, ne više od 128 redova), a potom iste ispisuje na standardni izlaz sortirane rastuće po broju indeksa, a u slučaju istog broja indeksa, opadajuće po godini. Koristiti algoritam \textbf{insertion sort}. Pretpostavka je da su svi redovi zadati u ispravnom formatu.\\
\textbf{NAPOMENA}: \textit{ako je upotrebljen neki drugi algoritam, na zadatku se moze osvojiti najviše 60\%.}
\begin{verbatim}
Primer 1:                     Primer 2:                   Primer 3:                    
 
Standardni ulaz:              Standardni ulaz:            Standardni ulaz:
234/2014 Marko Markovic       56/1999 Marko Maric
1/2014 Pera Peric             223/2015 Pera Peric
234/2011 Branko Brankovic     224/2015 Mira Miric
123/2012 Branko Brankovic

Standardni izlaz:             Standardni izlaz:            Standardni izlaz:
1/2014 Pera Peric             56/1999 Marko Maric
123/2012 Branko Brankovic     223/2015 Pera Peric
234/2014 Marko Markovic       224/2015 Mira Miric
234/2011 Branko Brankovic

\end{verbatim}


\item Sa standarnog ulaza unosi se prvo broj $k$, a zatim ime datoteke u kojoj se nalaze informacije o studentima (u svakom redu ime, prezime i prosek).
Napisati funkcije za rad sa listama studenata:\\
(a) \textbf{Cvor *napraviCvor(char *ime, char *prezime, double prosek)} koja vraća pokazivač na novi čvor liste,\\
(b) \textbf{void dodajNaKraj(Cvor **glava, char *ime, char *prezime, double prosek)} koja kreira novi čvor i dodaje ga na kraj liste,\\
(c) \textbf{double suma(Cvor *glava, double k)} koja vraća sumu proseka svih studenata koji imaju prosek veći od $k$.\\
Zatim u glavnom programu testirati napisane funkcije, prvo učitati listu studenata iz datoteke dodavanjem na kraj, a zatim pozvati funkciju suma i ispisati vrednost na standardni izlaz zaokruženu na dve decimale.\\
\textbf{NAPOMENA}: \textit{Zadatak se mora uraditi pomoću liste, inače nosi 0 poena. Ime i prezime neće sadržati više od po 15 karaktera. Ukoliko prosek nije u intervalu [6,10] ispisati -1 na standardni izlaz za greške.}
\begin{verbatim}
Primer 1:                             Primer 2:  (primer prazne liste)           

studenti.txt:                         studenti.txt:
Mika Mikic 8.3
Zoran Petkovic 9.4
Goran Peric 7.4
Milan Milanovic 9.3
Milan Maric 6.5

Standardni ulaz:                      Standardni ulaz:      
7.5 studenti.txt                      7.5 studenti.txt

Standardni izlaz:                     Standardni izlaz:     
27.00                                 0.00
---------------------------------------------------------------------------
Primer 3:                             Primer 4 (nepostojeca datoteka):   

lista.txt:
Branko Brankovic 8.2
Petar Petrovic 6.7
Goran Gruzic 6.1
Aleksandar Golubovic 8.0

Standardni ulaz:                      Standardni ulaz:
8.3 lista.txt                         7.7 lista.txt

Standardni izlaz:                     Standardni izlaz za greške:
0.00                                  -1

\end{verbatim}

\item Sa standardnog ulaza unosi se binarno stablo pretrage. Napisati program koji računa zbir vrednosti čvorova koji imaju tačno jedno dete. Na standardni izlaz ispisati zbir, a potom i sve čvorove kojima je vrednost veća od izračunatog zbira, infiksno (LKD).\\
\textbf{NAPOMENA}:\textit{ Za rad sa binarnim pretraživačkim stablima obavezno koristiti datu biblioteku (\textbf{stabla.h} i \textbf{stabla.c}). Zadatak se mora rešiti korišćenjem binarnog pretraživačkog stabla. U suprotnom broj osvojenih poena je $0$.}
\begin{verbatim}
Primer 1:               Primer 2:               Primer 3:            Primer 4            
Standardni ulaz:        Standardni ulaz:        Standardni ulaz:     Standardni ulaz:
6 8 -7 -4 7             10 5 15 -20 7 12 16     -3 -2 1 3 4          1

Standardni izlaz:       Standardni izlaz:       Standardni izlaz:    Standardni izlaz:
1 6 7 8                 0 5 7 10 12 15 16       -1 1 3 4             0 1
\end{verbatim}  
 
\end{enumerate}	
