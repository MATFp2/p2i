\chapter{Testovi i ispiti 2007/08}
% *************************************************************
% *************************************************************
% *************************************************************

\section{Test 1, 14.04.2008.}



\begin{enumerate}
\item Napisati program koji iz datoteke \v cije se ime zadaje
kao prvi argument komandne linije \v cita sve re\v ci i zatim
odre\d uje i ispisuje re\v c koja se
pojavila najvi\v se puta u toj datoteci (pretpostavljamo da su sve
re\v ci u datoteci du\v zine najvi\v se $20$ karaktera).
Koristiti pogodno definisanu strukturu.

\item Za zadati ceo broj $n$ ve\'ci od 9 mo\v zemo primetiti da je suma
cifara tog broja uvek manja od broja $n$.
Neka je $n_1$ suma cifara broja $n$,
neka je $n_2$ suma cifara broja $n_1$,
$n_3$ suma cifara broja $n_2$, $\ldots$,
$n_{i+1}$ suma cifara broja $n_{i}$. Va\v{z}i
$n > n_1 > n_2 > \ldots > n_i$ dok god su brojevi
$n_i$ ve\'ci od 9, pa se u jednom trenutku sigurno
dolazi do jednocifrenog broja $n_k$. Taj indeks $k$
zavisi od po\v{c}etnog broja $n$.

Na primer:
\begin{itemize}
\item za $n=10$, $n_1 = 1+0 = 1$, va\v{z}i $k=1$
\item za $n=39$, $n_1 = 3+9 = 12$, $n_2 = 1+2 = 3$, va\v{z}i  $k=2$
\item za $n=595$, $n_1 = 5+9+5 = 19$, $n_2 = 1+9 = 10$, $n_3 = 1+0 = 1$,
va\v{z}i  $k =3$
\end{itemize}

Napisati program koji za uneto $n$ sa standardnog ulaza ra\v cuna
njemu odgovaraju\'ci broj $k$. Zadatak re\v siti bez kori\v s\' cenja nizova.


\item Za datu kvadratnu matricu ka\v zemo da je \emph{magi\v cni
kvadrat} ako je suma elemenata u svakoj koloni i svakoj vrsti
jednaka. Napisati program koji sa standardnog ulaza u\v citava
prirodni broj $n$ ($n<10$) i zatim elemente kvadratne matrice,
proverava da li je ona \emph{magi\v cni kvadrat} i ispisuje
odgovaraju\' cu poruku na standardni izlaz.

Primer, matrica:

\begin{verbatim}
1 5 3 1
2 1 2 5
3 2 2 3
4 2 3 1
\end{verbatim}

je magi\v cni kvadrat.
\end{enumerate}


\section{Test 2, 17.06.2008.}


\begin{enumerate}
\item
Napisati program koji implementira red pomo\' cu jednostruko
povezane liste (\v cuvati pokaziva\v ce i na po\v cetak i na kraj
reda).
Sa standardnog ulaza se unose brojevi koje sme\v stamo u red sa
nulom ($0$) kao oznakom za kraj unosa.
Napisati funkcije za:

\begin{itemize}
\item Kreiranje elementa reda;
\item Ubacivanje elementa na kraj reda;
\item Izbacivanje elementa sa po\v cetka reda;
\item Ispisivanje reda;
\item Osloba\d anje reda.
\end{itemize}


\item
Napisati program koji formira ure\d eno binarno stablo bez ponavljanja
elemenata. Elementi stabla su celi brojevi i unose se sa ulaza, a oznaka
za kraj unosa je nula. Napisati funkciju koja proverava da li je uneto
stablo \emph{lepo}. Stablo je \emph{lepo} ako za svaki \v cvor
stabla va\v zi da mu se broj \v cvorova u levom i u desnom podstablu
razlikuju najvi\v se za jedan.

\item
Iz datoteke \emph{ulaz.txt} u\v citavamo niz celih brojeva (pri \v cemu
du\v zina datoteke nije unapred poznata). Napisati program koji
ovaj niz sortira pozivom funkcije \emph{qsort} i zatim ga
upisuje u datoteku \emph{izlaz.txt}.
\end{enumerate}


\section{Zavr\v{s}ni ispit, juni 2008.}



\begin{enumerate}

\item
Napisati program koji sa standardnog ulaza u\v citava pozitivne cele
brojeve dok ne u\v cita nulu kao oznaku za kraj. Na standardni izlaz
ispisati koji broj se pojavio najvi\v se puta me\d u tim brojevima.
Na primer, ako se na ulazu pojave brojevi:
\verb+2 5 12 4 5 2 3 12 15 5 6 6 5+
program treba da vrati broj $5$.

Zadatak re\v siti kori\v{s}\'cenjem dinami\v cke realokacije i strukture
koja sadr\v zi ceo broj i broj njegovih pojavljivanja.

\item
Napisati program koji formira sortiranu listu od niza celih brojeva koji se
unose sa standardnog ulaza. Oznaka za kraj unosa je $0$. Napisati funkcije za
formiranje \v cvora liste, ubacivanje elementa u ve\' c sortiranu listu,
ispisivanje elemenata liste i osloba\d anje liste.


\item Napisati funkcije potrebne za ispisivanje elemenata koji se nalaze
na najve\' coj dubini binarnog stabla.

Na primer, za stablo

\begin{minipage}[t]{120mm}
\begin{verbatim}
        5
      /   \
     3     6
   /   \
  2     4
\end{verbatim}
\end{minipage}

\noindent
treba ispisati: {\tt 2 4}.

(Pretpostavljamo da je stablo ve\' c zadato. \emph{Ne treba} pisati dodatne
funkcije za kreiranje \v cvora, uno\v senje elementa u stablo i
osloba\d anje stabla)


\item
Sa standardnog ulaza u\v citavamo prvo broj studenata a zatim i
njihove podatke. Za svakog studenta dobijamo ime (niska od najvi\v se
30 karaktera) i broj indeksa (ceo broj). Napisati program koji
sortira ovaj niz studenata po imenima studenata pozivom standardne
funkcije \emph{qsort} i zatim \v stampa tako dobijeni niz na standardni
izlaz.
\end{enumerate}



\section{Zavr\v{s}ni ispit, septembar 2008.}




\begin{enumerate}

\item

Napisati program koji sa standardnog ulaza u\v citava tekst nepoznate
du\v zine i sme\v sta ga u (dinami\v cki) niz karaktera. Oznaka za kraj
unosa je uneta $0$. Nakon toga ispisati uneti tekst na standardni izlaz
po $10$ karaktera u redu. Zadatak re\v siti kori\v s\' cenjem dinami\v cke
realokacije.

Na primer, ako je uneto:

\verb+"Ja polazem ispit iz programiranja2"+

program treba da ispi\v se:

\begin{verbatim}
Ja polazem
ispit iz p
rogramiran
ja2
\end{verbatim}

\item

Napisati program koji formira ure\d eno binarno stablo koje sadr\v zi cele
brojeve. Brojevi se unose sa standardnog ulaza sa nulom kao oznakom za kraj
unosa. Napisati funkcije za:
\begin{itemize}
\item Kreiranje jednog \v cvora stabla;
\item Ubacivanje elementa u ure\d eno stablo (bez ponavljanja elemenata);
\item Ispisivanje stabla u preorder (prefix) redosledu;
\item Odre\d ivanje zbira elemenata stabla;
\item Ispisivanje listova stabla;
\item Osloba\d anje stabla.
\end{itemize}

Primer, za stablo:
\begin{verbatim}
    5
   / \
  3   7
 / \
2   4
\end{verbatim}

zbir svih elemenata je $5+3+7+2+4 = 21$ a listovi su: $2$, $4$ i $7$.

\item

Napisati program koji iz datoteke \v cije se ime zadaje kao argument
komandne linije, \v cita prvo broj elemenata niza pa zatim i elemente niza
(celi brojevi). Ovaj niz sortirati pozivom funkcije \emph{qsort} a zatim za
uneti ceo broj sa standradnog ulaza proveriti, pozivom funkcije
\emph{bsearch}, da li se nalazi u nizu ili ne i ispisati odgovaraju\' cu
poruku.
\end{enumerate}


\section{Zavr\v{s}ni ispit, oktobar 2008.}


\begin{enumerate}

\item

Napisati program koji sa standardnog ulaza unosi sortirani niz celih brojeva
dok ne unesemo nulu kao oznaku za kraj (niz alocirati dinami\v cki).
U slu\v caju da niz nije sortiran ispisati podatak o gre\v sci na standardni
izlaz a u suprotnom ispisati \emph{medijanu} tog niza.
Medijana (sortiranog) niza koji ima $n$
\v clanova je, u slu\v caju kada je $n$ neparan broj, sredi\v snji element
niza odnosno, u slu\v caju kada je $n$ paran broj, srednja vrednost dva
sredi\v snja elementa.

Na primer,

ako je dat niz $1$, $3$, $5$, $6$, $12$, medijana je broj $5$

a ako je dat niz $3$, $5$, $8$, $13$, $20$, $25$, medijana je broj $10.5$.

\item

Napisati program koji kreira jednostruko povezanu listu. Elementi (celi
brojevi) se unose sa standardnog ulaza dok se ne unese nula kao oznaka za
kraj. Napisati:
\begin{itemize}
\item funkciju koja kreira jedan \v cvor liste,
\verb+CVOR* napravi_cvor(int br)+
\item funkciju za ubacivanje broja na kraj liste,
\verb+void ubaci_na_kraj(CVOR** pl, int br)+
\item funkciju koja skida element sa po\v cetka liste (pri \v cemu se menja
po\v cetak liste) i vra\' ca vrednost tog broja kao povratnu vrednost,

\verb+int izbaci_sa_pocetka(CVOR** pl)+
\item funkciju za ispisivanje elemenata liste,
\verb+void ispisi_listu(CVOR* l)+
\item funkciju za osloba\d anje liste,
\verb+void oslobodi_listu(CVOR* l)+
\end{itemize}

\item

Napisati program koji sa standardnog ulaza u\v citava broj artikala (ne
vi\v se od $50$) a zatim imena (karakterske niske du\v zine do $20$
karaktera) i cene artikala (ceo broj). Ovaj niz artikala sortirati po ceni
(pozivom funkcije \emph{qsort}) a zatim za uneti ceo broj $c$ sa standardnog
ulaza prona\' ci (pozivom funkcije \emph{bsearch}) naziv artikla sa
tom cenom. Ako takav artikal ne postoji ispisati odgovaraju\' cu
poruku.
\end{enumerate}


