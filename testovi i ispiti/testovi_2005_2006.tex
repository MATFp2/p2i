\chapter{Testovi i ispiti 2005/06}
% *************************************************************
% *************************************************************
% *************************************************************

\section{Test 1, 03.04.2006.}


1. Kori\v s\' cenjem nizova izra\v cunati pribli\v znu vrednost
realnog broja $1/x$ (za $0 < x \le 2$) sa zadatom ta\v cno\v s\' cu
$\epsilon$ ($0 \le \epsilon \leq 0.01$) na slede\' ci na\v cin:
\begin{itemize}
\item odrediti elemente nizova $(a)$ i $(c)$ koju su zadati sa:

$a[0] = 1$

$c[0] = 1-x$

$a[i] = a[i-1] \cdot (1+c[i-1]), \;\;\; \textrm{za} \;\;\; i \ge 1$

$c[i] = c[i-1] \cdot c[i-1], \;\;\; \textrm{za} \;\;\; i \ge 1$

\item za vrednost broja $1/x$ uzima se ona vrednost $a[n]$ za koju
je zadovoljen uslov $-\epsilon \le a[n]-a[n-1] \le \epsilon$.
\end{itemize}

Prostor za nizove $(a)$ i $(c)$ dinami\v cki alocirati i pove\' cavati
im du\v zinu (u ve\'cim blokovima) dok ne bude zadovoljena tra\v zena
ta\v cnost.

Na primer:

za $x = 0.5$ i $\epsilon = 0.0001$ program treba da vrati $2.00000$

za $x = 0.5$ i $\epsilon = 0.01$ program treba da vrati $1.999969$ \\


2. Napisati funkciju koja kao (jedini) arugment ima ime datoteke
koja sadr\v{z}i dimenziju kvadratne matrice i njene elemente redom
(po vrstama). Funkcija treba da ispi\v{s}e elemente matrice u grupama
koje su paralelne sa sporednom dijagonalom matrice.

Na primer, datoteka sa sadr\v{z}ajem \verb|3 1 2 3 4 5 6 7 8 9|
opisuje matricu

$\begin{array}{ccc}
1 & 2 & 3 \\
4 & 5 & 6 \\
7 & 8 & 9
\end{array}$

a funkcija treba da ispi\v{s}e slede\'ce:

$\begin{array}{ccc}
1 &  & \\
2 &  4 & \\
3 & 5 & 7 \\
6 & 8 & \\
9 &  &
\end{array}$

Mo\v{z}e se pretpostaviti da matrica nije dimenzije ve\'ce od
$100 \times 100$. \\

3. Napraviti program za a\v{z}uriranje reda \v{c}ekanja u studentskoj
poliklinici. Treba obezbediti:

\begin{itemize}
\item funkciju koja sa standardnog ulaza \v{c}ita broj indeksa studenta
koji je do\v{s}ao (samo jedan ceo broj);
\item funkciju koja ubacuje studenta (sa datim brojem indeksa) na kraj
liste \v{c}ekanja u vremenu $O(1)$;
\item funkciju koja odre{\d}uje slede\'ceg studenta za pregled i bri\v{s}e
ga sa liste u vremenu $O(1)$;
\item funkciju koja \v{s}tampa trenutno stanje reda \v{c}ekanja.
\item funkciju koja oslobadja celu listu.
\end{itemize}




\section{Test 2, 01.06.2006.}



1. Napisati program koji za uredjeno binarno stablo ispisuje
elemente na najve\' coj dubini. Napisati funkcije za kreiranje
\v cvora, uno\v senje elementa u stablo, oslobadjanje stabla.

Na primer, za stablo

\begin{minipage}[t]{120mm}
\begin{verbatim}
        5
      /   \
     3     6
   /   \
  2     4
\end{verbatim}
\end{minipage}

\noindent
program treba da ispi\v se: {\tt 2 4}.

2. Sa ulaza \v citamo niz re\v ci (dozvoljeno je ponavljanje re\v ci) i
sme\v stamo ih u listu (koja osim re\v ci \v cuva i broj pojavljivanja
za svaku re\v c). Napisati funkciju koja sortira listu algoritmom
\emph{bubble sort} po broju pojavljivanja re\v ci. Napisati funkciju
koja ispisuje listu po\v cev\v si od re\v ci koja se pojavila najvi\v se puta.

3. Sa standardnog ulaza u\v citavamo prvo broj studenata a zatim i
njihove podatke. Za svakog studenta dobijamo ime (niska od najvi\v se
30 karaktera) i broj indeksa (ceo broj). Napisati program koji
sortira ovaj niz studenata po imenima studenata pozivom standardne
funkcije \emph{qsort} a zatim pronalazi broj indeksa studenta \v cije
se ime zadaje sa standardnog ulaza pozivom funkcije \emph{bsearch}.

4. Napisati program koji simulira rad sa velikim brojevima.
Cifre velikog broja sme\v stati u niz. Pretpostavljamo da broj
ne\' ce imati vi\v se od $1000$ cifara.

Napisati funkcije za:
\begin{itemize}
\item Uno\v senje broja sa standardnog ulaza (pri \v cemu cifre
broja treba uneti na standardni na\v cin tj. po\v cev\v si od
cifre najve\' ce te\v zine) i sme\v stanje u niz koji se predaje
kao argument funkcije. Du\v zinu niza (odnosno broja) vratiti kao
povratnu vrednost funkcije.
\item Ispis broja (odnosno niza) na standardni izlaz.
\item Oduzimanje dva velika broja (niz za sme\v stanje rezultata proslediti
funkciji kao argument). Pretpostavljamo da je prvi argument ve\' ci od drugog.
\end{itemize}

Pre poziva funkcije za oduzimanje brojeva obezbediti da cifra najmanje
te\v zine bude sme\v stena u $a[0]$ (ako je $a$ niz koji predstavlja
veliki broj).

Za unete brojeve {\tt 456 189} program treba da ispise {\tt 267}.

Za unete brojeve {\tt 456 89} program treba da ispise {\tt 367}.
\vspace*{3mm}



\section{Programiranje II, Zavr\v{s}ni ispit, jun 2006.}



\begin{enumerate}
\item
Napisati program za izra\v cunavanje koeficijenata polinoma $T_{n}(x)$ za
proizvoljno veliko $n$. Pri tome je $T_{0}(x) = a$, $T_{1}(x) = b \cdot x + c$
(celobrojne vrednosti $n$, $a$, $b$ i $c$ su argumenti komandne linije), a
$T_{n}$ se izra\v cunava na osnovu formule $T_{n} = x \cdot T_{n-1} + T_{n-2}$.
Polinom upisati u datoteku \v cije se ime zadaje kao peti argument komandne linije.
Primer: nakon poziva

\verb|program 2 1 2 3 izlaz.txt|

\noindent
u datoteci \verb|izlaz.txt| bi\'ce sadr\v{z}aj:

\verb|T2(x) = 2*x^2 + 3*x^1 + 1|


\item
Celobrojni aritmeti\v cki izraz koji uklju\v{c}uje jednu promenljivu predstavljen
je binarnim stablom. Na primer, izraz $2 + (3 * x)$ predstavljen je stablom:

\begin{verbatim}
   +
  / \
 2   *
    / \
   3   x
\end{verbatim}

Izraz mo\v{z}e da sadr\v{z}i samo binarne operatore \verb|+| i \verb|*|.

\begin{itemize}
\item[-]
Definisati strukturu \verb|cvor| kojom se mogu opisati \v{c}vorovi
ovakvog stabla.

\item[-]
Napisati funkciju

\verb|int vrednost(char ime_promenljive, int vrednost_promenljive, cvor* koren)|

\noindent
koja za promenljivu \verb|ime_promenljive| sa vredno\v s\' cu
\verb|vrednost_promenljive| ra\v cuna vrednost izraza koji je predstavljen
stablom \v{c}iji je koren \verb|*koren| i vra\' ca tu vrednost kao povratnu
vrednost funkcije. Ukoliko u izrazu postoji promenljiva \v{c}ija vrednost
nije zadata, ispisuje se poruka o gre\v{s}ci i vra\'ca vrednost 0.

Na primer, ako je \verb|izraz| pokaziva\v{c} na koren stabla koje opisuje navedni izraz,
onda se pozivom funkcije

\verb|vrednost('x', 1, izraz)|.

\noindent
dobija vrednost 5.

Ako je \verb|izraz| pokaziva\v{c} na koren stabla koje opisuje navedni izraz,
onda se pozivom funkcije

\verb|vrednost('y', 2, izraz)|.

dobija poruka:

\verb|Promenljiva x nije definisana|

i povratna vrednost 0.

\item[-]
Napisati funkcije za ispis u prefiksnom i u infiksnom poretku
drveta koje opisuje izraz.
\end{itemize}

Podrazumevati da su svi izrazi koji se dobijaju kao argumenti
ispravno formirani.

\item
U datotekama \v cija se imena zadaju sa standardnog ulaza nalaze se re\v ci
koje su leksikografski sortirane (unutar svake datoteke). Napisati program koji
re\v ci iz ovih datoteka sme\v sta sortirano u tre\'cu datoteku (\v{c}ije se
ime tako{\d}e zadaje sa standardnog ulaza). Ne koristiti dinami\v{c}ki
alociranu memoriju, niti funkcije za pozicioniranje unutar datoteka.
Pretpostaviti da su sve re\v{c}i du\v{z}ine manje od 100 karaktera.

\end{enumerate}



\section{Programiranje II, Zavr\v{s}ni ispit, septembar 2006}



\begin{enumerate}

\item
Napisati program koji formira sortiranu listu od niza celih brojeva koji se
unose sa standardnog ulaza. Oznaka za kraj unosa je $0$. Napisati funkcije za
formiranje \v cvora liste, ubacivanje elementa u ve\' c sortiranu listu,
ispisivanje elemenata liste i osloba\d anje liste.

\item
Iz datoteke \v cije se ime zadaje kao argument komandne linije
\v cita se prvo dimenzija kvadratne matrice $n$, pa zatim elementi
matrice. Prostor za matricu alocirati dinami\v cki. Napisati program
koji:
\begin{enumerate}
\item
pronalazi sve elemente matrice A koji su jednaki zbiru svih svojih susednih
elemenata i \v stampa ih u obliku (broj vrste, broj kolone, vrednost elementa).
Pod susednim elementima elementa $a[i][j]$ podrazumevamo elemente $a[i-1][j-1]$,
$a[i-1][j]$, $a[i-1][j+1]$, $a[i][j-1]$, $a[i][j+1]$, $a[i+1][j-1]$, $a[i+1][j]$ i
$a[i+1][j+1]$ (ako postoje).
\item
nalazi i \v stampa sve \v cetvorke oblika\\
\verb"( A(i,j), A(i+1,j), A(i,j+1),A(i+1,j+1) )" u kojima su svi elementi me\d usobno
razli\v citi.
\end{enumerate}

\item
Napisati program koji formira ure\d eno binarno stablo bez ponavljanja
elemenata. Elementi stabla su celi brojevi i unose se sa ulaza, a oznaka
za kraj unosa je $0$. Napisati funkciju koja proverava da li je uneto
stablo uravnote\v zeno. Stablo je uravnote\v zeno ako za svaki \v cvor
stabla va\v zi da mu se visina levog i desnog podstabla razlikuju
najvi\v se za jedan.
\end{enumerate}




\section{Programiranje II, Zavr\v{s}ni ispit, oktobar 2006}



\begin{enumerate}

\item
Napisati funkciju koja za dva niza karaktera proverava da li je:
\begin{itemize}
\item [(a)]
prvi podniz drugog (elementi prvog niza se ne moraju nalaziti na susednim
pozicijama u drugom nizu).
\item [(b)]
prvi podstring drugog (karakteri prvog stringa se moraju nalaziti na susednim
pozicijama u drugom stringu).
Ako jeste funkcija treba da vrati poziciju prvog niza u drugom odnosno $-1$ ako
nije.
\end{itemize}

\item
Napisati program koji simulira rad sa stekom. Napraviti funkcije push (za
ubacivanje elementa u stek), pop (za izbacivanje elementa iz steka) i
funkciju peek (koja na standardni izlaz ispisuje vrednost elementa koji se nalazi
na vrhu steka).

\item
Napisati program koji formira binarno uredjeno stablo. Napisati funkcije za:
\begin{itemize}
\item [(a)]
ubacivanje elementa u stablo, ispisivanje stabla, oslobadjanje stabla
\item [(b)]
sabiranje elemenata u listovima stabla, izra\v cunavanje ukupnog broja \v cvorova
stabla i izra\v cunavanje dubine stabla.
\end{itemize}
\end{enumerate}




\section{Programiranje 2, Zavr\v{s}ni ispit, januarski rok, 2007}


\begin{enumerate}
\item Napisati program koji za uredjeno binarno stablo
ispisuje sve listove (list je \v cvor stabla koji nema potomaka).
Napisati funkcije za kreiranje \v cvora,
uno\v senje elementa u stablo, ispis stabla i oslobadjanje stabla.

Na primer: za stablo

\begin{verbatim}

        5
      /   \
     3     6
   /   \
  2     4

\end{verbatim}

program treba da ispi\v se: $2$ $4$ $6$.

\item
Napisati funkciju koja za dva niza karaktera proverava da li je
prvi podstring drugog (elementi prvog stringa se moraju nalaziti na susednim
pozicijama u drugom stringu).

\item
Iz datoteke \v cije se ime zadaje kao argument komandne linije
\v cita se prvo dimenzija kvadratne matrice $n$, pa zatim elementi
matrice. Prostor za matricu alocirati dinami\v cki. Napisati program
koji ra\v cuna sumu elemenata matrice koji se nalaze iznad glavne dijagonale.

Na primer, za matricu:
\begin{verbatim}
1 2 3
4 5 6
7 8 9
\end{verbatim}

program izra\v cunava sumu $2+3+6$ i ispisuje vrednost $12$.

\end{enumerate}



\section{Programiranje 2, Zavr\v{s}ni ispit, februarski rok, 2007}


\begin{enumerate}
\item Napisati program koji za uredjeno binarno stablo ra\v cuna ukupan broj
listova u stablu (list je \v cvor stabla koji nema potomaka).
Napisati funkcije za kreiranje \v cvora,
uno\v senje elementa u stablo, ispis stabla i oslobadjanje stabla.

Na primer: za stablo

\begin{verbatim}

        5
      /   \
     3     6
   /   \
  2     4

\end{verbatim}

program treba da ispi\v se: $3$.

\item
Napisati funkciju koja za dva niza karaktera proverava da li je
jedan string permutacija drugog (jedan string je permutacija drugog stringa ako
je od njega dobijen samo preme\v stanjem njegovih karaktera - bez ikakvog brisanja
ili dodavanja karaktera).

Npr. za stringove \emph{abc} i \emph{cba} funkcija treba da ispi\v se poruku
da stringovi jesu permutacija. Za stringove \emph{aab} i \emph{ab} funkcija treba
da ispi\v se poruku da stringovi nisu permutacija.

\item
Iz datoteke \v cije se ime zadaje kao argument komandne linije
\v cita se prvo dimenzija kvadratne matrice $n$, pa zatim elementi
matrice. Prostor za matricu alocirati dinami\v cki. Napisati program
koji ra\v cuna suma elemenata matrice koji se nalaze na sporednoj dijagonali.

Na primer, za matricu:
\begin{verbatim}
1 2 3
4 5 6
7 8 9
\end{verbatim}

program izra\v cunava sumu $3+5+7$ i ispisuje vrednost $15$.
\end{enumerate}

