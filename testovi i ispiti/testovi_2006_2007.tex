\chapter{Testovi i ispiti 2006/07}
% *************************************************************
% *************************************************************
% *************************************************************

\section{Test 1, 17.04.2007. Grupa 1}


\begin{enumerate}

\item
Neka je, za zadati ceo broj $n$, $n_1$ proizvod cifara broja $n$,
$n_2$ proizvod cifara broja $n_1$, $\ldots$,
$n_k$ proizvod cifara broja $n_{k-1}$, pri \v cemu je
$k$ najmanji prirodan broj za koji je $n_k$ jednocifren. Na primer:
\begin{itemize}
\item za $n=10$, $n_1 = 1*0 = 0$, zna\v ci $k=1$
\item za $n=25$, $n_1 = 2*5 = 10$, $n_2 = 1*0 = 0$, zna\v ci $k=2$
\item za $n=39$, $n_1 = 3*9 = 27$, $n_2 = 2*7 = 14$, $n_3 = 1*4 = 4$,
zna\v ci $k=3$
\end{itemize}
Napisati: (a) rekurzivnu; (b) iterativnu funkciju koja za dato $n$
ra\v cuna $k$. Zadatak re\v siti bez kori\v s\' cenja nizova.

\item
Napisati program koji sa standardnog ulaza u\v citava pozitivne cele
brojeve dok ne u\v cita nulu kao oznaku za kraj. Na standardni izlaz
ispisati koji broj se pojavio najvi\v se puta me\d u tim brojevima.
Na primer, ako se na ulazu pojave brojevi:
\verb+2 5 12 4 5 2 3 12 15 5 6 6 5+
program treba da vrati broj $5$.
Zadatak re\v siti kori\v{s}\'cenjem dinami\v cke realokacije.

\item
Napisati program koji iz datoteke, \v cije se ime zadaje kao prvi
argument komandne linije, \v cita prvo dimenziju kvadratne matrice $n$,
a zatim elemente matrice (pretpostavljamo da se u datoteci nalaze brojevi
pravilno raspore\d eni,
odnosno da za dato $n$, sledi $n \times n$ elemenata matrice). Matricu
dinami\v cki alocirati.
Nakon toga, na standardni izlaz ispisati redni broj kolone koja ima
najve\' ci zbir elemenata.
Na primer, za datoteka sa sadr\v{z}ajem:
\vspace*{-8mm}
\begin{center}
\begin{verbatim}
3
1 2 3
7 3 4
5 3 1
\end{verbatim}
\end{center}
\vspace*{-3mm}
program treba da ispi\v se redni broj $0$ (jer je suma elemenata u nultoj
koloni \verb|1 + 7 + 5 = 13|, u prvoj \verb|2 + 3 + 3 = 8|, u drugoj
\verb|3 + 4 + 1 = 8|).
\end{enumerate}





\section{Test 1, 17.04.2007. Grupa 2}


\begin{enumerate}
\item
Neka je, za zadati ceo broj $n$, $n_1$ suma cifara broja $n$,
$n_2$ suma cifara broja $n_1$, $\ldots$, $n_k$ suma cifara broja
$n_{k-1}$, pri \v cemu je $k$ najmanji prirodan broj za koji je
$n_k$ jednocifren. Na primer:
\begin{itemize}
\item za $n=10$, $n_1 = 1+0 = 1$, zna\v ci $k=1$
\item za $n=39$, $n_1 = 3+9 = 12$, $n_2 = 1+2 = 3$, zna\v ci $k=2$
\item za $n=595$, $n_1 = 5+9+5 = 19$, $n_2 = 1+9 = 10$, $n_3 = 1+0 = 1$,
zna\v ci $k =3$
\end{itemize}
Napisati: (a) rekurzivnu; (b) iterativnu funkciju koja za dato
$n$ ra\v cuna $k$. Zadatak re\v siti bez kori\v s\' cenja nizova.

\item
Napisati program koji sa standardnog ulaza u\v citava pozitivne cele
brojeve dok ne u\v cita nulu kao oznaku za kraj. Na standardni izlaz
ispisati koji broj se pojavio najvi\v se puta me\d u tim brojevima.
Na primer, ako se na ulazu pojave brojevi:
\verb+2 5 12 4 5 2 3 12 15 5 6 6 5+
program treba da vrati broj $5$.
Zadatak re\v siti kori\v{s}\'cenjem dinami\v cke realokacije.

\item
Napisati program koji iz datoteke, \v cije se ime zadaje kao prvi
argument komandne linije, \v cita prvo dimenziju kvadratne matrice $n$,
a zatim elemente matrice (pretpostavljamo da se u datoteci nalaze brojevi
pravilno raspore\d eni,
odnosno da za dato $n$, sledi $n \times n$ elemenata matrice). Matricu
dinami\v cki alocirati.
Nakon toga, na standardni izlaz ispisati redni broj vrste koja ima
najve\' ci zbir elemenata.
Na primer, za datoteka sa sadr\v{z}ajem:
\vspace*{-8mm}
\begin{center}
\begin{verbatim}
3
1 2 3
7 3 4
5 3 1
\end{verbatim}
\end{center}
\vspace*{-2mm}
program treba da ispi\v se redni broj $1$ (jer je suma elemenata u nultoj
vrsti \verb|1 + 2 + 3 = 6|, u prvoj \verb|7 + 3 + 4 = 14|, u drugoj
\verb|5 + 3 + 1 = 9|).
\end{enumerate}



\section{Test 2, 19.06.2007., grupa 1.}


\begin{enumerate}

\item Napisati program koji implementira red pomo\' cu jednostruko
povezane liste (\v cuvati pokaziva\v ce i na po\v cetak i na kraj reda).
Sa standardnog ulaza se unose brojevi koje sme\v stamo u red sa
nulom ($0$) kao oznakom za kraj unosa.
Napisati funkcije za:

\begin{itemize}
\item Kreiranje elementa reda;
\item Ubacivanje elementa na kraj reda;
\item Izbacivanje elementa sa po\v cetka reda;
\item Ispisivanje reda;
\item Osloba\d anje reda.
\end{itemize}

\item Napisati program koji formira ure\d eno binarno stablo koje sadr\v zi
cele brojeve. Brojevi se unose sa standardnog ulaza sa nulom kao oznakom
za kraj unosa.
Napisati funkcije za:

\begin{itemize}
\item Ubacivanje elementa u ure\d eno stablo (bez ponavljanja elemenata);
\item Ispisivanje stabla u inorder (infix) redosledu;
\item Osloba\d anje stabla;
\item Odre\d ivanje najmanje dubine lista stabla.

Na primer, za stablo:

\begin{verbatim}
    5
   / \
  3   7
 / \
2   4
\end{verbatim}

funkcija treba da vrati $2$ (jer se na toj dubini nalazi list $7$).
\end{itemize}

\item Napisati program koji sa standardnog ulaza u\v citava podatke o
studentima tako \v sto za svakog studenta dobijamo prezime
(karakterska niska od najvi\v se $30$ karaktera) i broj indeksa (ceo broj).
Pretpostavka je da studenata nema vi\v se od $100$.

Sortirati ovaj niz studenata po prezimenima studenata pozivom standardne
funkcije {\tt qsort} i ispisati ih na standardni izlaz.

\end{enumerate}


\section{Test 2, 19.06.2007., grupa 2.}


\begin{enumerate}

\item Napisati program koji implementira stek pomo\' cu jednostruko
povezane liste. Sa standardnog ulaza se unose brojevi koje sme\v
stamo u stek sa nulom ($0$) kao oznakom za kraj unosa. Napisati
funkcije za:

\begin{itemize}
\item Kreiranje elementa steka;
\item Ubacivanje elementa na po\v cetak steka;
\item Izbacivanje elementa sa po\v cetka steka;
\item Ispisivanje steka;
\item Osloba\d anje steka.
\end{itemize}

\item Napisati program koji formira ure\d eno binarno stablo koje sadr\v zi
cele brojeve. Brojevi se unose sa standardnog ulaza sa nulom kao
oznakom za kraj unosa. Napisati funkcije za:

\begin{itemize}
\item Ubacivanje elementa u ure\d eno stablo (bez ponavljanja elemenata);
\item Ispisivanje stabla u preorder (prefix) redosledu;
\item Osloba\d anje stabla;
\item Odre\d ivanje najve\' ce dubine lista stabla.

Na primer, za stablo:

\begin{verbatim}
    5
   / \
  3   7
 / \
2   4
\end{verbatim}

funkcija treba da vrati $3$ (jer se na toj dubini nalazi list $4$).
\end{itemize}

\item Napisati program koji sa standardnog ulaza u\v citava podatke o
studentima tako \v sto za svakog studenta dobijamo prezime
(karakterska niska od najvi\v se $30$ karaktera) i broj indeksa (ceo broj).
Pretpostavka je da studenata nema vi\v se od $100$.

Sortirati ovaj niz studenata po brojevima indeksa studenata
pozivom standardne funkcije {\tt qsort} i ispisati ih na
standardni izlaz.

\end{enumerate}


\section{Programiranje 2, Zavr\v{s}ni ispit, juni 2007.}


\begin{enumerate}


\item
Napisati program koji sa standardnog ulaza u\v citava dve
karakterske niske du\v zine do $20$ karaktera i proverava da li je
prva niska podniz druge niske (odnosno da li se svi karakteri prve
niske nalaze u drugoj nisci, ne obavezno u istom redosledu).
Napisati funkciju koja vra\' ca $1$ ako je prva niska podniz
druge, odnosno $0$ u suprotnom.

\item
Napisati program koji za datoteku \v cije se ime zadaje kao prvi
argument komandne linije odre\d uje i ispisuje re\v c koja se
pojavljuje najvi\v se puta u toj datoteci (pretpostavljamo da su
re\v ci du\v zine najvi\v se $20$ karaktera). Zadatak re\v siti
kori\v s\' cenjem dinami\v cke realokacije.


\item
Napisati program koji formira sortiranu listu od celih brojeva
koji se unose sa standardnog ulaza. Oznaka za kraj unosa je $0$.

Napisati funkcije za:

\begin{itemize}
\item Formiranje \v cvora liste,
\item Ubacivanje elementa u ve\' c sortiranu listu,
\item Ispisivanje elemenata liste u rastu\' cem poretku u vremenu $O(n)$,
\item Ispisivanje elemenata liste u opadaju\' cem poretku u vremenu $O(n)$,
\item Osloba\d anje liste.
\end{itemize}

Napomena: potrebno je da lista bude takva da funkcije za ispis
liste u rastu\' cem i u opadaju\' cem poretku \emph{ne koriste
rekurziju niti dodatnu alociranu memoriju} a rade u vremenu
$O(n)$.
\end{enumerate}



\section{Programiranje 2, Zavr\v{s}ni ispit, septembar 2007.}




\begin{enumerate}
\item Napisati program koji formira ure\d eno binarno stablo bez
ponavljanja elemenata. Elementi stabla su celi brojevi i unose se sa
standardnog ulaza (oznaka za kraj unosa je 0). Napisati funkcije za
kreiranje elementa stabla, umetanje elementa u ure\d eno stablo,
\v stampanje stabla, brisanje stabla i odre\d ivanje sume elemenata u
listovima stabla.

\item Napisati program koji simulira rad sa stekom. Napisati funkcije
\emph{push} (za ubacivanje elementa u stek), \emph{pop} (za izbacivanje
elementa iz steka) i funkciju \emph{peek} (koja na standardni izlaz
ispisuje vrednost elementa koji se nalazi na vrhu steka - bez brisanja tog
elementa iz steka).

\item Napisati program koji sa standardnog ulaza u\v citava tekst nepoznate
du\v zine i sme\v sta ga u (dinami\v cki) niz karaktera. Oznaka za kraj
unosa je uneta 0.
Nakon toga ispisati uneti tekst na standardni izlaz po $10$ karaktera u
redu.
Zadatak re\v siti kori\v s\' cenjem dinami\v cke realokacije.

Na primer, ako je uneto:

"Ja polazem ispit iz programiranja2"

program treba da ispise:

\begin{verbatim}
Ja polazem
ispit iz p
rogramiran
ja2
\end{verbatim}

\item Napisati program koji sa standardnog ulaza unosi prvo broj artikala
a zatim i podatke o artiklima
(ime artikla - karakterska niska du\v zine do $20$ karaktera i cena artikla
 - ceo broj), sortira ih po ceni (pozivom funkcije qsort) i
nakon toga (pozivom funkcije bsearch) odre\d uje naziv artikla
\v ciju cenu korisnik zadaje sa standardnog ulaza.
\end{enumerate}


\section{Programiranje 2, Zavr\v{s}ni ispit, oktobar 2007.}



\begin{enumerate}

\item Napisati program koji formira ure\d eno binarno stablo bez
ponavljanja elemenata \v ciji elementi su imena studenata i brojevi
njihovih indeksa (struktura). Pretpostavka je da ime studenta nije du\v ze
od $30$ karaktera i da je indeks dat kao ceo broj.
Napisati program koji sa standardnog ulaza \v cita podatke o studentima,
sme\v sta ih u stablo (ure\d eno prema brojevima indeksa) i \v stampa
podatke o studentima u opadaju\' cem poredku prema brojevima indeksa.
Oznaka za kraj unosa je kada se umesto imena studenta unese niska $"kraj"$.
Napisati funkcije za kreiranje \v cvora stabla, umetanje studenta u
stablo, brisanje stabla i ispis stabla na opisan na\v cin.

\item Napisati program koji simulira red u studentskoj poliklinici.
Napisati funkcije \emph{add} (za ubacivanje studenta na kraj reda),
\emph{get} (za izbacivanje studenta sa po\v cetka reda) i funkciju
\emph{peek} (koja na standardni izlaz ispisuje ime studenta koji se
nalazi na po\v cetku reda - bez brisanja tog studenta iz reda).
Podaci o studentu se sastoje od imena studenta (karakterska niska du\v zine
ne ve\' ce od $30$) i broja indeksa studenta (ceo broj).

\item Napisati program koji sa standardnog ulaza u\v citava prvo
dimenzije matrice ($n$ i $m$) a zatim redom i elemente matrice (ne
postoje pretpostavke o dimenziji matrice). Nakon toga u datoteku \v cije
se ime zadaje kao prvi argument komandne linije, zapisati indekse ($i$ i $j$)
onih elemenata matrice koji su jednaki zbiru svih svojih susednih elemenata
(pod susednim elementima podrazumevamo okolnih $8$ polja matrice ako postoje).
Na primer, za matricu:
\begin{verbatim}
1 1 2 1 3
0 8 1 9 0
1 1 1 0 0
0 3 0 2 2
\end{verbatim}
polja na pozicijama [1][1], [1][3], [3][2] i [3][4] zadovoljavaju tra\v zeni
uslov pa u datoteku treba upisati:
\begin{verbatim}
1 1
1 3
3 2
3 4
\end{verbatim}
\item Napisati program koji sa standardnog ulaza unosi prvo broj studenata
a zatim i podatke o studentima
(ime studenata - karakterska niska du\v zine do $30$ karaktera i
broj indeksa studenta - ceo broj), sortira ih po imenu studenta
leksikografski (pozivom funkcije qsort) i
nakon toga (pozivom funkcije bsearch) odre\d uje broj indeksa studenta
\v cije ime korisnik zadaje sa standardnog ulaza.
\end{enumerate}

