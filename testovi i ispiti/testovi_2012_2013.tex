% *************************************************************
% *************************************************************
% *************************************************************
\chapter{Testovi i ispiti 2012/13}
% *************************************************************
% *************************************************************
% *************************************************************

\newpage

%---------------------------------------------------------------------

\section{I smer, Programiranje 2 2012/2013, Drugi test}
\subsection{(I Grupa)}

\textbf{NAPOMENA:} Na desktopu napraviti direktorijum
\verb|InicijaliAsistenta_ImePrezime_BrojIndeksa_Grupa|,
\\ npr. \verb|AK_MarkoMarkovic_mi12201_1|. Zadatke \v cuvati u ovom
direktorijumi pod nazivom \textbf{1.c, 2.c}.  U slu\v caju da zadaci
budu sa\v cuvani na nekom drugom mestu ili poddirektorijumu ne\'ce
biti pregledani!

\begin{z} Napisati funkciju koja sortira slova unutar niske karaktera i
  ispisuje ih na standardni izlaz. Napisati program koji proverava da
  li su dve niske karaktera koje se zadaju kao argumenti komandne
  linije anagrami. Dve niske su anagrami ako se sastoje od istog broja
  istih karaktera. Program treba na standardni izlaz da ispi\v se
  $true$ ili $false$.
\end{z}

\begin{verbatim}
Primer 1:    ./anagram miroljub ljubomir          bijlmoru bijlmoru true
Primer 2:    ./anagram vatra trava                aartv aartv true
Primer 3:    ./anagram racunar racun              aacnrru acnru false
Primer 4:    ./anagram program gram               agmoprr agmr false
\end{verbatim}

\begin{z} Napisati funkciju
  \verb|int count_zero_pairs(unsigned x)| koja broji koliko se puta
  kombinacija $00$ (dve uzastopne nule) pojavljuje u binarnom zapisu
  celog neozna\v cenog broja (ako se u bitskoj reprezentaciji nekog
  broja jave tri uzastopne nule, sredi\v snja se broji dva puta, u
  levom i desnom paru). Napisati program koji za broj iz prvog
  argumenta komandne linije na standardni izlaz ispisuje rezultat
  funkcije.
\end{z}

\begin{verbatim}
Primer 1:   ./zeros 487                                  23
Primer 2:   ./zeros 255                                  23
Primer 3:   ./zeros 3567                                 19
Primer 4:   ./zeros 1024                                 29
\end{verbatim}

\vskip 2cm

\subsection{(II Grupa)}

\textbf{NAPOMENA:} Na desktopu napraviti direktorijum
\verb|InicijaliAsistenta_ImePrezime_BrojIndeksa_Grupa|,
\\ npr. \verb|AK_MarkoMarkovic_mi12201_2|. Zadatke \v cuvati u ovom
direktorijumi pod nazivom \textbf{1.c, 2.c}.  U slu\v caju da zadaci
budu sa\v cuvani na nekom drugom mestu ili poddirektorijumu ne\'ce
biti pregledani!

\begin{z} Sa standardnog ulaza unosi se broj re\v{c}i \verb|n|,
za kojim sledi \verb|n| re\v{c}i (u svakom redu po jedna).  Nakon
unosa re\v ci unosi se i broj \verb|q|.  Re\v{c}i nisu du\v{z}e od 20
karaktera.  U\v{c}itati re\v{c}i u niz, a zatim ih sortirati. Ispisati
re\v c na poziciji \verb|q|.  Kriterijum sortiranja je suma
\verb|ascii| kodova slova svake re\v{c}i, poredak je rastu\'{c}i.
\end{z}

\begin{verbatim}
Primer 1:                Primer 2:                Primer 3:                   Primer 4:
3                        4                        4                           2
programiranje            test                     aaaaa                       funkcija
alokacija                kolokvijum               zzzzz                       C
memorija                 ispit                    bbbbb                       1
3                        usmeni                   ttttt
                         1                        2
--------------------------------------------------------------------------------------------
programiranje            test                     bbbbb                       C
\end{verbatim}

\begin{z} Napisati funkciju
  \verb|unsigned swap_pair(unsigned x, int i, int j)| koja razmenjuje
  vrednosti bitova na pozicijama \verb|i| i \verb|j|, i vra\'ca
  rezultat kao povratnu vrednost. Bit sa najmanjom te\v zinom nalazi
  se na poziciji 0, bit do njega na poziciji 1, itd... Napisati i
  program koji testira ovu funkciju. Sa standardnog ulaza se unose
  redom broj kome se invertuju bitovi, i pozicije bitova, a rezultat
  funkcije se ispisuje na standardni izlaz.
\end{z}

\begin{verbatim}
Primer 1:                Primer 2:                Primer 3:                   Primer 4:
159 0 1                  48 4 0                   255 1 9                     249 10 5
159                      33                       765                         1241
\end{verbatim}


\section{Programiranje II, Zavr\v{s}ni ispit, jun 2013.}


\begin{enumerate}
\item Trougao je zadat duzinama svoje tri stranice. Napraviti strukturu koja opsuje trougao.
Napisati funkciju koja ra\v cuna povr\v sinu trougla.
Kao argumenti komadne linije zadati su podaci o dva trougla. Ispisati koji trougao ima ve\'cu povr\v sinu
(na standarni izlaz ispisati \verb|prvi| ili \verb|drugi|).

\item Sa standardnog ulaza se ucitava broj n a zatim i niz od n niski(svaka niska je maksimalne duzine 20 karaktera, duzina niza n
nije unapred poznata niti ogranicena).
Sortirati taj niz u opadajucem poretku pri cemu je kriterijum poredenja broj samoglasnika u reci.
Ispisati sortirani niz na standardni izlaz.
U slucaju greske ispisati "-1" na standardni izlaz.

\item Napisati funkciju unsigned int fja(unsigned int x) koja vraca broj dobijen od broja x kada se prvih 8 bitova broja
(bitovi na najvecim tezinama) postave na 0, poslednjih 4 bita(bitovi na najmanjim tezinama) se postave na 0110,
 a ostatak broja ostaje nepromenjen. Testirati pozivom u main-u.

\item Napisati funkciju \verb|void umetni(cvor* lista)| koja izmedju svaka dva el-
ementa u listi umece element koji predstavlja razliku susedna dva.

\item Napisati funkciju int prebroj(cvor* drvo) koja vraca broj elemenata stabla drvo koji su iste parnosti kao oba svoja sina(
sva tri parna ili neparna).
Ukoliko je cvor list ili ima samo jednog sina ne ulazi u zbir.

\end{enumerate}



\section{Programiranje II, Zavr\v{s}ni ispit, jun2 2013.}



NAPOMENA: Na desktopu napraviti direktorijum sa imenom \verb|inicjaliAsistenta_ImePrezime_brojIndeksa|.
U tom direktorijumu \v cuvati zadatke -- 1.c, 2.c, 3.c, 4.c. 5.c
Na desktopu se nalazi folder \verb|p2_jun| sa datotekama za rad sa listama i drvetom. Date datoteke potrebno je koristiti
po pravilima za rad sa vi\v se datoteka.

\begin{z}Ime datoteke dato je kao argument komandne linije. U datoteci zameniti svaku cifru slede\' com \verb|0->1, 1->2,...|.
Broj 9 zameniti sa \verb|0|. U slu\v caju gre\v ske na standardni izlaz ispisati \verb|-1|.
\end{z}
\begin{verbatim}
Primer 1:                |     Primer 2:                      Primer 3:
./a.out dat.txt          |     ./a.out zad                    ./a.out test.dat
dat.txt: 123abc789       |     zad: 9202jsshsh 120            test.dat:  kazablanka je 1967. bila
                         |          qoqo 2782 20                         kako uhvatiti ziravu 89 puta
dat.txt: 234abc890       |                                               90 puta napisati euklidov algoritam
                         |      zad: 0313jsshsh 231
______________________   |           qoqo 3893 31
Primer 4:                |                                    test.dat: kazablanka je 2078. bila
./a.out                  |                                              kako uhvatiti ziravu 90 puta
                         |                                              01 puta napisati euklidov algoritam
 -1                      |
\end{verbatim}

\begin{z} Kao argumenti komandne linije dati su celi brojevi $a, b, c, d, e$. Binarnom pretragom na\' ci ceo broj na intervalu $[d, e]$
nulu funkcije $a\cdot x^2 + b\cdot x + c$ sa ta\v cno\v s\'cu \verb|0.0001|. U slu\v caju gre\v ske na standardni izlaz napisati \verb|-1|.
Ako su argumenti komandne linije ispravno dati podrazumevati da funkcija ima ta\v cno jednu nulu na datom intervalu.
Na standardni izlaz ispisati na\d enu vrednost. Broj ispisati sa 3 decimale.
\end{z}
\begin{verbatim}
Primer 1:                Primer 2:         Primer 3:                     Primer 4:
./a.out 5 8 -4 0 1       ./a.out 4         ./a.out 10 -2 -15 -2 0        ./a.out -3 6 10 0 4

0.400                    -1                -1.129                        3.082
\end{verbatim}

\begin{z}
 Napisati funkciju  \verb|unsigned int f3(unsigned int x)| koja vra\' ca broj koji predstavlja odraz u ogledalu
polaznog broja \verb|x|. Testirati pozivom u main funkciji -- sa standardnog ulaza se unosi broj, rezultat ispisati na standarni izlaz.
 Na primer, ako je ulaz broj \v ciji je binarni zapis \verb|00101|, izlaz je broj \v ciji je binarni zapis \verb|10100|.
\end{z}
\begin{verbatim}
Primer 1:           Primer 2:            Primer 3:          Primer 4:
3                   1234                 -90                82720000

3221225472          1260388352           1260388352         11302688
\end{verbatim}

\begin{z}
Napisati funkciju \verb|Cvor* izbaci(Cvor *lista1, Cvor *lista2)|  koja iz \verb|liste1|
izbacuje sve elemente koji se pojavljuju u \verb|listi2|. Testirati funkciju pozivom u main-u,
sa standarnog ulaza se u\v citaju elementi prve liste sve dok se ne unese 0.
Potom se u\v citavaju elementi druge liste sve dok se ne u\v cita 0. Elemente liste dodavati na kraj.
Potom pozvati funkciju i novodobijenu listu ispisati na standarni izlaz.
Dozvoljeno je pravljenje nove liste.
\end{z}
\begin{verbatim}
Primer 1:          Primer 2:             Primer 3:            Primer 4:
1 3 5 0            3 -90 2 8 7 0         7 8 0                0
3 4 6 7 0          8 3 -4 0              0                    10 34 5 67 1 2 0

izlaz: 1 5         izlaz: -90 2 7        izlaz: 7 8           izlaz:
\end{verbatim}


\begin{z}
Napisati funkciju \verb|int ravnomerno_izbalansirano(Cvor *stablo)| koja proverava da li je stablo ravnomerno izbalansirano.
 Stablo je ravnomerno izbalansirano ako za svaki \v cvor va\v zi da je pozitivna razlika izme\d u dubine levog i dubine desnog podstabla najvi\v se \verb|1|.
Testirati funkciju pozivom u main-u, stablo se u\v citava sve dok se ne unese 0. Ukoliko jeste izbalansirano ispisati \verb|1|, a u suprotnom \verb|0|.
\end{z}
\begin{verbatim}
Primer 1:                             Primer 2:                    Primer 3:            Primer 4:

     10                                  10                        ulaz: 0                 10
    /  \                                /  \                       izlaz: 1               /  \
   7    20                             7    11                                           7    20
  / \   / \                           /                                                 / \
 5  8  15  23                        5                                                 5   8
                                    /
ulaz: 10 7 20 5 8 15 23 0          4                                                   ulaz: 10 7 5 8 20 0
izlaz: 1                                                                               izlaz: 1
                                   ulaz: 10 7 11 5 4 0
                                   izlaz: 0
\end{verbatim}



%----------------------------------------------------------------
\section{Programiranje II, Zavr\v{s}ni ispit, januar 2014.}
%----------------------------------------------------------------


\begin{z}
Argumenti komadne linije su ime datoteke i jedan ceo broj \verb|p| (\verb|p| $>$ 0).
U datoteku \verb|izlaz.txt| prepisati svaki \verb|p|-ti karakter iz ulazne datoteke.
U slu\v caju gre\v ske ispisati -1.
\end{z}
\begin{verbatim}
Primer 1:                                      Primer 2:
./a.out ulaz 3                                 ./a.out ulaz 10
ulaz: danas pisemo                             ulaz: Na Sretenje 1804. godine podignut je
      programe razne                                 Prvi srpski ustanak, a 1835.
      i mnogo nam je zanimljivo                      donet prvi demokratski ustav Srbije.

izlaz.txt: n sorreae o meami                   izlaz.txt: jogr aodib
-------------------------------------------------------------------------------------------

Primer 3:                                      Primer 4:
./a.out dat.txt -3                             ./a.out

-1                                             -1
\end{verbatim}


\begin{z}
 U datoteci \verb|slike.txt| nalaze se podaci o slikama -- ime slike (maksimalne du\v zine 20 karaktera,
 ne sadr\v zi praznine, dozvoljeno je kori\v scenje specifikatora \verb|%s| za unos imena) i
 veli\v cina u kilobajtima (neozna\v cen ceo broj). Definisati strukturu koja opisuje sliku.
 Sortirati niz slika u opadaju\'cem poretku pri \v cemu je kriterijum pore\d enja veli\v cina slike.
 Na standarni izlaz ispisati imena slika sortiranog niza. Broj slika u datoteci nije unapred
 poznat (koristiti dinami\v cku alokaciju). U slu\v caju gre\v ske ispisati -1.
\end{z}
\begin{verbatim}
Primer 1:                   Primer 2:                   Primer 3:           Primer 4:
slike.txt:                  slike.txt:                  slike.txt:          slike.txt:
krug 200                    Klod_Mone 400               p1 300              oblak 200
trougao 450                 Edvard_Munk 550             p2 400              livada.jpg 300
Ojlerov_krug 320            Salvador_Dali 320           p3 320              suma 220
Pashova_teorema 220         Gustav_Klimt 432                                planina 340
paralelogram 330            Rembrant 578                izlaz:
                            Francisko_Goja 349          p2                  izlaz:
izlaz:                      Paja_Jovanovic 490          p3                  planina
trougao                     Uros_Predic 390             p1                  livada.jpg
paralelogram                                                                suma
Ojlerov_krug                izlaz:                                          oblak
Pashova_teorema             Rembrant
krug                        Edvard_Munk
                            Paja_Jovanovic
                            Gustav_Klimt
                            Klod_Mone
                            Uros_Predic
                            Francisko_Goja
                            Salvador_Dali
\end{verbatim}

\begin{z}
Napisati funkciju |\verb|unsigned int f3(unsigned int x, int k, int p)|| koja u broju \verb|x| komplementira
\verb|k| bitova, po\v cev\v si od pozicije \verb|p|. Bitovi se broje sa desna na levo, po\v cev\v si od 0.
Bitovi koji se komplementiraju idu od pozicije \verb|p| do pozicije \verb|k + p - 1|.
Sa standarnog ulaza u\v citavaju se brojevi \verb|x|, \verb|k| i \verb|p|. Na standarni izlaz ispisati
izlaz funkcije \verb|f3|. U slu\v caju gre\v ske ispisati -1.
\end{z}
\begin{verbatim}
Primer 1:                   Primer 2:                   Primer 3:           Primer 4:
345 5 6                     18903 10 5                  456672 4 15         45 15 23

1689                        13879                       471656              -1
\end{verbatim}

\begin{z}
Dat je polinom reprezentovan listom, tako da \v cvorovi liste sadr\v ze koeficijente i stepene polinoma.
Dati polinom je nesre\d en, tj. mo\v ze sadr\v zati vi\v se \v clanova sa istim stepenom.
Napisati funkciju \verb|cvor* f4(cvor* lista)| koja sre\d uje polinom tako da:
\begin{itemize}
\item da se svaki stepen pojavljuje najvi\v se jedanput
\item da su svi koeficijenti koji su prisutni u listi razli\v citi od 0
\end{itemize}
U main-u se unosi broj elemenata liste, a potom i elementi liste
(svaki element liste se dodaje na kraj).
Ispisati na standarni izlaz polinom koja je dobijena sre\d ivanjem ulazne liste.
\end{z}
\begin{verbatim}
Primer 1:               Primer 2:                 Primer 3:            Primer 4:
4                       10                        4                    5
1 1                     -1 2                      3 1                  2 3
2 3                     3 3                       2 3                  1 2
-1 1                    1 2                       5 1                  4 5
1 3                     -3 3                      3 3                  -2 3
                        7 1                                            2 2
3*x^3                   2 5                       8*x^1 + 5*x^3
                        3 1                                            3*x^2 + 4*x^5
                        -1 5
                        -1 5
                        4 4

                        10*x^1 + 4*x^4
\end{verbatim}

\begin{z}
Napisati funkciju \verb|float f5(cvor* drvo)| koja ra\v cuna sumu svih elemenata drveta
takvih da su ve\'ci od sume svojih direktnih potomaka. Ne ra\v cunati \v cvorove koji
nemaju ni jednog direktog potomka.
Sa standarnog ulaza unose se elementi u drvo sve dok se ne unese 0.
Elementi su drveta su realni brojevi zapisani u jednostrukoj ta\v cnosti (koristiti \verb|float|).
Na standarni izlaz ispisati izlaz funkcije \verb|f5|. Rezultat ispisati sa 3 decimale.
Mogu\'ce je koristiti biblioteke za rad sa stablima (\verb|stbla.c| i \verb|stabla.h|),
kompajlirati sa \verb|gcc 5.c stabla.c|.
\end{z}
\begin{verbatim}
Primer 1:                                                    Primer 2:
         7.51                                                        6
       /      \                                                     /  \
      -2        8.1                                               -4    8
    /    \      /  \                                             / \   /
   -8     5    8    9.1                                        -11 5   7
   / \   / \        /                                           /
-10 -3 -1.5 6      8.5                                        -20

ulaz:                                                         ulaz:
7.51 -2 -8 5 -10 -3 -1.5 6 8.1 8 9.1 8.5 0                    6 -4 8 -11 5 7 -20 0

izlaz: 11.610                                                 izlaz: -1.000
---------------------------------------------------------------------------------------------

Primer 3:                                                  Primer 4:
       7.82                                                      -2
       /  \                                                     /   \
     -4   10                                                  -10    4
       \                                                      / \   /  \
        6                                                   -13 -4 -1   4.1

ulaz: 7.82 -4 10 6 0                                        ulaz: -2 -10 4 -13 -4 -1 4.1 0

izlaz: 7.820                                                izlaz: -8.000
\end{verbatim}

