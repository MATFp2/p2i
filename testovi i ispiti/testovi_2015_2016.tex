\chapter{Testovi i ispiti 2015/2016}

\section{I smer, Programiranje 2 2015/2016, prvi prakti\v cni test}

\subsection{Grupa I}

\begin{enumerate}
\item Kao argumenti komandne linije prosledjena su dva cela broja $k$ i $l$ ($2\le k \le l \le 10000$). Ispisati na standarni izlaz sve proste brojeve $p$ takve da vazi $k\le p\le l$. U slu\v caju gre\v ske ispisati $-1$ na standardni izlaz. 

\small
\begin{tabular}{ |l|l|l|l|l| }
\hline 
  Ulaz & ./a.out 2 10  & ./a.out 3 6 & ./a.out 10 20 & ./a.out 52 10 \\ \hline 
  Izlaz &  2 3 5 7 & 3 5 & 11 13 17 19 & -1 \\ \hline 
\end{tabular}
\normalsize

\item U datoteci ''r.txt'' nalazi se broj $n$, za kojim sledi $n$ reci ($n \ge 0$, maksimalna du\v zina re\v ci je 50 karaktera). U datoteku ''b.txt'' ispisati sve re\v ci koje predstavljaju cele brojeve u dekadnom zapisu, a u datoteku ''o.txt'' sve one koje to nisu. U slu\v caju gre\v ske ispisati $-1$ na standardni izlaz. 

\small
\begin{tabular}{ |l|l|l|l|l| }
\hline 
  Ulaz & 
  \mlcell{r.txt: \\5 baba deda 123 sd3f45 ff543g}&
  \mlcell{r.txt: \\10 1 a 22 b 333 c 4 d 5 ff } & 
  \mlcell{r.txt: \\0 } & 
  \mlcell{ }\\ 
  \hline 
  Izlaz &
  \mlcell{b.txt: \\123 \\ o.txt: \\baba deda sd3f45 ff543g} &  
  \mlcell{b.txt: \\1 22 333 4 5  \\ o.txt: \\a b c d ff } &  
  \mlcell{b.txt: \\  \\ o.txt: \\ }&  
  -1\\ 
  \hline 
\end{tabular}
\normalsize


\item Kreirati rekurzivnu funkciju \emph{int f3(int x)} koja u po\v cetnom zapisu broja $x$ izbacuje svaku neparnu cifru, ukoliko se ispred te cifre nalazi cifra 2. Kreirati program koji testira ovu funkciju, tako \v sto sa standardnog ulaza u\v citava ceo broj $x$, i na standardni izlaz ispisuje vrednost funkcije \emph{f3(x)}. U slu\v caju gre\v ske ispisati $-1$ na standardni izlaz. 

\textbf{Napomena.} Zadatak mora biti uradjen rekurzivno. Nije dozvoljeno kori\v s\' cenje stati\v ckih i globalnih promenljivih, menjanje zaglavlja funkcije i pisanje pomo\' cnih funkcija. 

\small
\begin{tabular}{ |l|l|l|l|l| }
\hline 
  Ulaz  &  2325 &  23523 & 12345 & 0 \\ \hline 
  Izlaz &  22 & 252 & 1245  & 0 \\ \hline 
\end{tabular}
\normalsize
\end{enumerate}

\subsection{Grupa II}

\begin{enumerate}
\item Kao argumenti komandne linije prosledjena su dva cela broja $k$ i $l$ ($0\le k \le l \le 10000$). Ispisati na standardni izlaz sumu cifara svih brojeva $x$, takvih da je $k\le x\le l$. U slu\v caju gre\v ske ispisati $-1$ na standardni izlaz. 

\small
\begin{tabular}{ |l|l|l|l|l| }
\hline 
  Ulaz &./a.out 0 9  & ./a.out 10 13& ./a.out 93 2145& ./a.out \\ \hline 
  Izlaz & 45 &  10 &28722  &  -1\\ \hline 
\end{tabular}
\normalsize

\item U datoteci ''u1.txt'' nalazi se broj $n_1$, za kojim sledi $n_1$ re\v ci, dok se u datoteci ''u2.txt'' nalazi broj $n_2$, za kojim sledi $n_2$ re\v ci ($n_1 \ge 0$, $n_2 \ge 0$, maksimalna du\v zina re\v ci je 50 karaktera). U datoteku ''i.txt'' upisati naizmeni\v cno re\v ci iz prve dve datoteke. U slu\v caju gre\v ske ispisati $-1$ na standardni izlaz. 

\small
\begin{tabular}{ |l|l|l|l|l| }
\hline 
  Ulaz & 
  \mlcell{u1.txt: \\5 a b c d e \\u2.txt: \\5 1 2 3 4 5}&
  \mlcell{u1.txt: \\2 a b   \\u2.txt: \\5 1 2 3 4 5} & 
  \mlcell{u1.txt: \\2 abc 123 \\u2.txt: \\0} & 
  \mlcell{}\\ 
  \hline 
  Izlaz &
  \mlcell{i.txt: \\a 1 b 2 c 3 d 4 e 5 } &  
  \mlcell{i.txt: \\a 1 b 2 3 4 5 } &  
  \mlcell{i.txt: \\abc 123 }&  
  -1\\ 
  \hline 
\end{tabular}
\normalsize

\item 
Kreirati rekurzivnu funkciju \emph{void f3(int *a, int na)} koja u po\v cetnom nizu $a$, duzine $na$, smanjuje vrednost broja za 1 ukoliko je on paran, i nakon njega se nalazi paran broj. Kreirati program koji testira ovu funkciju, tako \v sto sa standardnog ulaza u\v citava broj $na$ ($0 \le na \le 1000$), zatim vrednosti niza $a$, i na standardni izlaz ispisuje vrednosti izmenjenog niza.

\textbf{Napomena.} Zadatak mora biti uradjen rekurzivno. Nije dozvoljeno kori\v s\' cenje stati\v ckih i globalnih promenljivih, menjanje zaglavlja funkcije i pisanje pomo\' cnih funkcija. 

\small
\begin{tabular}{ |l|l|l|l|l| }
\hline 
  Ulaz  & 10 2 2 5 7 6 8 4 3 2 1 & 5 1 2 2 2 1 &  10 4 4 6 6 6 6 8 8 10 10& -1 \\ \hline 
  Izlaz & 1 2 5 7 5 7 4 3 2 1      &  1 1 1 2 1  &  3 3 5 5 5 5 7 7 9 10 & -1 \\ \hline  
\end{tabular}
\normalsize
\end{enumerate}


\subsection{Grupa III}

\begin{enumerate}
\item Preko argumenata komandne linije prosledjen je niz celih brojeva. Neka su $a$, $b$, $c$ redom minimum, maksimum i prose\v cna vrednost niza. Ispisati brojeve niza strogo ve\' ce od (a + b + c)/3. U slu\v caju gre\v ske ispisati $-1$ na standardni izlaz. 

\small
\begin{tabular}{ |l|l|l|l|l| }
\hline 
  Ulaz & ./a.out 1 2 3 4 5& ./a.out 4 11 10 5 6 15 8 9 19 8 & ./a.out 5 4 6 6 5 17 11 10 6 2 & ./a.out \\ \hline 
  Izlaz &4 5  &11 15 19 & 17 11 10&  -1\\ \hline 
\end{tabular}
\normalsize

\item  U datoteci ''dat.txt'' nalazi se re\v c $s$, zatim broj n i $n$ reci ($n \ge 0$, maksimalna du\v zina re\v ci je 50 karaktera). Napisati program koji na standardni izlaz ispisuje sve re\v ci kojima je re\v c $s$ sufiks. U slu\v caju gre\v ske ispisati $-1$ na standardni izlaz. 


\small
\begin{tabular}{ |l|l|l|l|l| }
\hline 
  Ulaz  &  
  \mlcell{dat.txt:\\ab 6 \\vjab ab a abcd feb egaab}&   
  \mlcell{dat.txt:\\ aa 7\\ a ab ba baa baaa b cba }& 
  \mlcell{dat.txt:\\ abc  0\\} &
  \\ 
  \hline 
  Izlaz &   
  vjab ab egaab &  
  baa baaa &  
  & 
  -1\\ 
  \hline 
\end{tabular}
\normalsize

\item 
Kreirati rekurzivnu funkciju \emph{void f3(int *a, int na, int suma\_prethodnih)} koja u u nizu $a$, du\v zine $na$, 
postavlja vrednost svakog broja na 0 ukoliko je ve\' ci od sume prethodnih brojeva u nizu. Kreirati program koji testira 
ovu funkciju, tako \v sto sa standardnog ulaza u\v citava broj $na$, zatim vrednosti niza $a$, i na standardni izlaz 
ispisuje izmenjeni niz. U slu\v caju gre\v ske ispisati $-1$ na standardni izlaz. 

\textbf{Napomena.} Zadatak mora biti uradjen rekurzivno. Nije dozvoljeno kori\v s\' cenje stati\v ckih i globalnih promenljivih, menjanje zaglavlja funkcije i pisanje pomo\' cnih funkcija. 

\small
\begin{tabular}{ |l|l|l|l|l| }
\hline 
  Ulaz  & 10 2 2 5 7 6 25 4 3 2 1 & 5 1 2 4 12 9 & 2 1 1& -1 \\ \hline 
  Izlaz & 0 2 0 7 6 0 4 3 2 1      & 0 0 0 0 9  &  0 1 & -1 \\ \hline  
\end{tabular}
\normalsize
\end{enumerate}

\section{Programiranje 2 2015/2016, ispit, jun}
\subsection{Grupa I}

\begin{enumerate}
\item U datoteci \textit{brojevi.txt} nalazi se niz brojeva u pokretnom zarezu. Napisati program koji prebrojava koliko puta je niz brojeva promenio uređenje iz neopadajućeg u opadajuće i obrnuto. Npr. za brojeve -1 2 2 3 1 0 početak niza je uređen neopadajuće dok od broja 1 postaje opadajuće uređen, pa je izlaz programa 1, jer je došlo do jedne promene uređenja. Ne postoji pretpostavka o du\v zini niza brojeva.

\small
\begin{tabular}{ |l|l|l|l|l| }
\hline 
  brojevi.txt & -2.0 -1.0 2.0 -1.0 2.0 -1.0 2.0  & 4.2 4.2 4.2 & 1.23 2.23 0 & 0 1 1 1 1 2 1 \\ \hline 
  Izlaz & 4 & 0 & 1 & 1 \\ \hline 
\end{tabular}
\normalsize

\item U datoteci \v cije ime se zadaje kao argument komadne linije su dati podaci o pravougaonicima. Pra\-vo\-uga\-oni\-ci su zadaci svojim imenom
   (maksimalne duzine 5 karaktera) i du\v zinama svojih stranica (dve du\v zine, realni brojevi). Sortirati pravougaonike rastu\' ce prema njihovoj povr\v sini.
   Ispisati imena pravougaonika tako sortiranog niza na standarni izlaz. 
   Ukupan broj pravougaonika nije poznat (\v citati do kraja datoteke).
   U slu\v caju gre\v ske (negativna du\v zina stranice, nepostoje\' ca datoteka, lo\v s broj argumenata) ispisati -1.

\small
\begin{tabular}{ |l|l|l|l|l| }
\hline 
  Pozivanje & ./a.out dat.txt  & ./a.out & ./a.out dat.txt & ./a.out dat.txt \\ \hline 
  Ulaz & 3 & 1 & 2 & 3 \\ \hline
  dat.txt & p1 1 1      & p0 1 2  & p1 2 4 & p1 10 2 \\
          & p2 0.2 0.3  &         & pr2 4 2 & p2 3 2 \\ 
          & p3 4 5  &   &         & p3 -1 1 \\ \hline
  Izlaz & p2 p1 p3  & -1      & p1 pr2 & -1 \\ \hline 
\end{tabular}
\normalsize


\item Jedan broj tipa \textit{int} sadrži 4 osmobitna dela, svaki od tih delova se može tumačiti kao jedan ASCII kod karaktera. Za uneto \textit{n} ($0 < n \le 100$) i niz od \textit{n} brojeva ispisati na standardni izlaz sve karaktere koji su kodirani unetim brojevima u istom redosledu. \\ Npr. za broj 1380009033 = 01010010|01000001|01000100|01001001 = 82 65 68 73 = 'R','A','D','I'. \\
U slu\v caju neispravnog ulaza (n izvan dozvoljenih granica) ispisati -1 na standardni izlaz.


\small
\begin{tabular}{ |l|l|l|l|l| }
\hline 
  
  Ulaz & 4 & 1 & 2 & 5\\ 
       & 1347571527 & 1330330153 & 1546145582 & 1298232421 \\
       & 1380011337 & & 776939823 &  1835103337\\
       & 1380011594 & &  & 1667983648 \\
       & 1159737889 & & & 1717660533 \\ 
       & & & & 1819567476 \\ \hline
  
  Izlaz & PROGRAMIRANJE 2! & OK:) & \textbackslash (O..O)/ &  Matematicki fakultet \\ \hline 
\end{tabular}
\normalsize

\item Iz datoteke \textit{lista.txt} se u\v citava lista celih brojeva. Listu modifikovati na slede\' ci na\v cin: ako su teku\' ci i njegov slede\' ci element parni, slede\' ci element uve\' cati za jedan, u suprotnom slede\' ci element smanjiti za jedan. Parnost broja se odnosi na teku\' cu, promenjenu vrednost broja. Na izlaz ispisati rezultuju\' cu listu.



\small
\begin{tabular}{ |l|l|l|l|l| }
\hline 
  
  lista.txt & 1 2 3 4 & 2 2 2 2 & 44 23 16 78 & 1 0 1 \\ \hline
  
  Izlaz & 1 1 2 5 & 2 3 1 1 & 44 22 17 77 & 1 -1 -2 \\ \hline 
\end{tabular}
\normalsize

\item Iz datoteke \textit{stablo.txt} se u\v citava stablo celih brojeva i kao argument komandne linije dat je ceo pozitivan broj k. Odrediti broj onih \v cvorova stabla \v ciji su svi potomci deljivi sa k. U slu\v caju gre\v ske (pogre\v snog broja argumenata komandne linije), na standardni izlaz ispisati -1.



\small
\begin{tabular}{ |l|l|l|l|l| }
\hline 
  
  Pozivanje & ./a.out 2 & ./a.out 3 & ./a.out & ./a.out 2 \\ \hline
  stablo.txt & 4 2 6 1  & 3 2 3 4 & 44 23 16 78 & 4 2 6 10 8 \\ \hline
  
  Izlaz & 1 & 0 & -1 & 3 \\ \hline 
\end{tabular}
\normalsize
\end{enumerate}

\subsection{Grupa II}

\begin{enumerate}


\item  Napisati program koji iz datoteke \textit{ulaz.txt} u\v citava pozitivne cele brojeve, a zatim na standardni izlaz ispisuje koji se broj pojavio najve\' ci broj puta. Nije poznato koliko maksimalno mo\v ze imati brojeva u datoteci. U slu\v caju gre\v ske (nedovoljno memorije pri alokaciji, nepostoje\' ca datoteka, lo\v s broj argumenata) na standardni izlaz ispisati -1.

\small
\begin{tabular}{ |l|l|l|l|l| }
\hline 
  ulaz.txt  & 5 1 2 3 4 5  & 1 1 1 2 2 2 2 & 2 5 2 5 2 & 100 0 100 \\ \hline 
  Izlaz & 5 & 2 & 2 & 100 \\ \hline 
\end{tabular}
\normalsize

\item U datoteci \textit{tablice.txt} nalazi se niz registarskih brojeva vozila u formatu GG BBB SS gde GG ozna\v cava skra\' cenicu grada, BBB numeri\v cki deo i SS slovni deo. Sa standardnog ulaza se u\v citava broj $n$ i na standardni izlaz ispisuje $n$-ti \v clan niza vozila sortiranog po numeri\v ckom delu registarskog broja u rastu\' cem poretku. Ukoliko su numeri\v cki delovi dva registarska broja jednaki, vr\v si se leksikografsko uređenje po oznaci grada a zatim, ukoliko su i oznake gradova jednake, leksikografski po slovnom delu. U slu\v caju gre\v ske (ako n nije validan indeks za niz, nepostoje\' ca datoteka) na standardni izlaz ispisati -1. 

\small
\begin{tabular}{ |l|l|l|l|l| }
\hline 
  tablice.txt & 
  \mlcell{ BG 100 AV \\ BG 101 MS \\ NI 256 PO \\ NS 145 RV }&
  \mlcell{BG 101 AV \\ BG 102 MS \\ NI 104 PO \\ NS 104 RV} & 
  \mlcell{BG 100 AV \\ BG 198 MS \\ NI 256 PO \\ NS 148 RV} & 
  \mlcell{BG 200 AV \\ BG 150 MS \\ NI 125 PO \\ NS 100 RV }\\ 
  \hline 
  Ulaz &
  \mlcell{2} &  
  \mlcell{3} &  
  \mlcell{2}&  
  10\\ 
  \hline
   Izlaz &
  \mlcell{NS 145 RV} &  
  \mlcell{NS 104 RV} &  
  \mlcell{BG 198 MS}&  
  -1\\ 
  \hline
\end{tabular}
\normalsize


\item Sa standardnog ulaza se unosi neozna\v cen broj $x$ i niz od 32 neozna\v cena broja. Ukoliko $i$-ti broj niza ima manje od $i$ jedinica u svom binarnom zapisu onda u $x$ na $i$-to mesto staviti bit 1, ako ima ta\v cno $i$ jedinica, u $x$ na $i$-to mesto staviti bit 0, a ukoliko ima vi\v se onda invertovati $i$-ti bit broja $x$. Na standarni izlaz ispisati novodobijeni broj.

\small
\begin{tabular}{ |l|l|l|l|l| }
\hline 
  Ulaz & 
  \mlcell{5\\
1 2 3 4 5 6 7 8\\
9 10 11 12 13 14\\
15 16 17 18 19 20 \\
21 22 23 24 25 26 \\
27 28 29 30 31 32}&
 \mlcell{0\\
32 31 30 29 28\\
27 26 25 24 23 \\
22 21 20 19 18 17\\
16 15 14 13 12 11\\
10 9 8 7 6 5 4 3 2 1}&\\
\hline
  Izlaz &
  \mlcell{4294967288} &  
  \mlcell{4294967295} &  
  \\ 
  \hline 
\end{tabular}
\normalsize

\item U\v citati listu celih brojeva iz datoteke \textit{lista.txt}, a zatim izme\dj u svaka dva elementa liste ubaciti njihovu apsolutnu razliku. Ne praviti novu listu. Nije dozovoljeno samo ispisati na izlaz tra\v zeni poredak, ve\' c je potrebno zaista modifikovati listu i onda nju ispisati.

\small
\begin{tabular}{ |l|l|l|l|l| }
\hline 
  Ulaz & 
  \mlcell{ 1 2 3 4 5  }&
  \mlcell{ 1 1 1 1 1 } & 
  \mlcell{-1 -2 -3 -4 -5} & 
  \mlcell{ 0 2 2 4 6 10  }\\ 
  \hline 
  Izlaz &
  \mlcell{1 1 2 1 3 1 4 1 5} &  
  \mlcell{1 0 1 0 1 0 1 0 1} &  
  \mlcell{-1 1 -2 1 -3 1 -4 1 -5}&  
  0 2 2 0 2 2 4 2 6 4 10 \\ 
  \hline 
\end{tabular}
\normalsize


\item Napisati program koji u\v citava binarno stablo pretrage iz datoteke \textit{stablo.txt} i za zadato $n$ ispisati razliku vrednosti maksimalnog i minimalnog \v cvora u stablu na nivou $n$. U slu\v caju gre\v ske (ako je $n < 1$ ili u stablu ne postoji nivo $n$) na standardni izlaz ispisati -1. 

\small
\begin{tabular}{ |l|l|l|l|l| }
\hline 
  stablo.txt & 
  \mlcell{ 3 1 5 2 0 4 8 2 6 }&
 \mlcell{ 1 2 3 4 5 6 7 8 9 10}&
 \mlcell{ 1 2 3 4 5 6 7 8 9 10  }&
 3 1 5 2 0 4 8 2 6 \\
  \hline 
  Ulaz &
  \mlcell{3} &  
  \mlcell{4} &  
  \mlcell{10}&  
  6\\ 
  \hline
   Izlaz &
  \mlcell{8} &  
  \mlcell{0} &  
  \mlcell{0}&  
  -1\\ 
  \hline
\end{tabular}
\normalsize
\end{enumerate}


\section{Programiranje 2 2015/2016, ispit, septembar}

\begin{enumerate}
\item
U datoteci brojevi.txt se nalaze celi brojevi. Napisati program
koji pronalazi i ispisuje na standardni izlaz dva broja koja se najmanje razlikuju po vrednosti (ako ima vi\v se parova koji se isto razlikuju, ispisati bilo koji par). U slu\v caju gre\v ske ispisati -1 na standardni izlaz.

\small
\begin{tabular}{ |l|l|l|l|l| }
\hline
   & Primer 1* &  Primer 2 &  Primer 3 &  Primer 4 \\ \hline
  brojevi.txt  & 1 2 10 5 & -1 0 0 1 & 10 -1 -2 6 3 & ne postoji \\ \hline
  Izlaz & 1 2 & 0 0 & -1 -2 & -1 \\ \hline
\end{tabular}
\normalsize

\item U datoteci kupci.txt se nalaze podaci o kupcima oblika ime koli\v cina gde je ime re\v c maksimalne du\v zine 20 karaktera, a koli\v cina pozitivan ceo broj. Sa standardnog ulaza se u\v citava ceo broj \emph{n}. Na standardni izlaz ispisati imena prvih \emph{n} kupaca koji su uzeli najvi\v se proizvoda. Pretpostaviti da su imena kupaca jedinstvena, a ukoliko ima vi\v se kupaca sa istom koli\v cinom kupljenih proizvoda, ispis sortirati leksikografski rastu\' ce po imenu. Ako ima manje od \emph{n} kupaca u datoteci, ispisati sve kupce.    

Nije poznat broj kupaca u datoteci. Pretpostaviti da su podaci u datoteci u ispravnom obliku. U slucaju gre\v ske ispisati -1 na standardni izlaz. \\
\textbf{Napomena}: neophodno je uraditi zadatak kori\v s\' cenjem dinami\v cke alokacije memorije (u suprotnom \' cete dobiti 0 poena).

\small
\begin{tabular}{ |l|l|l|l|l| }
\hline
  & Primer 1* &  Primer 2 &  Primer 3 &  Primer 4 \\ \hline
  ulaz & 3 & 3 & -2 & 3 \\ \hline
  kupci.txt & Pera 100 & Mika 120 & \v Zika 50 & \emph{ne postoji}\\
          & Ana 30 & Ana 120 & & \\
          & Stefan 2 & & &  \\
          & Marija 50 & & &  \\
          & Lena 45 & & &  \\
          & Pavle 105 & & &  \\
          & Marko 101 & & &  \\ \hline
  izlaz &Pavle  & Ana &-1& -1 \\
         & Marko &Mika & &  \\
         & Pera & & &  \\ \hline
\end{tabular}
\normalsize


\item Sa standardnog ulaza se u\v citava neozna\v cen ceo broj \emph{x}. Ukoliko je suma bitova u\v citanog broja parna, \v siftovati bitove broja za jedno mesto ulevo, a ukoliko je suma bitova neparna, \v siftovati bitove broja za jedno mesto udesno. Na standardni izlaz ispisati rezultuju\' ci neozna\v ceni ceo broj. 

\small
\begin{tabular}{ |l|l|l|l|l| }
\hline
  & Primer 1* &  Primer 2 &  Primer 3 &  Primer 4 \\ \hline
  ulaz & 3 & 7 &  8765
 &0\\ \hline
  izlaz & 6   & 3 &  4382 & 0\\\hline
\end{tabular}
\normalsize

\item Iz datoteke \textit{lista.txt} se u\v citava lista celih brojeva.
Kreirati funkciju \emph{void f4(\_cvor *g)} koja izbacuje one elemente liste kojima su u prvobitnoj listi oba suseda parni brojevi. Ispisati rezultuju\' cu listu na standardni izlaz.
U slucaju gre\v ske ispisati -1 na standardni izlaz.

\small
\begin{tabular}{ |l|l|l|l|l| }
\hline
  & Primer 1* &  Primer 2 &  Primer 3 &  Primer 4 \\ \hline
  lista.txt & 1 2 3 4 & 2 2 2 2 & 7 3 6 6 7 3 2 7 6 8 & 3  \\ \hline

  izlaz & 1 2 4 & 2 2 & 7 3 6 6 7 3 2 6 8 & 3 \\ \hline
\end{tabular}
\normalsize

\item Iz datoteke \textit{stablo.txt} se u\v citava stablo celih brojeva.  Ispisati one elemente za koje va\v zi da je zbir levog i desnog potomka razli\v cite parnosti od roditelja. Uzimati u obzir samo \v cvorove koji imaju oba potomka. U slu\v caju gre\v ske ispisati -1 na standardni izlaz.

\small
\begin{tabular}{ |l|l|l|l|l| }
\hline
& Primer 1* &  Primer 2 &  Primer 3 &  Primer 4 \\ \hline
  stablo.txt & 10 2 13 & 2 1 3 & 1 2 3 & 2 5 1 4 6\\ \hline

  izlaz & 10 &  &  &5\\ \hline
\end{tabular}
\normalsize

\end{enumerate}

