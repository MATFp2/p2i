\chapter{Testovi i ispiti 2011/2012}


---------------------------------------------------------------------
\section{Programiranje 2, I smer, 2011/12, Popravi test}
%---------------------------------------------------------------------

\subsection{Deo II - Grupa 1}

Sa standardnog ulaza unosi se lista celih brojeva dok se ne unese 0, a potom se unosi jedan broj.
Odrediti poziciju prvog pojavljivanja broja u listi i ispisati na standardni izlaz. Ukoliko broja nema u listi ispisati -1.
Pozicija u listi se broji po\v cev\v si od 1.\\

    \noindent U datoteci \verb|liste.h| nalaze se funkcije za rad sa listom:
    \begin{verbatim}
    void oslobodi(cvor* lista)
    cvor* ubaci_na_pocetak(cvor* lista, int br)
    cvor* novi_cvor(int br)
    void ispis(cvor* lista)
    \end{verbatim}

    \noindent Napraviti main.c u kome se testira rad programa. U main.c treba da stoji \verb|#include "liste.h"|. \\
    Kompaliranje:   \verb|gcc main.c liste.c|\\
    Pokretanje: \verb|./a.out|\\

  \textbf{NAPOMENA: Ukoliko se zadatak uradi bez kori\v s\' cenja liste broj osvojenih poena je 0.}

\emph{Primeri:}
\begin{verbatim}
  Primer 1:                        Primer 2:
  ulaz: -20 7 -31 4 5 6 0 -31      ulaz: 4 90 234 -8 0 322

  izlaz: 3                         izlaz: -1

\end{verbatim}

\begin{verbatim}
  Primer 3:                        Primer 4:
  ulaz: 9 7 -42 7 343 7 0 7        ulaz: 8 90 8 56 3 2 0 8

  izlaz: 2                         izlaz: 1
\end{verbatim}

\bigskip

\subsection{Deo II - Grupa 1}

Sa standardnog ulaza unosi se lista celih brojeva dok se ne unese 0.
  Odrediti broj pojavljivanja datog broja u listi.\\

  \noindent U datoteci \verb|liste.h| nalaze se funkcije za rad sa listom:
    \begin{verbatim}
    void oslobodi(cvor* lista)
    cvor* ubaci_na_pocetak(cvor* lista, int br)
    cvor* novi_cvor(int br)
    void ispis(cvor* lista)
    \end{verbatim}

    \noindent Napraviti main.c u kome se testira rad programa. U main.c treba da stoji \verb|#include "liste.h"|. \\
    Kompaliranje:   \verb|gcc main.c liste.c|\\
    Pokretanje: \verb|./a.out|

  \textbf{NAPOMENA: Ukoliko se zadatak uradi bez kori\v s\' cenja liste broj osvojenih poena je 0.}

\emph{Primeri:}

\begin{verbatim}
  Primer 1:                      Primer 2:
  ulaz: 56 -34 234 0 -34            ulaz: 3456 -67 -23 -67 45 0 -67

  izlaz: 1                          izlaz: 2

\end{verbatim}

\begin{verbatim}
  Primer 3:                      Primer 4:
  ulaz: 89 0 567 -34 0 2         ulaz: 907 8 8 2 3 8 45 0 8

  izlaz: 0                       izlaz: 4
\end{verbatim}


%---------------------------------------------------------------------
\section{Programiranje 2, I smer, 2011/12, Zavr\v{s}ni ispit --- jun}
%---------------------------------------------------------------------

\subsection{Deo II}

\begin{enumerate}
\item
Kao argumenti komandne linije zadaju su brojevi, nepoznato koliko.
  Napisati funkciju \\ \verb|int f1(char** brojevi, unsigned int broj_brojeva)| koja ra\v cuna njihov zbir.

Primeri:
\begin{verbatim}
Primer 1:           Primer 2:          Primer 3:              Primer 4:

./a.out 3 4         ./a.out            ./a.out 3 -56 231      ./a.out 0 561 -34 2 34 56

7                   0                   178                   619
\end{verbatim}

\item
  U datoteci \verb|tekst.txt| nalazi se tekst. Ako datoteka ne postoji ispisati $-1$ na standardni izlaz.
  Napisati funkciju \verb|int f2()| koja otvara datoteku, prebrojava i vra\'ca broj cifara koje se pojavljuju
  u datoteci.


Primeri (dati tekst se nalazi u datoteci \verb|tekst.txt|):
\begin{verbatim}
Primer 1:                              Primer 2:
danas je 5 puta lepsi dana             od 2011. godine moze
nego juce                              se putovati eko-taksijem kroz Bec

rez: 1                                 rez: 4


\end{verbatim}

\begin{verbatim}
Primer 3:                              Primer 4:
ovde nema cifara                       a ov23de se cif89re
                                       nalaze u 8 recima

rez: 0                                 rez: 5
\end{verbatim}

\item
  Napisati funkciju \verb|int f3()| koja u\v citava
  cele brojeve sa standarnog ulaza sve dok se ne u\v cita 0. Broj brojeva nije unapred poznat. Funkcija vra\'ca
  broj brojeva ve\'cih od proseka. Ako nema unetih brojeva funkcija treba da vrati $0$.

Primeri:
\begin{verbatim}
Primer 1:            Primer 2:         Primer 3:            Primer 4:
2 3 4 5 0            0                 56 -45 231 -34 0     76 -45 2 0

rez: 2               rez: 0            rez: 2               rez: 1
\end{verbatim}

\item
Data je
datoteka \verb|voce.txt| koja ima najvi\v se 100 redova. U svakom redu se nalazi ime vo\'cke (ne du\v ze od 20 karaktera) i cena (int).
Napisati funkciju \verb|int f4(int p, char* ime, int** cena)| koja otvara datoteku i u\v citava podatke iz nje u niz struktura i
sortira taj niz po cenama. U promenljive \verb|ime| i \verb|cena| upisuju se ime vo\' cke i cena vo\' cke koja se nalazi na poziciji \verb|p| u sortiranom nizu. Ukoliko  \verb|p| ima nekorektnu vrednost funkcija vra\'ca -1
a ina\v ce vra\'ca 1. \verb|p| je nekorektno ako je manje od 0 ili ve\'ce ili jednako od broja elemenata niza.


Primeri(podaci o vo\'ckama su u datoteci \verb|voce.txt|):
\begin{verbatim}
Primer 1:                Primer 2:                Primer 3:               Primer 4:

jabuka 10                ananas 12                jagode 14               limun 45
kruska 15                kivi 34                  breskve 23              grejpfrut 23
sljiva 23                                         pomorandze 11
malina 7                 p = 5                    banane 3                p = -1

p = 2                    funkcija vraca -1;       p = 1                   funkcija vraca -1;

ime = kruska;                                     ime = pomorandze
cena = 15;                                        cena = 11
funkcija vraca 1;                                 funkcija vraca 1;
\end{verbatim}

\item
Napisati funkciju \verb|unsigned int f6(unsigned int x)| koja
vra\' ca broj dobijen izdvajanjem prvih 8 bitova broja (bitovi na najve\'cim tezinama), a ostatak
broja popunjava jedinicama. Testirati pozivom u main-u.


Primeri:
\begin{verbatim}
Primer 1:            Primer 2:             Primer 3:           Primer 4:
x = 88888899         x = 679012345         x = 34              x = 837312000

rez: 100663295       rez: 687865855        rez: 16777215       rez: 838860799

\end{verbatim}

\item
Napisati funkciju \verb|int f7(cvor* lista)| koja menja elemente liste na slede\' ci na\v cin:
ako je teku\' ci element paran,
slede\' ci element uve\' cava za 1, a ako je element neparan slede\' ci element smanjuje za 1.
Parnost broja se odnosi na teku\' cu, promenjenu vrednost broja. Funkcija vra\'ca
vrednost poslednjeg elementa liste.

Primeri:
\begin{verbatim}
Primer 1:               Primer 2:
1->2->3->4->5->NULL     6->78->2->3->1->NULL

rez: 4                  rez: 2


\end{verbatim}

\begin{verbatim}
Primer 3:                     Primer 4:
-87->9->45->2->2->NULL        22->-34->0->2->1->NULL
rez: 1                        rez: 0
\end{verbatim}


\item
Napisati funkcju \verb|int f8(cvor* drvo, int nivo)|
koja vra\'ca  broj elemenata drveta koji se nalaze na nivou \verb|nivo| (\verb|nivo>0|).
Testirati pozivom u main-u.

Primeri:
\begin{verbatim}
Primer 1:                Primer 2:
sa standardnog ulaza     sa standardnog ulaza
se unose redom:          se unose redom:
6 4 12 3 10 32           7 2 10 23

      6                     7
     / \                   / \
    4   12                2   10
   /   /  \                     \
  3   10  32                     23

nivo: 3                  nivo: 1

rez: 3                   rez: 1


\end{verbatim}

\begin{verbatim}
Primer 3:                Primer 4:
sa standardnog ulaza     sa standardnog ulaza
se unose redom:          se unose redom:
10 12                    7 4 20 3 15 23 2

      10                     7
        \                   / \
         12                 4   20
                          /   /  \
nivo: 3                  3   15   23
                        /
rez: 0                 2

                         nivo: 2

                         rez: 2
\end{verbatim}
\end{enumerate}


%---------------------------------------------------------------------
\section{Programiranje 2, I smer, 2011/12, Zavr\v{s}ni ispit --- juli}
%---------------------------------------------------------------------

\subsection{Deo II}


\begin{enumerate}
\item Argumenti komadne linije su opcija (\verb|-m|, \verb|-v| ili \verb|-mv|) i re\v c.
Ukoliko je opcija \verb|-m| pretvoriti sva slova re\v ci u mala slova, ukoliko je opcija
\verb|-v| pretvoriti sva slova u re\v ci u velika slova, a ukoliko je opcija \verb|-mv|
pretvoriti sva mala slova u velika, a sva velika slova u mala slova. Ukoliko opcija nije zadata
ili je neta\v cno navedena ispisati -1.

\begin{small}
\begin{verbatim}
ulaz: ./a.out -m Dan28Mesec2        ulaz: ./a.out -v DanDanas;Pamtim

izlaz: dan28mesec2                  izlaz: DANDANAS;PAMTIM
----------------------------------------------------------------------------------

ulaz: ./a.out -mv VArljivoLeto68    ulaz: ./a.out -nn greska

izlaz: vaRLJIVOlETO68               izlaz: -1

\end{verbatim}
\end{small}


\item Napisati funkciju \verb|void f2(char* ime, int n)| koja preko argumenata komandne linije dobija
ime datoteke i ceo broj n. Iz datoteke ispisati na standardni izlaz svaku n-tu re\v c. Ukoliko
argumenti nisu pravilno zadati ispisati -1. Gre\v ske mogu biti da nije zadato ime datoteke
ili ceo broj. Pretpostavlja se da ako je une\v sen drugi argument je sigurno ceo broj.
Ukoliko datoteka ne postoji ispisati -1.


\begin{small}
\begin{verbatim}
ulaz: ./a.out primer1.txt 3              ulaz: ./a.out 4

izlaz: Volfganga Nemca koji godine,      izlaz: -1
       da pravi zarobljenik veliki
----------------------------------------------------------------------------------------------
ulaz: ./a.out                            ulaz: ./a.out primer2.txt 10

izlaz: -1                                izlaz: Ban Bil mir
\end{verbatim}
\end{small}


\item Sa standarnog ulaza unosi se ceo pozitivan broj i pozitivan broj \verb|p|. Napisati funkciju \\
\verb|unsigned int f3(unsigned int x,unsigned int p)| koja menja
mesta prvih i poslednjih \verb|p| bitova (npr. p = 3 i broj x = 111....010 treba da se dobije izlaz
x = 010...111)


\begin{small}
\begin{verbatim}
x = 1234567890         x = 567890567           x = 7484                 x = 827262627
p = 5                  p = 7                   p = 4                    p = 10

izlaz: 2442527433      izlaz: 265900688        izlaz: 3221232944        izlaz: 2832139461
\end{verbatim}
\end{small}


\item U datoteci \verb|podaci.txt| se nalaze re\v ci i pozitivni celi brojevi. Napisati funkciju
\verb|void f4(int m, int n)| koja sortira leksikografski re\v ci u rastu\'cem
poretku i  ispisuje re\v c na poziciji \verb|m|. Potom sortirati brojeve u opadaju\'cem poretku
i ispisati broj na poziciji \verb|n|. Ukoliko datoteka ne postoji ispisati -1. Ukoliko \verb|m|
i \verb|n| nisu u odgovaraju\'cem opsegu ispisati -1. Re\v ci su maksimalne du\v zine 20 karaktera.


\begin{small}
\begin{verbatim}
podaci.txt: Tekst 32 je sa 67 brojevima 166          podaci.txt: Drugi 789
            i 890 zato 2 je cudan.                               interesantni 56 3 4
m = 2                                                            primer
n = 2                                                m = 4
                                                     n = 2

izlaz: cudan                                         izlaz: -1
       67
------------------------------------------------------------------------------------------------

podaci.txt: 38 17 Londonski 67danas umetnik Patrik Vejl
            postao je 92 internet senzacija 14 42.
m = 4
n = 1

izlaz: internet
       42
\end{verbatim}
\end{small}


\item Napisati funkciju \verb|void f5(cvor* l1, cvor* l2)| koja spaja dve rastu\'ce sortirane liste tako
da rezultat i dalje bude sortiran. Rezultat sa\v cuvati u prvoj listi. Rezulatuju\' cu
listu ispisati na standardni izlaz. Lista se u\v citava sve dok se ne unese 0.

\begin{small}
\begin{verbatim}
(napomena: lista se unosi obrnutim redosledom i na kraju se unosi 0)

ulaz: 8->9->15->62->NULL                            ulaz: 8->9->15->62->NULL
      15->67->100->102->NULL                              2->4->16->NULL

izlaz: 8->9->15->15->62->67->100->102->NULL         izlaz: 2->4->8->9->15->16->NULL
\end{verbatim}
\end{small}


\item Napisati funkciju koja iz datoteke \verb|karakteri.txt|
ucitava karaktere i sme\v sta ih u binarno pretra\v zivacko drvo.
Uz svaki karakter \v cuvati i broj pojavljivanja karaktera u datoteci.
Ispisati na standardni izlaz karakter koji se pojavljuje najve\'ci broj puta u datoteci.


\begin{small}
\begin{verbatim}
karakteri.txt: Danas je lep dan.          karakteri.txt: 78665555512

izlaz: a                                  izlaz: 5
------------------------------------------------------------------------------

karakteri.txt: U Datoteci
               je probni test
               za Zadatke sa ispita.

izlaz: a (izlaz moze biti i: t, ' ' (blanko znak))
\end{verbatim}
\end{small}
\end{enumerate}


%---------------------------------------------------------------------
\section*{Programiranje 2, I smer, 2010/11, zavr\v{s}ni ispit, oktobar 2012}
%---------------------------------------------------------------------



\begin{enumerate}
\item Napisati funkciju \verb|int f1(int x, int p, int q)| koja kao argumente dobija 3 cela broja \verb|x|,
\verb|p|, \verb|q|, a kao rezultat vra\'ca broj
koji je jednak broju \verb|x| kome je invertovan svaki drugi bit izme\d u pozicije \verb|p| i \verb|q|,
\v citano sa desna na levo. Ukoliko \verb|p| ili \verb|q| izlaze iz opsega ili \verb|p| $>$ \verb|q|
vratiti \verb|-1| kao rezultat.

\begin{verbatim}
Primer 1:             Primer 2:             Primer 3:                 Primer 4:
ulaz: 3456 2 10       ulaz: -895 13 20      ulaz: 678910 21 31        ulaz: 657 19 33

izlaz: 2260           izlaz: -697215        izlaz: -1431675906        izlaz: -1
\end{verbatim}

\item Napisati funkciju \verb|cvor* f2(cvor* L)| koja dobija glavu liste L
kao argument, obr\'ce listu L i vra\'ca novu glavu.

\begin{verbatim}
ulaz: 4->1->2->67->0       ulaz: 0           ulaz: -56->28->31->-1000->0    ulaz: 10->9->8->1000000->0
izlaz: 67 2 1 4            izlaz:            izlaz: -1000 31 28 -56         izlaz: 1000000 8 9 10
\end{verbatim}


\item Stablo se u\v citava tako da su u listovima realni brojevi,
a u unutrasnjim cvorovima neka od operacija \verb|+|, \verb|-|, \verb|*| i \verb|/|.
Funkcija \verb|float f3(svor* drvo)| ra\v cuna izraz u stablu i vra\'ca rezultat. Ukoliko
do\d e do deljenja nulom funkcija vra\'ca -1.


\begin{verbatim}
Primer 1:           Primer 2:             Primer 3:                 Primer 4:

ulaz: + 3 4         ulaz: + * 5 4 2       ulaz: + 3 * * 5 4 3       ulaz: / + 5 4 3
izlaz: 7            izlaz: 22             izlaz: 63                 izlaz: 3
\end{verbatim}


\item Argumenti komandne linije su cu celi brojevi \verb|a|, \verb|n|, \verb|m|, \verb|p|. Napisati program koji ra\v cuna niz
\begin{verbatim}
a mod n,
2a mod n,
3a mod n
...
m*a mod n
\end{verbatim}
sortira ga u rastu\' cem poretku i ispisuje u datoteku \verb|rez.txt| p-ti \v clan niza (brojanje indeksa pocinje od 0)

\begin{verbatim}
Primer 1:                 Primer 2:                   Primer 3:
ulaz: ./a.out 5 6 3 1     ulaz: ./a.out 23 10 5 2     ulaz: ./a.out 10 10 100 45
rez.txt: 4                rez.txt: 5                  rez.txt: 0
----------------------------------------------------------------------------------------------------------------
\end{verbatim}

\begin{verbatim}
Primer 4:
ulaz: ./a.out 634 2 34 67
rez.txt: -1
\end{verbatim}

Ukoliko je do\v slo do neke od gre\v saka (nije ta\v can broj argumenata komandne linije, p$>$n,...)
u \verb|rez.txt| upisati \verb|-1|.

\item Argumenat komandne linije su ime datoteke i ceo broj \verb|n|.
Napisati funkciju \verb|void f5(char* ime_dat, int n)|
koja ispisuje na standardni izlaz svaku \verb|n|-tu re\v c iz datoteke.
Ukoliko datoteka ne postoiji ili je broj \verb|n| neispravno zadat ili
nema dovoljno argumenenata komadne linije ispisati \verb|-1|. Pretpostaviti
da je maksimalna du\v zina re\v ci 100.


\begin{verbatim}
Primer 1:                                          Primer 2:

./a.out dat.c 3                                    ./a.out p.txt 2
dat.c: Kada 7642oIO solja za kafu                  U sobi ima 2
       ekspres lonac, hleb i mleko.                kreveta, sto i racunar
       Spisak za kupovinu.                         tepih i jos neka sitna oprema.

izlaz: solja ekspres i za                          izlaz: sobi 2 sto racunar i neka oprema.
-----------------------------------------------------------------------------------------
\end{verbatim}

\begin{verbatim}
Primer 3:                                          Primer 4:

./a.out dat.c -67                                  ./a.out primer.txt

izlaz: -1                                          izlaz: -1
\end{verbatim}

\item U datoteci \verb|kupci.txt| se nalaze podaci o kupcima oblika \verb|ime kolicina| gde je \verb|ime| ime kupca
koji je kupio proizvod, a \verb|kolicina| koli\v cina proizvoda. Kupac se mo\v ze vi\v se puta pojaviti
u datoteci. Napraviti binarno pretra\v ziva\v cko drvo prema imenima kupaca. Na standarni izlaz ispisati
ime onog kupca koji je uzeo najvi\v se proizvoda. Pretpostavlja se da je datoteka dobro struktuirana.


\begin{verbatim}
kupci.txt:          kupci.txt:        kupci.txt:          kupci.txt:
pera 30             mika 10           zika 50
marko 10            mira 15           ana 70
jelena 14           mika 6
marko 0

izlaz: pera         izlaz: mika       izlaz: ana          izlaz:
\end{verbatim}
\end{enumerate}


%---------------------------------------------------------------------
\section{Programiranje 2, I smer, 2011/12, Zavr\v{s}ni ispit --- januar}
%---------------------------------------------------------------------

\subsection{Deo II}

\begin{enumerate}
\item Napisati funkciju \verb|float f5(float a, float b, float eps)| koja
ra\v cuna nulu funkcije $f(x) = 5*sin(x)*ln(x)$ na intervalu \verb|(a, b)| sa ta\v cno\v s\'cu \verb|eps|. Brojevi
\verb|a|, \verb|b| i \verb|eps| unose se sa standardnog ulaza.\\
\textbf{Napomena1}: koristiti algoritam binarne pretrage\\
\textbf{Napomena2}: u \verb|math.h| nalaze se \verb|float sin(float)| za ra\v cunanje sinusa i
 \verb|float log(float x)| za ra\v cunanje prirodnog logaritma.
Testirati funkciju pozivom u main-u.

\begin{verbatim}
Primer 1:        Primer 2:             Primer 3:        Primer 4:
0.9 2 0.01       2 4 0.000001          4 10 0.001       10 20 0.001

0.998828         3.141593              6.283142         12.566376
\end{verbatim}


\item Napisati funkciju koja pronalazi zbir parnih brojeva na neparnim pozicijama u nizu koji se zadaje kao argument komandne linije. Svi brojevi su pozitivni (ovo ne treba proveravati).
Brojanje pozicija pocinje indeksom 0. Rezultat se ispisuje na standardni izlaz.

\begin{verbatim}
Primer 1:                 Primer 2:    Primer 3:                                Primer 4:
\a.out 23 4 7 89 12       \a.out       \a.out 54 1 23 76 18 90 322 45678 12     \a.out 18 17 16

4                         0             45844                                   0
\end{verbatim}


\item Napisati program koji iz datoteke \verb|brojevi.txt| u\v citava cele brojeve (nije unapred poznato koliko
se brojeva nalazi u datoteci) pa zatim ceo broj sa standardnog ulaza. Program treba da vrati indeks
tog broja u nizu (ako se broj nalazi u nizu) ili indeks i vrednost elementa niza koji je po apsolutnoj
vrednosti najbli\v zi tom broju. Ukoliko ne bude kori\v s\'cena dinami\v cka alokacija memorije zadatak ne\'ce
biti priznat. Indeksiranje niza po\v cinje od 0.

\begin{verbatim}
Primer 1:                               Primer 2:                        Primer 3:
brojevi.txt: 23 -4 56 13 4 56 7         brojevi.txt: -7 8 14 12          brojevi.txt: -7 8 14 12
56                                      -8                               -13

2                                        1                               2

---------------------------------------------------------------------------------------------------------------
Primer 4:
brojevi.txt: -7 8 14 12
56

2
\end{verbatim}


\item Napisati funkciju \verb|void f4(cvor* lista)| koja iz liste izbacuje svaki drugi element.
U listi se nalaze celi brojevi. Lista se unosi sa standardnog ulaza sve dok se ne unese 0.

\begin{verbatim}

Primer 1:             Primer 2:         Primer 3:           Primer 4:
1 2 3 4 5 6 0         1 2 3 4 5 0       34 5 -78 23 0       -16 72 13 24 98 0

1 3 5                 1 3 5             34 -78              -16 13 98
\end{verbatim}



\item Napisati program koji u zavisnosti od toga da li je suma bitova broja parna ili neparna,
\v siftuje bitove broja za jedno mesto ulevo odnosno u desno.

\begin{verbatim}
Primer 1:           Primer 2:        Primer 3:       Primer 4:
3                   7                4567            8765

6                   3                9134            4382
\end{verbatim}


\item Funkcija vr\v si rekurzivnu rotaciju drveta oko svih \v cvorova,
dakle dobija se odraz prvobitnog drveta u ogledalu.
\begin{verbatim}
Primer 1:                                       Primer 2:

    7                                               2
   / \                                             / \
  2   10                                        -10   8
 / \ / \                                               \
4  5 9  17                                              7
       /
      12
       \
        13

unos: 7 2 4 5 10 9 17 12 13 0                   unos: 2 -10 8 7 0
\end{verbatim}
se transformi\v se u:
\begin{verbatim}
     7                                              2
    / \                                            / \
  10   2                                          8   -10
 / \  / \                                        /
17  9 5  4                                      7
 \
  12
 /
13

ispis: 7 10 17 12 13 9 2 5 4                   ispis: 2 8 7 -10
-----------------------------------------------------------------------------------
\end{verbatim}


\begin{verbatim}
Primer 3:                                   Primer 4:
   5                                          10
  / \                                        /  \
 2   10                                     7    23
                                           /     / \
unos: 5 2 10 0                            2     12  40

                                            unos: 10 7 2 23 12 40 0
\end{verbatim}
se transformi\v se u:
\begin{verbatim}
  5                                                10
 / \                                              /  \
10  2                                            23   7
                                                / \    \
ispis: 5 10 2                                  40 12   2

                                                 ispis: 10 23 40 12 7 2

\end{verbatim}
\end{enumerate}


