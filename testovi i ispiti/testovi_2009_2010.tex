% *************************************************************
% *************************************************************
% *************************************************************
\chapter{Testovi i ispiti 2009/10}
% *************************************************************
% *************************************************************
% *************************************************************



\section{Programiranje 2, I smer, 2009/10, Kolokvijum 1}


\begin{enumerate}
\item Napisati funkciju koja kao argumente prima kvadratnu matricu
  celih brojeva i njenu dimenziju, a vra\'ca 1 ako je matrica donja
  trougaona, odnosno 0 ako nije. Pretpostavka je da je maksimalna dimenzija
  matrice 100. Matrica je donja trougaona ako se u
  gornjem trouglu (iznad glavne dijagonale, ne uklju\v cuju\' ci je)
  nalaze sve nule. Napisati prate\'ci program koji omogu\'cava
  u\v{c}itavanje matrice iz datoteke \v{c}ija je putanja zadata prvim
  argumentom komandne linije, a format datoteke je takav da je prvi
  pro\v{c}itani broj dimenzija matrice, a potom sledi niz brojeva koji
  popunjavaju matricu sleva nadesno, odozgo na dole (vrstu po vrstu).

\item Napisati program koji omogu\'cava u\v{c}itavanje artikala iz
  datoteke koja se zadaje prvim argumentom komandne linije. Jedan
  artikal je definisan strukturom

\begin{verbatim}
typedef struct {
  char ime[50];
  float tezina;
  float cena;
} Artikal;
\end{verbatim}

Datoteka je ispravno formatirana i sadr\v{z}i najvi\v{s}e 500
artikala. Program treba da, nakon u\v{c}itavanja, ispi\v{s}e sve
artikle sortirane po zadatom kriterijumu. Kriterijum se zadaje kao
drugi argument komandne linije i mo\v{z}e biti: \verb|i| - sortirati
po imenu; \verb|t| - sortirati po te\v{z}ini; \verb|c| - sortirati po
ceni.

Podrazumeva se da je sortiranje u rastu\' cem redosledu, a da se
za sortiranje po imenu koristi leksikografski poredak (dozvoljeno
je kori\v s\' cenje funkcije \verb|strcmp|).  Koristiti generi\v
cku funkciju za sortiranje \verb|qsort| iz standardne biblioteke.

\item Napisati funkciju sa prototipom

\begin{verbatim}
double treci_koren(double x, double a, double b, double eps);
\end{verbatim}

koja, metodom polovljenja intervala, ra\v{c}una tre\'ci koren
zadatog broja $x$ ($x \geq 1$), tj.~za dato $x$ odre\d uje broj
$k$ za koji va\v{z}i $k^3=x$, sa ta\v cno\v s\' cu $eps$ i sa
po\v{c}etnom pretpostavkom da se broj $k$ nalazi u datom intervalu
$[a,b]$. Napisati prate\'ci program koji omogu\'cava korisniku da
unese broj $x$ i tra\v{z}enu ta\v{c}nost. Korisnik ne unosi
po\v{c}etnu procenu intervala, a za po\v{c}etni interval mo\v{z}e
se uzeti $[0,x]$. Ispisati poruku o gre\v{s}ci ako je uneto $x$
manje od $0$.
\end{enumerate}

%------------------------------------------------------------------------------------------------


\section{Programiranje 2, I smer, 2009/10, Zavr\v sni ispit (juni 2010)}


\begin{enumerate}
\item Napisati program koji sa standardnog ulaza u\v{c}itava pozitivne cele
brojeve dok ne u\v{c}ita nulu kao oznaku za kraj. Na standardni izlaz ispisati
koji broj se pojavio najvi\v{s}e puta me{\d}u tim brojevima. Na primer, ako se
na ulazu pojave brojevi: 2 5 12 4 5 2 5, program treba da ispi\v{s}e broj 5.
Zadatak resiti kori\v{s}\'cenjem dinami\v{c}ke realokacije i strukture koja
sadr\v{z}i ceo broj i broj njegovih pojavljivanja. Slo\v{z}enost izvr\v{s}avanja
napisanog programa nije bitna.

\item Napisati funkciju \verb|cvor* nova_lista(cvor* L1, cvor* L2);|
koja od dve date liste \verb|L1| i \verb|L2|, u kojima se \v{c}uvaju celi brojevi, formira
novu listu koja koja sadr\v{z}i alterniraju\'ce raspore{\d}ene elemente iz lista \verb|L1| i
\verb|L2| (prvi element iz \verb|L1|, prvi element iz \verb|L2|, drugi element \verb|L1|,
drugi element \verb|L2| itd.) sve dok ima elemenata u {\em obe} liste. Ne formirati nove
\v{c}vorove, ve\'c samo postoje\'ce \v{c}vorove povezati u jednu listu, a kao rezultat
vratiti po\v{c}etak te formirane liste.

\item Neka su \v{c}vorovi binarnog stabla koje opisuje aritmeti\v{c}ki izraz opisani
slede\'com strukturom:

\begin{verbatim}
typedef struct tagCvor
{
    int tipCvora;
    int tipOperatora;
    int vrednostKonstante;
    char oznakaPromenljive;

    struct tagCvor *levo;
    struct tagCvor *desno;
} CVOR;
\end{verbatim}

\v{C}lan \verb|tipCvora| moze imati slede\'ce vrednosti:
\begin{itemize}
\item[0] - \v{c}vor je operator i mora imati oba deteta;
\item[1] - \v{c}vor je konstanta i ne sme imati decu, vrednost je u \v clanu \verb|vrednostKonstante|;
\item[2] - \v{c}vor je promenljiva sa oznakom koju defini\v{s}e \v{c}lan \verb|oznakaPromenljive|
i ne sme imati decu.
\end{itemize}

\v{C}lan \verb|tipOperatora| mo\v{z}e imati slede\'ce vrednosti:
\begin{itemize}
\item[0] - sabiranje
\item[1] - mno\v{z}enje
\end{itemize}

\begin{itemize}
\item[(a)] Napisati funkciju \verb|void dodela(CVOR* koren, char promenljiva, int vrednost);|
koja menja izraz dat korenom u izraz u kojem je svako pojavljivanje promenljive sa datom
oznakom zamenjeno konstantom sa datom vredno\v{s}\'cu.

\item[(b)] Napisati funkciju \verb|int imaPromenljivih(CVOR *koren);|
koja ispituje da li izraz dat korenom sadr\v{z}i promenljive. Funkcija vra\'ca
vrednost razli\v{c}itu od nule ako izraz ima promenljivih, odnosno nulu
u suprotnom.

\item[(c)] Napisati funkciju \verb|int vrednost(CVOR *koren, int* v);|
koja na adresu na koju pokazuje drugi argument funkcije sme\v sta  vrednost izraza datog korenom \verb|koren| i vra\'ca vrednost \verb|0|,
u slu\v{c}aju da izraz nema promenljivih. U slu\v{c}aju da izraz ima
promenljivih funkcija treba da vra\'ca \verb|-1|.
\end{itemize}

Podrazumevati da je drvo izraza koje se prosle{\d}uje funkcijama ispravno konstruisano.
\end{enumerate}


\section{Programiranje 2, I smer, 2009/10, Zavr\v sni ispit (septembar 2010)}



\begin{enumerate}
\item Napisati funkciju \verb|void zamene(CVOR** p)| koja menja povezanu listu koja
je zadata svojim po\v cetkom tako da zameni mesta 1. i 2. elementu, potom 3. i 4.
itd. Na primer, lista \verb|1->2->3->4| postaje \verb|2->1->4->3|. Lista \verb|1->2->3->4->5->6|
postaje \verb|2->1->4->3->6->5|. Lista mo\v ze sadr\v zati i neparan broj elemenata, pri
\v cemu u tom slu\v caju poslednji ostaje na svom mestu. Nije dozvoljeno formiranje
nove liste - lista se mo\v ze jedino preurediti u okviru postoje\' cih struktura.

\item Dato je binarno stablo koje sadr\v zi cele brojeve. Napisati funkciju
\verb|CVOR* nadjiNajblizi(CVOR* koren)| koja vra\' ca pokaziva\v c na \v cvor koji je najbli\v zi
korenu i pri tom je deljiv sa 3. Ako ima vi\v se \v cvorova na istom rastojanju od
korena koji zadovoljavaju svojstvo, vratiti pokaziva\v c na bilo koji.

Na primer, u binarnom stablu
\begin{verbatim}
      5
     / \
    4   11
   /   /  \
  3   8   13
         /  \
        6   10
\end{verbatim}

\v Cvorovi sa vrednostima 3 i 6 zadovoljavaju uslov, ali je 3 bli\v zi korenu, po\v sto
su potrebna dva poteza da bi se stiglo do njega, odnosno 3 poteza da bi se
stiglo do broja 6.

\item Data je datoteka \verb|apsolventi.txt|. U svakom redu datoteke nalaze se podaci o apsolventu:
  \emph{ime} (ne ve\' ce od 20 karaktera), \emph{prezime} (ne ve\' ce od 20 karaktera),
  \emph{broj preostalih ispita}. Datoteka je dobro formatirana i broj redova datoteke
   nije poznat. Potrebno je u\v citati podatke iz datoteke, odrediti prose\v can broj zaostalih ispita
   i potom ispisati imena i prezimena studenta koji imaju ve\' ci broj zaostalih ispita od prose\v cnog u datoteku
   \v cije ime je zadato kao argument komadne linije. NAPOMENA: koristiti strukturu
\begin{verbatim}
     typedef struct {
          char ime[20];
          char prezime[20];
          int br_ispita;
     } APSOLVENT;
\end{verbatim}
\end{enumerate}


\section{Programiranje 2, I smer, 2009/10, Zavr\v sni ispit (oktobar 2010)}


\begin{enumerate}
\item Argument programa je putanja tekstualne datoteke koja
sadr\v{z}i isklju\v{c}ivo cele brojeve. Napisati program koji
pronalazi i ispisuje na standardnom izlazu dva broja koja se
najmanje razlikuju (ako ima vi\v{s}e parova koji se isto razlikuju,
ispisati bilo koji par). Uputstvo: koristiti funkciju \verb|qsort|.

\item Neka je \v cvor binarnog stabla definisan kao

\begin{verbatim}
typedef struct cvorStabla
{
  int broj;
  struct cvorStabla* levi;
  struct cvorStabla* desni;
} CVOR;
\end{verbatim}

Napisati funkciju \verb|void obrni(CVOR* koren);| koja menja mesto levom
i desnom podstablu za svaki \v cvor. Na primer, stablo

\begin{verbatim}
    5
  /   \
 3     8
  \   / \
   4 6   9
\end{verbatim}

bi posle transformacije postalo:

\begin{verbatim}
     5
   /   \
  8     3
 / \   /
9   6 4
\end{verbatim}

Dozvoljeno je isklju\v civo reorganizovanje strukture drveta pomo\' cu postoje\' cih
pokaziva\v ca, ne i formiranje novog drveta.


\item Data je lista. Svaki \v cvor liste opisan je strukturom:

\begin{verbatim}
    typedef struct CVOR
    {
        int vr;
        struct CVOR* sled;
    }cvor;
\end{verbatim}

    Napisati funkciju \verb|void udvaja(cvor* lista, int br)| koja
    udvaja svako pojavljivanje elementa \verb|br| u listi \verb|lista|. \\
    Npr. za listu \verb|1->7->6->7->1->4->7| i br = 7 potrebno je dobiti listu: \verb|1->7->7->6->7->7->1->4->7->7|.
\end{enumerate}



\section{Zavr\v{s}ni ispit, Oktobar 2 2010.}


\begin{enumerate}
\item Date su dve liste. \v Cvor liste opisan je strukturom
\begin{verbatim}
    tupedef struct cvorliste
    {
        int vr;
        struct cvorliste *sled;
    }cvor;
\end{verbatim}

  Napisati funkciju \verb|void uredjuje(cvor **lista1, cvor* lista2)|
  koja izbacije iz liste \verb|lista1| sve elemente koji se nalaze
  u listi \verb|lista2|.

\item Ime tekstualne datoteke zadaje se kao argument komadne linije.
      U svakom redu datoteke nalazi se ime proizvoda (ne vi\v se od 20 karaktera)
      i koli\v cina u kojoj se proizvodi (broj redova datoteke nije poznat). Proizvodi su
      leksikografski pore\d ani. Sa standardnog ulaza u\v citava se ime proizvoda. Kori\v s\' cenjem
      sistemske funkcije \verb|bsearch| prona\' ci u kojoj meri se dati proizvod proizvodi.
      NAPOMENA: koristiti strukturu

\begin{verbatim}
        tupedef struct
        {
            char ime[20];
            int kolicina;
        }proizvod;
\end{verbatim}

\item Binarno drvo je perfektno balansirano akko za svaki \v cvor va\v zi da se broj \v cvorova levog
      i desnog podrveta razlikuje najvi\v se za jedan. Napisati funkciju koja proverava da li binarno drvo
      perfektno balansirano.
\end{enumerate}

