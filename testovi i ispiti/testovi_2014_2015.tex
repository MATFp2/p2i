% *************************************************************
% *************************************************************
% *************************************************************
\chapter{Testovi i ispiti 2014/15}
% *************************************************************
% *************************************************************
% *************************************************************


%---------------------------------------------------------------------
\section{Programiranje 2, I smer, 2014/15, prvi prakticni test}
%---------------------------------------------------------------------


\subsection{Grupa 1}

\begin{enumerate}

\item Sa standardnog ulaza se u\v citava ceo broj \emph{x}, ceo broj \emph{n} ($n \le 100$), a potom i niz od \emph{n} celih brojeva. Napisati rekurzivnu funkciju \emph{void f(int a[], int n, int x)}, koja u nizu \emph{a} poslavlja na nulu sve parove susednih elementa \v ciji je zbir (u po\v cetnom nizu) jednak \emph{x}. Rezultuju\' ci niz ispisati na standardni izlaz.



\small

\begin{tabular}{ |l|l|l|l|l| }

\hline

  Ulaz & \mlcell{3 9\\1 2 7 6 0 3 2 1 4} & \mlcell{2 2\\1 1} & \mlcell{5 5 \\8 2 3 2 9} & \mlcell{7 8\\ 1 5 2 6 3 4 3 9} \\ \hline

  Izlaz &  0 0 7 6 0 0 0 0 4 & 0 0& 8 0 0 0 9 & 1 0 0 6 0 0 0 9\\ \hline

\end{tabular}

\normalsize



\item Sa standardnog ulaza se u\v citava ceo broj \emph{n} ($n \le 100$), a potom i niz od \emph{n} celih brojeva, uredjenih rastu\' ce. Binarnom pretragom prona\' ci indeks prvog elementa koji je ve\' ci od prose\v cne vrednosti u nizu (element sa najmanjim indeksom koji ispunjava dato svojstvo). Ispisati dobijeni indeks na standardni izlaz (u slu\v caju da takav broj ne postoji ne pisati ni\v sta).  U slu\v caju gre\v ske ispisati -1 na standardni izlaz.



\small

\begin{tabular}{ |l|l|l|l|l| }

\hline

  Ulaz & \mlcell{10 \\ 5 10 12 20 100 101 102 596 703 1001}&\mlcell{8 \\ 1 2 8 10 65 102 104 500} & \mlcell{0 }& \mlcell{3 \\ 100 100 100}\\ \hline

  Izlaz &7 & 5 &  &  \\ \hline

\end{tabular}

\normalsize



\item U datoteci "kompleksni.txt" se nalaze kompleksni brojevi (njihov ta\v can broj nije unapred poznat). Svaki kompleksan broj je zadat sa dva broja tipa \emph{float}. Sortirati kompleksne

   brojeve nerastu\' ce prema veli\v cini modula. Dobijeni sortiran niz upisati u datoteku "sortirani\_kompleksni.txt". Maksimalan

   broj kompleksnih brojeva je 1000. U slu\v caju gre\v ske ispisati -1 na standardni izlaz.



\small

\begin{tabular}{ |l|l|l|l|l| }

\hline

  Ulaz &



  \mlcell{kompleksni.txt:\\19.89 3.56 \\ 6.87 19.05 \\ 10.21 11.32 \\ 10.23 2.78 \\ 6.04 17.19 \\ 19.55 15.53} &

  \mlcell{kompleksni.txt:\\4.22 12.63\\14.49 18.18\\5.76 6.91\\15.39 4.44}&

  & kompleksni.txt: \\ \hline

  Izlaz &

  \mlcell{sortirani\_kompleksni.txt:\\19.55 15.53\\6.87 19.05\\19.89 3.56\\6.04 17.19\\10.21 11.32\\10.23 2.78} &

  \mlcell{sortirani\_kompleksni.txt:\\14.49 18.18\\15.39 4.44\\4.22 12.63\\5.76 6.91}&

  -1 & sortirani\_kompleksni.txt:\\ \hline

\end{tabular}

\normalsize

\end{enumerate}



\subsection{Grupa 2}


\begin{enumerate}

\item Sa standardnog ulaza se u\v citava ceo broj \emph{n} ($n \le 100$), a potom i niz od \emph{n} celih brojeva.  Napisati rekurzivnu funkciju \emph{void f(int a[], int n)} koja u nizu \emph{a} postavlja na nulu sve elemente koji su (u po\v cetnom nizu) jednaki zbiru svojih suseda. Rezultuju\' ci niz ispisati na standardni izlaz.



\small

\begin{tabular}{ |l|l|l|l|l| }

\hline

  Ulaz & \mlcell{6 \\ 1 4 3 6 3 3} & \mlcell{5 \\1 0  3 3 0}& \mlcell{2 \\1 1} & \mlcell{6 \\1 2 1 1 3 2} \\ \hline

  Izlaz & 1 0 3 0 3 3 & 1 0 0 0 0 & 1 & 1 0 1 1 0 2 \\ \hline

\end{tabular}

\normalsize



\item Sa standardnog ulaza se u\v citava ceo broj \emph{k}, ceo broj \emph{n} ($n \le 100$), a potom i niz od \emph{n} celih brojeva, uredjenih rastu\' ce. Napisati funkciju koja binarnom pretragom nalazi indeks prvog $k$-tocifrenog elementa (element sa najmanjim indeksom koji ispunjava dato svojstvo). Ispisati dobijeni indeks na standardni izlaz (u slu\v caju da takav broj ne postoji ne pisati ni\v sta). U slu\v caju gre\v ske ispisati -1 na standardni izlaz.



\small

\begin{tabular}{ |l|l|l|l|l| }

\hline

  Ulaz & \mlcell{4 10 \\ 5 10 12 20 100 101 102 596 703 1001}&\mlcell{3 8 \\ 1 2 8 10 65 102 104 500} & \mlcell{5 0 }& \mlcell{4 3 \\ 100 100 100}\\ \hline

  Izlaz &9 & 5 &  &  \\ \hline

\end{tabular}

\normalsize



\item Datoteka "artikli.txt" sadr\v zi informacije o artiklima.

Format datoteke je takav da je najpre dat broj artikala, a potom u svakom

slede\' cem redu su date informacije o artiklu: \emph{naziv} (najvi\v se 20 karaktera), \emph{cena} (ceo broj), \emph{komada} (ceo broj).

Artikala nikad nema vi\v se od 200. U\v citati datoteku u niz struktura ARTIKAL,

a potom sortirati niz prema ukupnoj vrednosti artikala ($vrednost = cena \cdot komada$) u opadaju\' cem

poretku i ispisati ga na standardni izlaz. U slu\v caju gre\v ske ispisati -1 na standardni izlaz.



\small

\begin{tabular}{ |l|l|l|l|l| }

\hline

  Ulaz &

  \mlcell{artikli.txt:\\3\\ frizider 23450 2\\ pegla 4500 15\\ usisivac 7000 4}&\mlcell{artikli.txt:\\5\\ a1 162 130\\ a2 160 136\\a3 172 182\\a4 173 183\\a5 101 116} & \mlcell{}&

  \mlcell{artikli.txt:\\0} \\ \hline

  Izlaz &\mlcell{pegla 4500 15\\frizider 23450 2\\usisivac 7000 4 }& \mlcell{a4 173 183\\a3 172 182\\ a2 160 136\\a1 162 130\\ a5 101 116} & -1 & \\ \hline

\end{tabular}

\normalsize

\end{enumerate}



\subsection{Grupa 4}

\begin{enumerate}

\item Ceo broj \emph{x} se u\v citava sa standardnog ulaza. Napisati rekurzivnu funkciju \emph{int f(int x)} koja u datom broju $x$ uklanja sve cifre koje su (u po\v cetnom broju) jednake zbiru svojih suseda. Rezultat funkcije ispisati na standardni izlaz.



\small

\begin{tabular}{ |l|l|l|l|l| }

\hline

  Ulaz & 14363 & 11 & 13216 & 10330 \\ \hline

  Izlaz & 133 & 11 & 1216 & 100\\ \hline

\end{tabular}

\normalsize



\item Sa standardnog ulaza se u\v citava ceo broj \emph{x}, ceo broj \emph{n} ($n \le 100$), a potom i niz od \emph{n} celih brojeva, uredjenih rastu\' ce. Napisati funkciju koja u rastu\' ce uredjenom nizu celih brojeva pronalazi broj koji je najbli\v zi datom broju $x$. Ukoliko ima vi\v se takvih brojeva pronalazi onaj sa najmanjim indeksom. Funkcija vra\' ca vrednost

pronadjenog broja i treba da radi u vremenu $O(log(n))$. Ispisati dobijeni broj na standardni izlaz (u slu\v caju da takav broj ne postoji ispisati 0). U slu\v caju gre\v ske ispisati -1 na standardni izlaz.



\small

\begin{tabular}{ |l|l|l|l|l| }

\hline

  Ulaz & \mlcell{11 10 \\ 5 10 12 20 100 101 102 596 703 1001}&\mlcell{200 8 \\ 1 2 8 10 65 102 104 500} & \mlcell{5 0 }& \mlcell{54 3 \\ 100 100 100}\\ \hline

  Izlaz &10 & 104 & 0 & 100  \\ \hline

\end{tabular}

\normalsize



\item U datoteci \v cije se ime zadaje kao prvi argument komandne linije se nalaze pozitivni razlomci, u svakom redu po jedan, ne vi\v se od 256 (njihov ta\v can broj nije unapred poznat). Jedan razlomak je zadat kao par brojeva tipa \emph{float}.

   Koriste\' ci qsort, sortirati ih rastu\' ce, i tako sortirane ih upisati u datoteku \v cije se ime zadaje kao drugi argument komenandne linije. U slu\v caju gre\v ske ispisati -1 na standardni izlaz.



\small

\begin{tabular}{ |l|l|l|l|l| }

\hline

  Ulaz &

  \mlcell{a.out u.txt i.txt \\u.txt:\\14.49 18.18\\15.39 4.44\\4.22 12.63\\5.76 6.91} &

  \mlcell{a.out u.txt i.txt \\u.txt:\\ 1 2 \\ 3 4 \\ 5 0 \\ 6 7}&

  \mlcell{a.out u.txt i.txt }&

  \mlcell{a.out u.txt i.txt \\u.txt:}

  \\ \hline

  Izlaz &

  \mlcell{i.txt:\\4.22 12.63\\14.49 18.18\\5.76 6.91\\15.39 4.44} &

  -1 &

  -1 &

  sortirani\_kompleksni.txt:

  \\ \hline

\end{tabular}

\normalsize

\end{enumerate}



\subsection{Grupa 3}

\begin{enumerate}

\item Sa standardnog ulaza se u\v citava ceo broj \emph{n} ($n \le 100$), a potom i niz od \emph{n} celih brojeva. Napisati rekurzivnu funkciju \emph{int f(int a[], int n)} koja ra\v cuna proizvod svih neparnih brojeva prosledjenog celobrojnog niza. Rezultat funkcije ispisati na standardni izlaz.



\small

\begin{tabular}{ |l|l|l|l|l| }

\hline

  Ulaz & \mlcell{6 \\ 1 4 3 6 3 3} & \mlcell{4 \\ 2 4 6 8} & \mlcell{3 \\ 5 0 2}& 0\\ \hline

  Izlaz &  27 &1 & 5 & 1\\ \hline

\end{tabular}

\normalsize



\item Kao argumenti komandne linije zadaju se dva broja tipa \emph{float} $a$ i $b$ ($a \le b$). Sa standarnog ulaza se unosi 11 brojeva tipa \emph{float} (redom $a_0, a_1, a_2, ...,a_{10}$) koji predstavljaju koeficijente polinoma $a_{10}x^{10} + a_9x^9 + a_8x^8 + ... + a_0$. Napisati funkciju koja tra\v zi nulu polinoma na intervalu $[a, b]$. Pretpostaviti da \' ce na intervalu $[a, b]$ uvek postojati ta\v cno jedna nula funkcije, i da su vrednosti polinoma u ta\v ckama \emph{a} i \emph{b} razli\v citog znaka. Koristiti metod polovljenja intervala. Rezultat ispisati na standardni izlaz zaokru\v zen na dve decimale. U slu\v caju gre\v ske ispisati -1 na standardni izlaz.



\small

\begin{tabular}{ |l|l|l|l|l| }

\hline

  Ulaz & \mlcell{a.out 0 2 \\ -1 0 1 0 0 0 0 0 0 0 0}&\mlcell{a.out -7.5 1.3 \\ 1 0 -2 3.2 0 0 0 0 0 0 0} & \mlcell{a.out -1 0 \\ -4 0 5 -0.2 0 0 0 0 0 0 0}& \mlcell{a.out 20 30 \\ -4 0 5 -0.2 0 0 0 0 0 0 0}\\ \hline

  Izlaz & 1.00 & -0.52 & -0.88 & 24.97 \\ \hline

\end{tabular}

\normalsize



\item U prvom redu datoteke ''proizvodi.txt'' dat je broj $n$ ($n \le 1000$), a zatim u svakom od narednih rarednih $n$ redova naziv proizvoda i koli\v cina. Proizvod sa istim nazivom se mo\v ze pojaviti vi\v se puta u vi\v se razli\v citih redova u datoteci. Potrebno je u\v citati sve proizvode u niz (bez ponavljanja proizvoda), u kome \' ce se uz svaki proizvod \v cuvati njegova ukupna koli\v cina pro\v citana iz datoteke. Niz sortirati na osnovu ukupne koli\v cine rastu\' ce i ispisati ga na standarndi izlaz.  U slu\v caju gre\v ske ispisati -1 na standardni izlaz.



\small

\begin{tabular}{ |l|l|l|l|l| }

\hline

  Ulaz &

  \mlcell{proizvodi.txt:\\5\\p1 20\\p2 50\\p1 40\\p3 10\\p2 5}&\mlcell{proizvodi.txt:\\8\\p1 18 \\ p2 2 \\ p3 17 \\ p4 12 \\ p1 9 \\ p2 9 \\ p3 13 \\ p4 14} & \mlcell{}&

  \mlcell{proizvodi.txt:\\0} \\ \hline

  Izlaz &\mlcell{p3 10\\p2 55\\p1 60}& \mlcell{p2 11 \\ p4 26 \\ p1 27 \\ p3 30} & -1 & \\ \hline

\end{tabular}

\normalsize
\end{enumerate}





\section{Programiranje 2, I smer, 2013/14, Drugi prakti\v cni test}

\subsection{Prva grupa}

NAPOMENA: Na desktopu napraviti direktorijum sa imenom
\verb|inicijaliAsistenta_ImePrezime_brojIndeksa_1|.  U tom
direktorijumu \v cuvati zadatke -- 1.c, 2.c, 3.c. U drugom i tre\' cem
zadatku korisiti pomo\' cne funkcije iz liste.[hc], a u fajlovima 2.c
i 3.c napisati samo tra\v zenu i main funkciju.

\setcounter{z}{0}

\begin{z}
  Napisati funkciju {\tt unsigned int f1(unsigned int x)} koja u datom
  broju invertuje svaki tre\'ci bit. Prvi bit koji se invertuje je bit
  najmanje te\v zine.  Sa standardnog ulaza se unosi ceo pozitivan
  broj. Ispisati rezultat funkcije na standardni izlaz.
\end{z}
\begin{verbatim}
Ulaz:     0           345         1024        1
Izlaz:    1227133513  1227133712  1227134537  1227133512
\end{verbatim}

\begin{z}
  Napisati funkciju {\tt int f2(cvor* lista)} koja za elemente liste
  $a_1, a_2, ..., a_n$ ra\v cuna $a_1 - a_2 + a_3 - ... +
  (-1)^{n+1}a_n$. Dozvoljeno je dodati jo\v s jedan argument u
  funkciju f2.  Lista se u\v citava sa standardnog ulaza, sve dok se
  ne unese nula (koja se ne treba nalaziti u listi), elemenati se
  dodaju na kraj liste, a rezultat funkcije se ispisuje na standardni
  izlaz.
\end{z}

\begin{verbatim}
Ulaz:     4 2 5 3 8 0     12 3 3 24 25 0     4 6 2 1 2 4 0    0
Izlaz:    12              13                 -3               0
\end{verbatim}

\begin{z}
  Napisati funkciju {\tt void f3(cvor* lista, int k)} koja modifikuje
  zadatu listu tako \v sto iza svakog broja deljivog sa {\tt k}
  ume\'ce -1. Lista se u\v citava sa standardnog ulaza, sve dok se ne
  unese nula, potom se u\v citava k, a izmenjenu listu ispisati na
  standardni izlaz.
\end{z}
\begin{verbatim}
Ulaz:     4 2 5 3 8 0 2        12 3 3 24 25 0 3           4 6 2 1 2 4 0 1                   0 5
Izlaz:    4 -1 2 -1 5 3 8 -1   12 -1 3 -1 3 -1 24 -1 25   4 -1 6 -1 2 -1 1 -1 2 -1 4 -1
\end{verbatim}


\subsection{Drugi prakti\v cni test - Druga grupa}

NAPOMENA: Na desktopu napraviti direktorijum sa imenom
\verb|inicijaliAsistenta_ImePrezime_brojIndeksa_2|.  U tom
direktorijumu \v cuvati zadatke -- 1.c, 2.c, 3.c. U drugom i tre\' cem
zadatku korisiti pomo\' cne funkcije iz liste.[hc], a u fajlovima 2.c
i 3.c napisati samo tra\v zenu i main funkciju.

\setcounter{z}{0}

\begin{z}
  Napisati funkciju {\tt unsigned int f1(unsigned int x, unsigned int
    k, unsigned int p)} koja u datom broju invertuje prvih $k$ i
  poslednjih $p$ bitova. U slu\v caju da su $k$ i $p$ u zbiru dovoljno
  veliki, mo\v ze se desiti da neki bitovi budu dva puta
  invertovani. Bitovi broja se \v citaju sa desna na levo. Sa
  standardnog ulaza se unose celi pozitivni brojevi $x$, $k$ i
  $p$. Ispisati rezultat funkcije na standardni izlaz.
\end{z}
\begin{verbatim}
Ulaz:     0 2 3         23345 2 1         1024 1 4        1 3 2
Izlaz:    3758096387    2147506994        4026532865      3221225478
\end{verbatim}

\begin{z}
  Napisati funkciju {\tt int f2(cvor* lista, int k)} koja vra\'ca zbir
  elemenata u listi deljivih sa k. Lista se u\v citava sa standardnog
  ulaza, sve dok se ne unese nula (koja se ne treba nalaziti u listi),
  elemenati se dodaju na kraj liste, potom se u\v citava k, a rezultat
  funkcije se ispisuje na standardni izlaz.
\end{z}
\begin{verbatim}
Ulaz:     4 2 5 3 8 0 2    12 3 3 24 25 0 3     4 6 2 1 2 4 0 5     0 8
Izlaz:    14               42                   0                   0
\end{verbatim}

\begin{z}
  Napisati funkciju {\tt void f3(cvor* lista)} koja modifikuje zadatu
  listu tako \v sto iza svakog broja ume\'ce broj njegovih cifara.
  Lista se u\v citava sa standardnog ulaza, sve dok se ne unese nula,
  a izmenjenu listu ispisati na standardni izlaz.
\end{z}
\begin{verbatim}
Ulaz:     4 2 5 3 8 0            -12 312 3 24 25 0             1024 0      0
Izlaz:    4 1 2 1 5 1 3 1 8 1    -12 2 312 3 3 1 24 2 25 2     1024 4
\end{verbatim}


\subsection{Drugi prakti\v cni test - Tre\' ca grupa}

NAPOMENA: Na desktopu napraviti direktorijum sa imenom
\verb|inicijaliAsistenta_ImePrezime_brojIndeksa_3|.  U tom
direktorijumu \v cuvati zadatke -- 1.c, 2.c, 3.c. U drugom i tre\' cem
zadatku korisiti pomo\' cne funkcije iz liste.[hc], a u fajlovima 2.c
i 3.c napisati samo tra\v zenu i main funkciju.

\setcounter{z}{0}

\begin{z}
  Napisati funkciju {\tt unsigned int f1(unsigned int x, unsigned int
    k)} koja u datom broju invertuje svako k-to pojavljivanje
  jedinice.  Bitovi broja se \v citaju sa desna na levo. Sa
  standardnog ulaza se unosi ceo pozitivan broj $x$ i $k$. Ispisati
  rezultat funkcije na standardni izlaz.
\end{z}
\begin{verbatim}
Ulaz:     0 2         23345 2         1024 1         1 3
Izlaz:    0           4641            0              1
\end{verbatim}

\begin{z}
  Napisati funkciju {\tt int f2(cvor* lista1, cvor* lista2)} koja
  vra\'ca skalarni proizvod dve liste. Pretpostaviti da su liste iste
  du\v zine. Liste se u\v citavaju sa standardnog ulaza, sve dok se ne
  unese nula (koja se ne treba nalaziti u listi), elemenati se dodaju
  na kraj liste, a rezultat funkcije se ispisuje na standardni izlaz.
\end{z}
\begin{verbatim}
Ulaz:     1 2 3 0 4 5 6 0    10 9 0 4 3 0      4 0 5 0     0 0
Izlaz:    32                 67                20          0
\end{verbatim}

\begin{z}
  Napisati funkciju {\tt void f3(cvor* lista)} koja modifikuje zadatu
  listu tako \v sto ukljanja svaki broj koji je ve\' ci od svog
  predhodnika.  Lista se u\v citava sa standardnog ulaza, sve dok se
  ne unese nula, a izmenjenu listu ispisati na standardni izlaz.
\end{z}
\begin{verbatim}
Ulaz:     4 2 5 3 8 0    12 3 3 24 25 0   4 6 2 1 2 4 3 0    3 0
Izlaz:    4 2 3          12 3 3           4 2 1 3            3
\end{verbatim}

\section{Programiranje 2, 2014/2015, I smer, zavr\v{s}ni ispit, jun 1}
\subsection{Grupa 1}

NAPOMENA: Na desktopu napraviti direktorijum sa imenom \verb|InicijaliAsistenta_ImePrezime_BrojIndeksa_1|.
U tom direktorijumu \v cuvati zadatke -- 1.c, 2.c, 3.c, 4.c, 5.c\\
U direktorijumu {\tt g1} nalaze se funkcije za rad sa listama ({\tt liste.c} i {\tt liste.h}) i
funkcije za rad sa stablima ({\tt stabla.c} i {\tt stabla.h}).  \\

\bigskip

\begin{enumerate}
\item Kao argumenti komandne linije su zadata dva pravougaonika sa svoim dimenzijama, redom \v sirinom i visinom: \emph{s1 v1 s2 v2} (tipa $float$). Ispisati na standardni izlaz koliko najvi\v se prvih pravougaonika mo\v ze da stane u drugi, tako da su odgovaraju\' ce stranice paralelne (svaka stranica koja ozna\v cava \v sirinu prvog pravougaoniku je paralelna sa  stranicom koja  ozna\v cava \v sirinu drugog pravougaonika). U slu\v caju gre\v ske ispisati $-1$ na standardni izlaz.

\small
\begin{tabular}{ |l|l|l|l|l| }
\hline
  Ulaz & ./a.out 3.2 4.1 2.1 16.2 & ./a.out 2.1 3.2 9.8 9.1 & ./a.out 2 2 4 4 & ./a.out \\ \hline
  Izlaz & 0 & 8 & 4 & -1\\ \hline
\end{tabular}
\normalsize

\item U datoteci $duzi.txt$ se nalazi spisak du\v zi zadat ta\v ckama. Format datoteke je takav da je najpre zadat broj du\v zi, a pitom u svakom narednom redu du\v z u vidu \v cetiri koordinate: \emph{Ax Ay Bx By} (tipa $float$). Potrebno je u\v citati du\v zi iz datoteke, sortirati ih opadaju\' ce prema njihovoj du\v zini i ispisati tako sortirani niz na standardni izlaz. U svakom redu se ispisuju \emph{Ax Ay Bx By d}, gde je $d$ du\v zina du\v zi. Sve podatke ispisati zaokru\v zene na dve decimale. Koristiti dinami\v cku alokaciju memorije. U slu\v caju gre\v ske ispisati $-1$ na standardni izlaz.  Za koren broja tipa \emph{float} koristiti funkciju \emph{sqrtf}.

\small
\begin{tabular}{ |l|l|l|l|l| }
\hline
Ulaz
&\mlcell{$duzi.txt$:\\4\\2.09	7.33	9.12	1.58\\5.67	4.01	1.25	0.62\\6.73	8.61	1.88	8.49\\3.77	8.82	9.93	6.99}
&\mlcell{$duzi.txt$:\\4\\4 2 4 4 \\6 6 2 7\\5 5 1 9\\7 0 4 5}
&\mlcell{$duzi.txt$:\\4\\0 4 7 9\\1 7 7 0\\2 8 4 4\\4 1 6 8}
&\mlcell{}\\ \hline
Izlaz
&\mlcell{2.09 7.33 9.12 1.58 9.08\\3.77 8.82 9.93 6.99 6.43\\5.67 4.01 1.25 0.62 5.57\\6.73 8.61 1.88 8.49 4.85}
&\mlcell{7.00 0.00 4.00 5.00 5.83\\5.00 5.00 1.00 9.00 5.66\\6.00 6.00 2.00 7.00 4.12\\4.00 2.00 4.00 4.00 2.00}
&\mlcell{1.00 7.00 7.00 0.00 9.22\\0.00 4.00 7.00 9.00 8.60\\4.00 1.00 6.00 8.00 7.28\\2.00 8.00 4.00 4.00 4.47}
&\mlcell{-1} \\ \hline
\end{tabular}
\normalsize

\item Sa standardnog ulaza se unosi ceo broj $n$ ($n \le 32$), a zatim i niz od $n$ neozna\v cenih celih brojeva. Formirati neozna\v ceni ceo broj $x$ tako \v sto se na $i$-ti bit broja $x$ postavlja vrednost $i$-tog bita $i$-tog broja niza. Broj $x$ ispisati na standardni izlaz. Bitove broja \v citati od pozicija manje te\v zine ka pozicijama ve\' ce te\v zine. Podrazumevana vrednost bitova broja $x$ je $0$. U slu\v caju gre\v ske ispisati $-1$ na standardni izlaz.

\small
\begin{tabular}{ |l|l|l|l|l| }
\hline
Ulaz & 7 12 45 72 415 146 333 85 & 5 1024 64 128 31 511& 5 127 0 0 63 128 & 41\\ \hline
Izlaz & 88 & 24 & 9 & -1\\ \hline
\end{tabular}
\normalsize

\item Napisati funkcuju koja bri\v se svaki element liste koji je manji od sume svih prethodnih elemenata u listi. Prilikom ra\v cunanja sume uzeti u obzir i prethodno obrisane elemente. Kreirati glavni program koji u\v citava listu, poziva funkciju \emph{f4} i ispisuje dobijenu listu na izlaz. U slu\v caju gre\v ske ispisati $-1$ na standardni izlaz.

\small
\begin{tabular}{ |l|l|l|l|l| }
\hline
Ulaz & 1 2 3 1 2 3 9 30 0 &1 2 4 8 16 32 0& 51 27 84 28 62 3 28 0& 5 2 4 1 4 20 100 84 21 200 0\\ \hline
Izlaz & 1 2 3 30 & 1 2 4 8 16 32 & 51 84 & 5 20 100\\ \hline
\end{tabular}
\normalsize


\item Sa standardnog ulaza se u\v citavaju dva stabla, $s1$ i $s2$. Ispitati da li $s1$ i $s2$ imaju istu strukturu (dva stabla imaju istu strukturu ako se jedno mo\v ze u
potpunosti preklopiti preko drugog i obrnuto). U slu\v caju da nemaju istu strukturu ispisati $-1$ na standardni izlaz. U slu\v caju da imaju istu strukturu potrebno je izmeniti vrednosti u \v cvorovima stabla $s1$ tako \v sto se vrednost svakog \v cvora stabla $s1$ uve\' ca za vrednost odgovaraju\' ceg \v cvora stabla $s2$.  Izmenjeno stablo $s1$ ispisati na standardni izlaz. U slu\v caju gre\v ske ispisati $-1$ na standardni izlaz.

\small
\begin{tabular}{ |l|l|l|l|l| }
\hline
  Ulaz &
  \mlcell{10 5 15 12 13 0\\12 7 20 15 17 0} &
  \mlcell{10 5 15 12 11 0\\12 7 20 15 17 0} &
  \mlcell{30 15 46 11 0\\20 13 81 9 0}&
  \mlcell{10 8 0\\10 8 12 0} \\ \hline
  Izlaz &
  12 22 27 30 35 &
  -1&
  20 28 50 127&
  -1\\ \hline
\end{tabular}
\normalsize

\end{enumerate}


\subsection{Grupa 2}

NAPOMENA: Na desktopu napraviti direktorijum sa imenom \verb|InicijaliAsistenta_ImePrezime_BrojIndeksa_2|.
U tom direktorijumu \v cuvati zadatke -- 1.c, 2.c, 3.c, 4.c, 5.c\\
U direktorijumu {\tt g2} nalaze se funkcije za rad sa listama ({\tt liste.c} i {\tt liste.h}) i
funkcije za rad sa stablima ({\tt stabla.c} i {\tt stabla.h}).  \\

\bigskip

\begin{enumerate}
\item Kao argumenti komandne linije su zadate tri ta\v cke sa svoim koordinatama: \emph{x1 y1 x2 y2 x3 y3} (tipa $float$). Izra\v cunati du\v zinu puta koji po\v cinje u ta\v cki (0,0), prolazi kroz sve tri ta\v cke redom i zavr\v sava se u ta\v cki (0,0). Rezultat ispisati na standardni izlaz zaokru\v zen na dve decimale. U slu\v caju gre\v ske ispisati $-1$ na standardni izlaz. Za koren broja tipa \emph{float} koristiti funkciju \emph{sqrtf}.

\small
\begin{tabular}{ |l|l|l|l|l| }
\hline
  Ulaz & ./a.out 1 0 1 1 0 1 & ./a.out 3 4 -2 1 2 8 & ./a.out 2 -1 4 3 -2 1 & ./a.out 1 2 3\\ \hline
  Izlaz & 4.00 & 27.14 & 15.27 & -1 \\ \hline
\end{tabular}
\normalsize

\item U datoteci $proizvodi.txt$ se nalazi spisak proizvoda. Format datoteke je takav da je najpre zadat broj proizvoda, a zatim u svakom narednom redu naziv proizvoda (maksimalno 20 karaktera), cena i koli\v cina (tipa $float$).  Potrebno je u\v citati proizvode iz datoteke, sortirati ih opadaju\' ce prema ukupnoj vrednosti (\emph{cena * koli\v cina}) i ispisati tako sortirani niz na standardni izlaz. U svakom redu se ispisuju naziv, cena, koli\v cina i ukupna vrednost. Sve podatke ispisati zaokru\v zene na dve decimale. Koristiti dinami\v cku alokaciju memorije. U slu\v caju gre\v ske ispisati $-1$ na standardni izlaz.

\small
\begin{tabular}{ |l|l|l|l|l| }
\hline
Ulaz
&\mlcell{$proizvodi.txt$:\\4\\p1 2.09 7.33\\p2 5.67 4.01 \\p3 6.73 8.61\\p4 3.77 8.82}
&\mlcell{$proizvodi.txt$:\\4\\p1 4 2\\p2 6 6\\p3 5 5\\p4 7 0}
&\mlcell{$proizvodi.txt$:\\4\\p1 0 4\\p2 1 7\\p3 2 8\\p4 4 1}
&\mlcell{}\\ \hline
Izlaz
&\mlcell{p3 6.73 8.61 57.95\\p4 3.77 8.82 33.25\\p2 5.67 4.01 22.74\\p1 2.09 7.33 15.32}
&\mlcell{p2 6.00 6.00 36.00\\p3 5.00 5.00 25.00\\p1 4.00 2.00 8.00\\p4 7.00 0.00 0.00}
&\mlcell{p3 2.00 8.00 16.00\\p2 1.00 7.00 7.00\\p4 4.00 1.00 4.00\\p1 0.00 4.00 0.00}
&\mlcell{-1} \\ \hline
\end{tabular}
\normalsize

\item Napisati program koji sa standardnog ulaza u\v citava neozna\v cen ceo broj $x$ i tri broja $i$, $j$ i $k$. U broju $x$ razmeniti vrednosti dva bloka bitova du\v zine $k$, gde prvi blok po\v cinje bitom na poziciji $i$, a drugi bitom na poziciji $j$. Dobijeni broj ispisati na standardni izlaz. Bitove broja \v citati od pozicija manje te\v zine ka pozicijama ve\' ce te\v zine. U slu\v caju gre\v ske ispisati $-1$ na standardni izlaz. Gre\v skom smatrati preklapanja blokova, kao i ako neki blok isko\v ci van granica neozna\v cenog celog broja.

\small
\begin{tabular}{ |l|l|l|l|l| }
\hline
  Ulaz &
  1234567 0 4 3&
  341 2 17 5&
  2047 20 3 7&
  32567536 10 5 4\\ \hline
  Izlaz &
  1234672&
  2752769&
  133170183&
  32562576\\ \hline
\end{tabular}
\normalsize

\item Napisati funkcuju \emph{void f4(cvor *lista)} koja bri\v se svaki element liste koji je manji od prethodnog elemenata u listi, a ve\' ci od slede\' ceg. Prilikom brisanja uzeti u obzir i prethodno obrisane elemente. Prvi i poslednji element se ne bri\v su. Kreirati glavni program koji u\v citava listu, poziva funkciju \emph{f4} i ispisuje dobijenu listu na izlaz. U slu\v caju gre\v ske ispisati $-1$ na standardni izlaz.

\small
\begin{tabular}{ |l|l|l|l|l| }
\hline
  Ulaz &
  1 4 2 6 3 1 4 2 1 0&
  5 4 3 2 1 0&
 8 1 3 2 7 4 3 1 4 6 2 1 0&
  0\\ \hline
  Izlaz &
  1 4 2 6 1 4 1&
  5 1&
  8 1 3 2 7 1 4 6 1&
  \\ \hline
\end{tabular}
\normalsize

\item Napisati funkciju \emph{cvor* f5(cvor* s)} koja za svaki \v cvor u stablu menja redosled njegovog levog i desnog direktnog potomka ukoliko levo podstablo ima ve\' cu dubinu od desnog podstabla. Ispisati dobijeno stablo na izlazu. Dubina predstavlja najdu\v zi put od korena do lista. Kreirati glavni program koji u\v citava stablo, poziva funkciju \emph{f5} i ispisuje dobijeno stablo na izlaz. U slu\v caju gre\v ske ispisati $-1$ na standardni izlaz.

\small
\begin{tabular}{ |l|l|l|l|l| }
\hline
  Ulaz &
  20 10 30 5 12 7 0&
  10 5 15 3 1 0&
  20 10 30 9 15 25 12 17 0&
  10 5 3 8 7 0\\ \hline
  Izlaz &
  30 20 12 10 5 7&
  15 10 5 3 1&
  30 25 20 9 10 12 15 17&
  10 3 5 8 7\\ \hline
\end{tabular}
\normalsize

\end{enumerate}


\section{I smer, Programiranje 2 2014/2015, zavr\v{s}ni ispit, jun2 2015}

Na \textit{Desktop}-u napraviti direktorijum \v cije je ime u formatu \textbf{InicijaliAsistenta\_ImeIPrezime\_BrojIndeksa\_1}. Na primer, \textbf{AZ\_PeraPeric\_mi14231\_1}. Sve zadatke sa\v cuvati u ovom direktorijumu. Zadatke imenovati sa \textbf{1.c}, \textbf{2.c}, \textbf{3.c}, \textbf{4.c} i \textbf{5.c}.

\bigskip

\begin{enumerate}
\item
Kao argument komadne linije zadaje se jedna re\v c. Ispisati
na standarni izlaz re\v c
koja se dobije od zadate re\v ci tako \v sto se prvo slovo ponovi jednom, drugo dva puta, ..., \emph{n}-to \emph{n} puta. U slu\v caju gre\v ske
ispisati -1 na standardni izlaz.

\small
\begin{tabular}{ |l|l|l|l|l| }
\hline
  Ulaz & ./a.out petar & ./a.out 12345 & ./a.out p2ispit & ./a.out \\ \hline
  Izlaz & peetttaaaarrrrr & 122333444455555 & p22iiisssspppppiiiiiittttttt & -1\\ \hline
\end{tabular}
\normalsize

\item
U datoteci \textbf{polinomi.txt} se nalaze polinomi zadati svojim koeficijentima. Prvo se zadaje ukupan broj polinoma,
a zatim u svakom narednom redu po jedan polinom. Svaki od tih redova sadr\v zi ime
polinoma (maksimalne du\v zine 20 karaktera), broj koeficijenata polinoma (ceo neozna\v cen broj \emph{n}) i koeficijente (tipa \emph{float}, ukupno \emph{n} njih).
Sortirati polinome opadaju\'ce prema
vrednosti u ta\v cki \emph{x} koja je prosle\d ena programu kao argument komandne linije (tipa \emph{float}).
Ispisati imena i vrednosti sortiranih polinoma na standarni izlaz.
Sve podatke ispisati zaokru\v zene na dve decimale.
Koristiti dinami\v cku alokaciju memorije. U slu\v caju gre\v ske ispisati -1 na standardni izlaz.

\small
\begin{tabular}{ |l|l|l|l|l| }
\hline
Ulaz
&\mlcell{a.exe 1\\$polinomi.txt$:\\5\\x\^{}2+3x+5.1~~3 5.1 3 1\\x\^{}3+3x+5~~~~4 5 3 0 1\\x\^{}2-8~~~~~~~~~~~3 -8 0 1\\8.5x~~~~~~~~~~~~~2 0 8.5\\12x-4~~~~~~~~~~~2 -4 12}
&\mlcell{a.exe 3\\$polinomi.txt$:\\5\\x\^{}2+3x+5.1~~3 5.1 3 1\\x\^{}3+3x+5~~~~4 5 3 0 1\\x\^{}2-8~~~~~~~~~~~3 -8 0 1\\8.5x~~~~~~~~~~~~~2 0 8.5\\12x-4~~~~~~~~~~~2 -4 12}
&\mlcell{a.exe -3\\$polinomi.txt$:\\5\\x\^{}2+3x+5.1~~3 5.1 3 1\\x\^{}3+3x+5~~~~4 5 3 0 1\\x\^{}2-8~~~~~~~~~~~3 -8 0 1\\8.5x~~~~~~~~~~~~~2 0 8.5\\12x-4~~~~~~~~~~~2 -4 12}
&\mlcell{a.exe}\\ \hline
Izlaz
&\mlcell{x\^{}2+3x+5.1~~~9.10\\x\^{}3+3x+5~~~9.00\\8.5x~~~8.50\\12x-4~~~8.00\\x\^{}2-8~~~-7.00}
&\mlcell{x\^{}3+3x+5~~~41.00\\12x-4~~~32.00\\8.5x~~~25.50\\x\^{}2+3x+5.1~~~23.10\\x\^{}2-8~~~1.00}
&\mlcell{x\^{}2+3x+5.1~~~5.10\\x\^{}2-8~~~1.00\\8.5x~~~-25.50\\x\^{}3+3x+5~~~-31.00\\12x-4~~~-40.00}
&\mlcell{-1} \\ \hline
\end{tabular}
\normalsize



\item
Sa standarnog ulaza u\v citava se neozna\v cen ceo broj \emph{x}, neozna\v cen ceo broj \emph{n}
i niz od \emph{n} celih neozna\v cenih brojeva ($n \le 32$). Odrediti neozna\v cen ceo broj \emph{y} koji se dobija na slede\'ci na\v cin: porede se
\emph{i}-ti bit broja \emph{x} i \emph{i}-ti bit \emph{i}-tog broja niza. Ukoliko su jednaki na
\emph{i}-to mesto broja \emph{y} se postavlja bit 1, a ina\v ce se postavlja 0 (ukoliko \emph{i}-ti broj niza ne postoji, podrazumevati da je vrednost odgovaraju\' ceg bita 0).
Ispisati broj \emph{y} na standardni izlaz. U slu\v caju gre\v ske ispisati -1 na standardni izlaz.

\small
\begin{tabular}{ |l|l|l|l|l| }
\hline
  Ulaz & 1023 7 0 0 1023 1023 0 0 0 & 348712 4 1235 964914 24214 4212 & 12345 0 &726431 2 4967 349672 \\ \hline
  Izlaz & 4294966284 & 4294618576 & 4294954950 & 4294240865\\ \hline
\end{tabular}
\normalsize

\item
Napisati funkciju \emph{void f4(cvor* lista, int k)} koja u datoj listi izme\d u svaka dva elementa
\v ciji su zbir ili razlika jednaki datom broju \emph{k} ume\' ce -1. Glavni program u\v citava listu i ceo broj \emph{k}.
Potrebno je ispisati rezultuju\' cu listu na stadardni izlaz. Nije dozvoljeno kori\v s\'cenje pomo\'cne liste. U slu\v caju gre\v ske ispisati -1 na standardni izlaz.
Ne analizirati prvi i poslednji element liste jer oni nemaju oba suseda.

\small
\begin{tabular}{ |l|l|l|l|l| }
\hline
  Ulaz & 1 2 3 1 2 3 0 3 & 4 2 1 5 6 2 4 0 2 & 1 3 1 1 3 1 0 2 & 0 5 \\ \hline
  Izlaz & 1 -1 2 3 1 -1 2 3 & 4 -1 2 1 5 6 2 -1 4 & 1 -1 3 -1 1 -1 1 -1 3 -1 1 & \\ \hline
\end{tabular}
\normalsize

\item
Napisati funkciju \emph{int f5(cvor* s, int k)} koja ra\v cuna zbir svih parnih elemenata stabla \emph{s} na nivou \emph{k}, umanjen za zbir svih neparnih elemenata stabla \emph{s} na nivou \emph{k}.
Glavni program u\v citava stablo i ceo broj \emph{k}. Potrebno je ispisati rezultat funkcije \emph{f5} na standardni izlaz. U slu\v caju gre\v ske ispisati -1 na standardni izlaz.

\small
\begin{tabular}{ |l|l|l|l|l| }
\hline
  Ulaz &
  20 10 30 5 12 7 0 3&
  10 5 15 3 1 0 2&
  20 10 30 9 15 25 12 17 0 3&
  10 5 3 8 7 0 10\\ \hline
  Izlaz &
  7&
  -20&
  -49&
  0\\ \hline
\end{tabular}
\normalsize

\end{enumerate}


\section{I smer, Programiranje 2 2014/2015, zavr\v{s}ni ispit, septembar 2015}

Na \textit{Desktop}-u napraviti direktorijum \v cije je ime u formatu
\textbf{InicijaliAsistenta\_ImePrezime\_BrojIndeksa\_1}. Na primer,
\textbf{AZ\_PeraPeric\_mi14231\_1}. Sve zadatke sa\v cuvati u ovom
direktorijumu. Zadatke imenovati sa \textbf{1.c}, \textbf{2.c},
\textbf{3.c}, \textbf{4.c} i \textbf{5.c}. U poslednja dva zadatka
koristiti prilo\v zene biblioteke za rad sa listama (liste.[hc]) i
stablima (stabla.[hc]) i kompilirati ih iz dve C datoteke.

\begin{enumerate}
\item 
Kao argument komadne linije zadaju se tri parametra -- re\v c, slovo,
broj. Izmeniti re\v c tako da se izme\d u prva dva pojavljivanja datog
slova u re\v ci svaki karakter uve\'ca za dati broj.  U slu\v caju
gre\v ske ispisati {\tt -1}.
\begin{verbatim}
  Primer 1:               Primer 2:             Primer 3:             Primer 4:
  ./a.out danas a 3       ./a.out danas n 3     ./a.out oktobar 50    ./a.out proGramiraNjer r 5

  daqas                   danas                 -1                    prtLramiraNjer
\end{verbatim}


\item
U datoteci {\tt bioskop.txt} se nalaze podaci o filmovima koji se
prikazuju u bioskopu. Ukupan broj filmova nije unapred poznat. Podaci
su zapisani na slede\'ci na\v cin: {\tt ime\_filma} (jedna re\v c, ne
du\v za od 50 kraktera), {\tt vreme\_prikazivanja} (vreme je oblika
{\tt HH:MM}).  Pretpostaviti da su podaci u datoteci ispravno zadati.
Sortirati podatke rastu\'ce prema vremenu prikazivanja i na standarni
izlaz ispisati imena filmova tako sortiranog niza. Potom ispisati onaj
sat ({\tt HH:00}) u kome ima najvi\v se projekcija.  U slu\v caju
gre\v ske ispisati {\tt -1}.
\begin{verbatim}
 Primer 1:                           Primer 2:              Primer 3:               Primer 4:
 Fantasticna_cetvorka 22:40          (nema datoteke)        (prazna datoteka)       Poseta 20:30
 Noc_u_muzeju_2 14:00 
 Bekstvo_iz_Sosenka 20:20
 Gradovi_na_papiru 19:00
 Haos_u_najavi 19:20
 Malci 19:10
 
 Noc_u_muzeju_2                      -1                     -1                      Poseta
 Gradovi_na_papiru                                                                  20:00
 Malci
 Haos_u_najavi
 Bekstvo_iz_Sosenka
 Fantasticna_cetvorka
 19:00
\end{verbatim}

\item
  Napisati funkciju koja na osnovu neozna\v cenog broja $x$ formira
  nisku $s$ koja sadr\v zi heksadekadni zapis broja $x$, koriste\' ci
  algoritam za brzo prevo\d enje binarnog u heksadekadni zapis (svake
  $4$ binarne cifre se zamenjuju jednom odgovaraju\' com heksadekadnom
  cifrom).  Napisati program koji tu funkciju testira za broj koji se
  zadaje sa standardnog ulaza.
\begin{verbatim}
Primer 1:                Primer 2:                Primer 3:                Primer 4:
11                       1024                     12345                    123456789
0000000B                 00000400                 00003039                 075BCD15
\end{verbatim}

\item
Napisati funkciju {\tt cvor* f4(cvor* lista, int k)} koja u datoj
listi izbacuje susedne elemente \v ciji zbir je jednak datom broju
{\tt k}. Potrebno je ispisati tako dobijenu listu na stadardni
izlaz. Nije dozvoljeno kori\v s\'cenje pomo\'cne liste. Nije dovoljno
samo ispisati tra\v zenu listu ve\' c je potrebno elemente zaista
izbaciti i konstruisati novu listu. Elementi liste su celi brojevi,
lista se unosi sa standarnog ulaza sve dok se ne unese {\tt 0}. Nakon
unosa elemenata liste unosi se broj {\tt k}.
\begin{verbatim}
Primer 1:             Primer 2:              Primer 3:              Primer 4:
13 4 5 10 0 9         13 4 5 9 9 0 9         13 4 5 4 3 0 9         4 5 3 -2 11 -2 11 -2 0 9

13 10                 13 9 9                 13 3                   3 
\end{verbatim}

\item
Napisati funkciju {\tt int f5(cvor* stablo)} koja u datom stablu
odre\d uje broj onih elemenata kod kojih je zbir cifara svih elemenata
levog podstabla strogo ve\'ci od zbira cifara svih elemenata desnog
podstabla. Testirati funkciju pozivom u main-u. Stablo se unosi sa
standardnog ulaza sve dok se ne unese 0. Elementi stabla su celi
pozitivni brojevi.
\begin{verbatim}
Primer 1:                        Primer 2:         Primer 3:                         Primer 4:
52 38 64 21 40 55 103 88 0       304 0             104 88 110 78 99 105 120 55 0     1111 -100 0

1                                0                 4                                 1
\end{verbatim}
\end{enumerate}
