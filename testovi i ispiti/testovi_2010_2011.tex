% *************************************************************
% *************************************************************
% *************************************************************
\chapter{Testovi i ispiti 2010/11}
% *************************************************************
% *************************************************************
% *************************************************************

%---------------------------------------------------------------------
\section{Programiranje 2, I smer, 2010/11, Test 2 (zadaci)}
%---------------------------------------------------------------------
\subsection{I grupa}
\begin{enumerate}
 \item Napisati rekurzivnu funkciju koja iz datog broja n izbacuje pojavljivanja svih parnih cifara
       koje su se nalaze na parnim mestima broja n, i svih neparnih cifara koje se nalaze na neparnim
       mestima broja n. Mesto cifre u broju \v citati sdesna na levo, po\v cev od indeksa 1. Odrediti
       vremensku slo\v zenost algoritma.
\end{enumerate}

\subsection{II grupa}
\begin{enumerate}
 \item Napisati rekurzivnu funkciju koja za zadato k i n crta ``stepenice``. Svaka stepenica je \v sirine k,
      a ima n stepenika. Na primer k = 4, n = 3:
\begin{verbatim}
****
    ****
        ****
\end{verbatim}
Izra\v cunati vremensku slo\v zenost algoritma.
\end{enumerate}

\subsection{III grupa}
\begin{enumerate}
 \item  Dat je broj n i neka su $a_1, \ldots, a_k$ sleva na desno cifre broja n. Napisati rekurzivnu funkciju koja izra\v cunava:
      $a_1 + 2*a_2 + \ldots + k*a_k$.  Izra\v cunati vremensku slo\v zenost.
\end{enumerate}


%---------------------------------------------------------------------
\section{Programiranje 2, I smer, 2010/11, Test 3 i Test 4 (zadaci)}
%---------------------------------------------------------------------

\subsection{I grupa}
\begin{enumerate}
 \item Napisati rekurzivnu funkciju koja svaku parnu cifru $c$ u broju $n$ zamenjuje sa $c/2$.
       Napisati glavni program koji kao argument komandne linije dobija broj $n$, a na standardni izlaz
       ispisuje novi broj.

  \item Sa standardog ulaza se zadaje ime tekstualne datoteke koja sadr\v zi podatke o
    artiklima prodavnice. Datoteka je u formatu:\\
    \verb|<bar kod> <ime artikla> <proizvodjac> <cena>|
    i sortirana je prema \verb|<bar kod>|.
    Nije unapred po-znat broj artikala u datoteci.
    U\v citati podatke o artiklima u niz (niz alocirati dinami\v cki).
    Zatim se sa standardnog ulaza unose bar kodovi artikla sve dok se ne unese 0.
    Izra\v cunati ukupnu cenu unetih proizvoda.
    (koristiti ugradjenu f-ju \verb|bsearch| za tra\v zenje artikla sa datim bar kodom).
\end{enumerate}


\subsection{II grupa}
\begin{enumerate}
 \item Napisati rekurzivnu funkciju koja svaku parnu cifru $c$ u broju $n$ zamenjuje sa $c/2$.
       Napisati glavni program koji kao argument komandne linije dobija broj $n$, a na standardni izlaz
       ispisuje novi broj.

  \item Sa standardog ulaza se zadaje ime tekstualne datoteke koja sadr\v zi podatke o
    artiklima prodavnice. Datoteka je u formatu:\\
    \verb|<bar kod> <ime artikla> <proizvodjac> <cena>|
    i sortirana je prema \verb|<bar kod>|.
    Nije unapred po-znat broj artikala u datoteci.
    U\v citati podatke o artiklima u niz (niz alocirati dinami\v cki).
    Zatim se sa standardnog ulaza unose bar kodovi artikla sve dok se ne unese 0.
    Izra\v cunati ukupnu cenu unetih proizvoda.
    (koristiti ugradjenu f-ju \verb|bsearch| za tra\v zenje artikla sa datim bar kodom).
\end{enumerate}



%---------------------------------------------------------------------
\section{Programiranje 2, I smer, 2010/11, Test 7 (zadaci)}
%---------------------------------------------------------------------

\subsection{I grupa}
\begin{enumerate}
 \item Napisati f-ju koja invertuje n bitova po\v cev\v si od pozicije p u broju x.

 \item Data je struktura student:
\begin{verbatim}
typedef struct
{
    char ime[20];
    char prezime[20];
}student;
\end{verbatim}
Napisati funkciju \\ \verb|void pronadji(student *studenti, int n, int *max_p, int *max_p_i)|, koja bibliote\v ckim qsort--om, sortira niz, prvo prema
pema prezimenu, a zatim i prema imenu, a zatim u sortiranom nizu studenata pronalazi maksimalan broj studenata koji imaju isto prezime i
 maksimalan broj studenata koji imaju isto ime i prezime. Ove podatke vra\' ca kroz prosle\d ene parametre max\_p i max\_p\_i.

\item Napisati funkciju \verb|Cvor* dodaj_u_listu(Cvor *glava, Cvor *novi, int n)| koja dodaje novi \v cvor na n-to mesto u listi. Ukoliko lista ima
      manje od n elemenata novi \v cvor se dodaje na kraj liste. Kao rezultat funkcija vra\' ca adresu nove glave liste.

\end{enumerate}

\subsection{II grupa}
\begin{enumerate}
 \item Napisati funkciju koja izra\v cunava razliku suma bitova na parnim i neparnim pozicijama broja x.

 \item Definisati tip podataka koji defini\v se ta\v cku u ravni. Sa standardnog ulaza
se unosi broj ta\v caka , a zatim i njihove koordinate. Maximalni broj ta\v caka
nije unapred poznat.
Sortirati dati niz ta\v caka na osnovu rastojanja od koordinatnog po\v cetka.


\item Napisati funkciju \verb|void umetni(cvor* lista)| koja izme\d u svaka dva elementa u listi ume\' ce element koji predstavlja razliku susedna dva.

\end{enumerate}


%---------------------------------------------------------------------
\section{Programiranje 2, I smer, 2010/11, zavr\v{s}ni ispit, juni 2011}
%---------------------------------------------------------------------


\begin{enumerate}
\item \begin{description}
      \item{a)} Definisati tip podataka za predstavljanje studenata, za svakog studenta poznato je: nalog na Alasu (oblika napr. mr97125, mm09001),
        ime (maksimalne du\v zine 20 karaktera) i broj poena.
      \item{b)} Podaci o studentima se nalaze u datoteci \verb|studenti. txt| u obliku: nalog ime br.poena. Kao argument komadne linije korisnik unosi
        opciju i to mo\v ze biti \verb|-p| ili \verb|-n|. Napisati program koji sortira kori\v s\' cenjem ugra\d ene funkcije \verb|qsort|
        studente i to: po broju poena ako je prisutna opcija \verb|-p|, po nalogu ukoliko je prisutna opcija \verb|-n|, ili po imenu ako nije
        prisutna ni jedna opcija. Studenti se po nalogu sortiraju tako \v sto se sortiraju na osnovu godine, zatim na osnovu smera i na kraju na
        osnovu broja indeksa.
      \end{description}

\item Grupa od n plesa\v ca (na \v cijim kostimima su u smeru kazaljke na satu redom brojevi od 1 do n) izvodi svoju plesnu ta\v cku tako \v sto
      formiraju krug iz kog najpre izlazi k-ti plesa\v c (odbrojava se pov cev od plesa\v ca ozna\v cenog brojem 1 u smeru kretanja kazaljke na satu).
      Preostali plesa\v ci obrazuju manji krug iz kog opet izlazi k-ti plesa\v c (odbrojava se po\v cev os slede\' ceg suseda prethodno izba\v cenog,
      opet u smeru kazaljke na satu). Izlasci iz kruga se nastavljaju sve dok svi plesa\v ci ne budu isklju\v ceni. Celi brojevi \verb|n, k, (k<n)| se
      u\v citavu sa standardnog ulaza. Napisati program koji \' ce na standardni izlaz ispisati redne brojeve plesa\v ca u redosledu napu\v stanja kruga.
      PRIMER: za \verb|n = 5, k = 3| redosled izlazaka je 3 1 5 2 4. NAPOMENA: Zadatak re\v siti kori\v s\' cenjem kru\v zne liste.

\item Ime datoteke je argument komadne linije. U svakom redu datoteke nalazi se podatak: broj ili ime druge datoteke (zavrsava se sa .txt). Svaka
      datoteka je istog sadr\v zaja. Napisati program koji pose\' cuje sve datoteke do kojih mo\v ze sti\' ci iz prve (rekurzivno), pri tome ne
      pose\' cuje istu datoteku dva puta i \v stampa na standardni izlaz brojeve iz njih. NAPOMENA: Za pam\' cenje imena datoteka koristiti
      binarno pretra\v ziva\v cko drvo.

\end{enumerate}


%---------------------------------------------------------------------
\section{Programiranje 2, I smer, 2010/11, zavr\v{s}ni ispit, septembar 2011}
%---------------------------------------------------------------------


\begin{enumerate}
\item \begin{description} \item{a)} Napisati funkciju \verb|int skalarni_proizvod(int* a,int* b, int n)|
        koja ra\v cuna skalarni proizvod vektora a i b du\v zine n. (Sklarni proizvod dva vektora $a = (a_1, \ldots, a_n)$
        i $b = (b_1, \ldots, b_n)$ je suma $S = a_1\cdot b_1 + a_2\cdot b_2 + \ldots + a_n\cdot b_n$)
    \item{b)} Napisati funkciju \verb|int ortonormirana(int** A, int n)| kojom se proverava da li je
              zadata kvadratna matrica A dimenzije $n \times n$ ortonormirana. Matrica je ortonormirana ako je skalarni proizvod
              svakog para razli\v citih vrsta jednak 0, a skalarni proizvod vrste sa samom sobom 1. Funkcija
              vra\' ca 1 ukoliko je matrica ortonormirana, 0 u suprotnom.
    \item{c)} Napisati glavni program u kome se dimenzija matrice \verb|n| zadaje kao argument komandne linije.
              Nakon toga se elementi matrice u\v citavaju sa standardnog ulaza i pozivom funkcije se utvr\d uje
              da li je matrica ortonormirana. Maksimalna dimenzija matrice nije unapred poznata.
  \end{description}

\item  Iz datoteke \verb|brojevi.txt| se u\v citavaju celi brojevi u niz i nije poznato koliko ima brojeva
      u datoteci. Brojeve sortirati pozivom ugra\d ene funkcije \verb|qsort| pa onda ugra\d enom funkcijom
      \verb|bsearch| prona\' ci ceo broj koji se zadaje sa standardnog ulaza.

\item Dva binarna stabla su identi\v cna ako su ista po strukturi i sadr\v zaju, odnosno oba korena imaju
        isti sadr\v zaj i njihova odgovaraju\' ca podstabla su identi\v cna. Napistati funkciju koja proverava
        da li su dva binarna stabla identi\v cna.

\end{enumerate}



%---------------------------------------------------------------------
\section{Programiranje 2, I smer, 2010/11, zavr\v{s}ni ispit, oktobar 2011}
%---------------------------------------------------------------------

\begin{enumerate}

\item U svakom redu datoteke \verb|transakcije.txt| nalazi se identifikacija (niska maksimalne du\v zine 20) korisnika banke i iznos transakcije koju je korisnik napravio (ceo broj). Jedan korisnik mo\v ze imati vi\v se transakcija, a svaka je predstavljena celim brojem (negativan - isplata sa ra\v cuna, pozitivan - uplata na ra\v cun). Ispisati identifikacioni broj korisnika koji je najvi\v se zadu\v zen. \\
NAPOMENA: Kreirati strukturu \verb|klijent|, u\v citati sve korisnike u dinamicki alociran niz koji je u svakom trenutku sortiran po identifikaciji (obavezno koristiti ugra\d enu funkciju \verb|qsort| i za pronala\v zenje korisnika ugra\d enu funkciju \verb|bsearch|), a zatim u jednom prolasku kroz niz na\' ci najzadu\v zenijeg korisnika.

\item Napisati funkciju koja sa\v zima listu tako \v sto izbacuje svaki element koji se vi\v se puta pojavljuje u listi. \\
PRIMER: zadata lista: 1 3 8 3 1 2 3 6; rezultat: 8 2 6
dj
\item Neka je dat pokaziva\v c na koren binarnog stabla \v ciji \v cvorovi sadr\v ze cele brojeve. Napisati slede\' ce funkcije:
\begin{description}
\item{(a)} Funkciju koja vra\' ca broj \v cvorova koji su po sadr\v zaju ve\' ci od svih svojih potomaka.
\item{(b)} Funkciju koja ispisuje \v cvorove koji su ve\' ci od sume svih svojih potomaka.
\end{description}

\end{enumerate}

