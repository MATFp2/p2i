% *************************************************************
% *************************************************************
% *************************************************************
\chapter{Testovi i ispiti 2008/09}
% *************************************************************
% *************************************************************
% *************************************************************

\newpage

\section{Test 1, 01.04.2009.}


\begin{enumerate}
\item
Napisati program koji sa standardnog ulaza u\v citava pozitivne cele
brojeve dok ne u\v cita nulu kao oznaku za kraj. Na standardni izlaz
ispisati koji broj se pojavio najvi\v se puta me\d u tim brojevima.
Na primer, ako se na ulazu pojave brojevi:
\verb+2 5 12 4 5 2 3 12 15 5 6 6 5+
program treba da vrati broj $5$.
Zadatak re\v siti kori\v{s}\'cenjem dinami\v cke realokacije i uz kori\v s\' cenje struktura.

\item
Napisati rekurzivnu i iterativnu funkciju koja za uneto $n$ sa standardnog ulaza ra\v cuna $f(n)$ ako je $f(n)$ definisan na slede\' ci na\v cin:

$f(1)=1$, $f(2)=2$, $f(3)=3$, $f(n+3)=f(n+2)+f(n+1)+f(n)$, za $n>0$.
Napisati i program koji poziva ove dve funkcije.

\item
Napisati program koji sa standardnog ulaza unosi prvo dimenziju matrice
($n<10$)
pa zatim elemente matrice i izra\v cunava sumu elemenata iznad sporedne
dijagonale matrice.
Napisati funkciju koja ra\v cuna sumu elemenata iznad glavne dijagonale
koja se izvr\v sava u \v sto manjem broju koraka.

Primer:
\begin{verbatim}
4
1 2 3 4
5 6 7 8
9 1 2 3
4 5 6 7
\end{verbatim}

Suma elemenata iznad sporedne dijagonale je: $1+2+3+5+6+9=26$
\end{enumerate}




\section{Test 2, 06.06.2009.}



\begin{enumerate}
\item
Napisati funkcije za rad sa stekom \v{c}iji su elementi celi
brojevi. Treba napisati (samo) funkcije za dodavanje elementa u
stek \verb+void push(cvor** s, int br)+ , brisanje elementa iz
steka \verb+void pop(cvor** s)+ i funkciju za izra\v cunavanje
zbira svih elemenata koji su parni brojevi
\verb+int zbir_parnih(cvor* s)+.

\item Napisati funkcije za rad sa ure\d enim binarnim stablom \v{c}iji su
elementi celi brojevi. Treba napisati (samo) funkcije za dodavanje
elementa u stablo \verb+void dodaj(cvor** s, int br)+ , brisanje
stabla \verb+void obrisi(cvor* s)+ i funkciju za izra\v cunavanje
zbira listova stabla \verb+int zbir_listova(cvor* s)+.

\item
Napisati program koji radi sa ta\v ckama. Ta\v cka je
predstavljena svojim $x$ i $y$ koordinatama (celi brojevi). Sa
standardnog ulaza se u\v citava prvo broj ta\v caka a zatim
koordinate ta\v caka. Dobijeni niz struktura sortirati pozivom
funkcije \emph{qsort}. Niz sortirati po $x$ koordinati, a ako neke dve
ta\v cke imaju istu $x$ koordinatu onda ih sortirati po $y$ koordinati.

Ako na ulazu dobijete niz: $(4,6)$, $(2,9)$, $(4,5)$; sortirani niz \' ce
izgledati: $(2,9)$, $(4,5)$, $(4,6)$.

\end{enumerate}




\section{Zavr\v{s}ni ispit, juni 2009.}



Napisati program koji iz datoteke "reci.txt" \v cita redom sve
re\v ci (maksimalne du\v zine $20$) i smesta ih u:

\begin{enumerate}
\item Niz struktura (pretpostaviti da razli\v citih re\v ci u datoteci nema
vi\v se od $100$) u kome \' ce se za svaku re\v c \v cuvati i njen
broj pojavljivanja.

\item Red koji \' ce za svaku re\v c \v cuvati i njen broj pojavljivanja.
Napisati funkcije za ubacivanje elementa u red, \v stampanje reda,
brisanje reda i
odre\d ivanja najdu\v ze re\v ci koja se pojavila u redu.

\item Ure\d eno binarno drvo (ure\d eno leksikografski) koje u svakom
\v cvoru \v cuva re\v c i broj pojavljivanja te re\v ci. Napisati
funkcije za dodavanje elementa u stablo, ispisivanje stabla, brisanje
stabla i ra\v cunanje ukupnog
broja pojavljivanja svih re\v ci u stablu.

\end{enumerate}



\section{Zavr\v{s}ni ispit, septembar 2009.}



\begin{enumerate}
\item
Napisati program koji iz datoteke \v cije se ime zadaje kao argument
komandne linije u\v citava cele brojeve i sme\v sta ih u listu.
Napisati funkcije za rad sa listom, pravljenje elementa liste, ubacivanje
elementa na kraj liste, ispisivanje liste i brisanje liste.
Pretpostavka je da datoteka sadr\v zi samo cele brojeve.
\item
Napisati program koji sa standardnog ulaza u\v citava cele brojeve dok
se ne unese $0$ kao oznaka za kraj. Brojeve smestiti u ure\d eno
binarno stablo i ispisati dobijeno stablo. Napisati funkcije za formiranje
elementa stabla, ubacivanje elementa u stablo, ispisivanje stabla u
prefiksnom poretku, brisanje stabla.
\item
3. Sa standardnog ulaza u\v citavamo prvo broj studenata a zatim i
njihove podatke. Za svakog studenta dobijamo ime (niska od najvi\v se
30 karaktera) i broj indeksa (ceo broj). Napisati program koji
sortira ovaj niz studenata po imenima studenata pozivom standardne
funkcije \emph{qsort} a zatim pronalazi broj indeksa studenta \v cije
se ime zadaje sa standardnog ulaza pozivom funkcije \emph{bsearch}.

\end{enumerate}


\section{Zavr\v{s}ni ispit, oktobar 2009}


\begin{enumerate}

\item
Napisati program koji sa standardnog ulaza u\v citava pozitivne cele
brojeve dok ne u\v cita nulu kao oznaku za kraj. Na standardni izlaz
ispisati koji broj se pojavio najvi\v se puta me\d u tim brojevima.
Na primer, ako se na ulazu pojave brojevi:
\verb+2 5 12 4 5 2 3 12 15 5 6 6 5+
program treba da vrati broj $5$.

Zadatak re\v siti kori\v{s}\'cenjem dinami\v cke realokacije i strukture
koja sadr\v zi ceo broj i broj njegovih pojavljivanja.

\item
Napisati program koji formira listu od niza celih brojeva koji se
unose sa standardnog ulaza. Oznaka za kraj unosa je $0$. Napisati funkcije za
formiranje \v cvora liste, ubacivanje elementa na kraj liste,
ispisivanje elemenata liste i osloba\d anje liste i u programu demonstrirati
pozive ovih funkcija.


\item Napisati funkcije potrebne za ispisivanje elemenata koji se nalaze
na najve\' coj dubini binarnog stabla.

Na primer, za stablo

\begin{minipage}[t]{120mm}
\begin{verbatim}
        5
      /   \
     3     6
   /   \
  2     4
\end{verbatim}
\end{minipage}

\noindent
treba ispisati: {\tt 2 4}.

(Pretpostavljamo da je stablo ve\' c zadato. \emph{Ne treba} pisati dodatne
funkcije za kreiranje \v cvora, uno\v senje elementa u stablo i
osloba\d anje stabla)


\item
Sa standardnog ulaza u\v citavamo prvo broj studenata a zatim i
njihove podatke. Za svakog studenta dobijamo ime (niska od najvi\v se
30 karaktera) i broj indeksa (ceo broj). Napisati program koji
sortira ovaj niz studenata po imenima studenata pozivom standardne
funkcije \emph{qsort} i zatim \v stampa tako dobijeni niz na standardni
izlaz.
\end{enumerate}


\section{Zavr\v{s}ni ispit, Oktobar 2 2009.}



\begin{enumerate}

\item Napisati program koji formira ure\d eno binarno stablo bez
ponavljanja elemenata \v ciji elementi su imena studenata
(karakterska niska do $30$ karaktera). Napisati program koji sa
standardnog ulaza \v cita podatke o studentima, sme\v sta ih u
stablo (ure\d eno leksikografski) i \v stampa podatke o studentima
infiksno. Oznaka za kraj unosa je kada se umesto imena studenta
unese niska $"kraj"$. Napisati funkcije za kreiranje \v cvora
stabla, umetanje studenta u stablo, brisanje stabla i ispis stabla
na opisan na\v cin.

\item Napisati program koji simulira red u studentskoj poliklinici.
Napisati funkcije \emph{add} (za ubacivanje studenta na kraj
reda), \emph{get} (za izbacivanje studenta sa po\v cetka reda) i
funkcije za \v stampanje i brisanje reda. Podaci o studentu se
sastoje od imena studenta (karakterska niska du\v zine ne ve\' ce
od $30$) i broja indeksa studenta (ceo broj).

\item Napisati program koji sa standardnog ulaza u\v citava prvo
dimenzije matrice ($n$ i $m$) a zatim redom i elemente matrice (ne
postoje pretpostavke o dimenziji matrice). Nakon toga u datoteku
\v cije se ime zadaje kao prvi argument komandne linije, zapisati
sumu elemenata iznad glavne dijagonale.

\item Napisati program koji sa standardnog ulaza unosi prvo broj studenata
a zatim i podatke o studentima (ime studenata - karakterska niska
du\v zine do $30$ karaktera i broj indeksa studenta - ceo broj),
sortira ih po imenu studenta leksikografski (pozivom funkcije
qsort) i ispisuje sortiran niz na standardni ulaz.
\end{enumerate}

