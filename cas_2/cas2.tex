\documentclass{article}
\usepackage[utf8]{inputenc}
\title{{\bf Programiranje 2\\ \emph{Dinamička alokacija memorije}}}

\usepackage{preambula}

\def\d{{\fontencoding{T1}\selectfont\dj}}
\def\D{{\fontencoding{T1}\selectfont\DJ}}

\begin{document}

\maketitle

\section{Zadaci sa časa}

\begin{z}
Napisati program koji sa standardnog ulaza učitava dimenziju niza celih brojeva, a zatim i njegove
elemente. Ne praviti pretpostavke o dimenziji niza. Na standardni
izlaz ispisati učitane brojeve u obrnutom poretku.
U slučaju greške na standardni izlaz za greške ispisati {\tt -1}. 
\end{z}
\begin{Verbatim}%
  [formatcom=\color{primeri}]
\textbf{Primer1:                 Primer 2:                Primer 3:}
5                        7                        -5
1 2 3 4 5                12 10 56 -98 15 2 100    -1

5 4 3 2 1                100 2 15 -98 56 10 12
\end{Verbatim}

\begin{z}
Napisati program koji sa standardnog ulaza učitava niz razlomaka. Prvo se učita broj razlomaka,
a zatim i razlomci u obliku brojilac imenilac. Na standardni izlaz ispisati sve razlomke čija je vrednost veća od
prosečne vrednosti svih učitanih razlomaka, u obliku brojilac/imenilac. 
Ne praviti nikakve pretpostavke o maksimalnoj dimenziji niza. U slučaju greške na standardni izlaz za greške ispisati {\tt -1}. 
\end{z}
\begin{Verbatim}%
  [formatcom=\color{primeri}]
\textbf{Primer1:                 Primer 2:               }
5                        3
1 2                      -5 6
3 4                      100 31
5 7                      2 9
2 3
6 5

6/5                      100/31
\end{Verbatim}

\begin{z}
Napisati program koji učitava niz celih brojeva iz datoteke brojevi.txt  sve do unosa broja 0, koristeći funkciju za realokaciju memorije sa korakom k (koji se zadaje kao argument komandne linije). Na standardni izlaz ispisati sve brojeve koji su veći od središnjeg elementa niza. U slučaju parne dimenzije niza, središnji element računati kao aritmetičku sredinu dva elementa najbliža sredini. 
U slučaju greške na standardni izlaz za greške ispisati {\tt -1}. 
\end{z}
\begin{Verbatim}%
  [formatcom=\color{primeri}]
\textbf{Primer1:                 Primer 2:  }
./a.out 10               ./a.out 5
brojevi.txt              brojevi.txt                    
1 2 3 4 5 0              12 10 56 -98 15 2 100 55 0

4 5                      12 10 56 15 2 100 55
\end{Verbatim}

\begin{z}
Napisati program koji sa standardnog ulaza učitava matricu celih brojeva. Prvo se učitaju broj
vrsta i kolona matrice, a zatim i elementi matrice. Na standardni izlaz ispisati učitanu matricu.
Zatim napisati funkciju \\
\begin{center}{\tt int sum\_max(int **A, int n, int m)} \end{center}
koja računa zbir najvećih elemenata u svakoj vrsti. Ispisati rezultat izvršavanja funkcije na standardni izlaz.
 U slučaju greške na standardni izlaz za greške ispisati {\tt -1}.
\end{z}

\begin{Verbatim}%
  [formatcom=\color{primeri}]
\textbf{Primer1:                 Primer 2:               }
2 3                      3 3
1 2 3                    -5 6 6
4 5 6                    100 31 10
                         2 9 -55

1 2 3                    -5 6 6
4 5 6                    100 31 10
9                        2 9 -55
                         115
\end{Verbatim}

\begin{z}
Sa standardnog ulaza se učitava reč pretrage, dimenzija niza, a zatim i niz reči. 
Pretpostavljati da je maksimalna duzina reči 20 karaktera (prostor za reč alocirati dinamički).
Na standardni izlaz ispisati indeks prvog i poslednjeg pojavljivanja tražene reči u okviru unetog niza reči. U slučaju da 
se reč ne pojavljuje u nizu, ispisati -1 za obe pozicije. 
U slučaju greške na standardni izlaz za greške ispisati {\tt -1}.
\end{z}


\begin{Verbatim}%
  [formatcom=\color{primeri}] 
\textbf{Primer 1:                                              Primer 2:}
kisa	                                           programiranje
9                                                      3
danas je padala kisa a sutra kisa nece padati          mi volimo matematiku

3 6                                                    -1 -1 
\end{Verbatim}

\begin{z}
Definisati strukturu 
\begin{verbatim}
typedef struct{
    unsigned int a, b;
    char ime[5];
} PRAVOUGAONIK;
\end{verbatim}
kojom se opisuje pravougaonik du\v zinama svojih stranica i imenom. Napisati program koji iz datoteke \v cije ime se zadaje kao argument komandne linije u\v citava pravougaonike (nepoznato koliko), a zatim prvo ispisuje imena onih pravougaonika koji su kvadrati, a nakon toga ispisuje vrednost najve\' ce povr\v sine medju pravougaonicima koji nisu kvadrati. 
U slučaju greške na standardni izlaz za greške ispisati {\tt -1}. 
\begin{center}
\begin{Verbatim}%
  [formatcom=\color{primeri}]
\textbf{Primer 1:                        Primer 2:          Primer 3:           Primer 4:              Primer 5:}
./a.out pravougaonici.dat        ./a.out dva.dat    ./a.out tri.dat     ./a.out primerx.dat    ./a.out prazna.dat
pravougaonici.dat:               dva.dat:           tri.dat:            primerx.dat:           prazna.dat:
2 4 p1                           5 2 pm             5 5 m               9 7 p
3 3 p2                           4 7 pv             3 3 s
1 6 p3                                              8 8 xl

p2 8                             28                 m s xl              63         
\end{Verbatim}
\end{center}
\end{z}

\begin{z}
Definisati strukturu \verb|STUDENT| koja sadr\v zi:
\begin{itemize}
\item \verb|puno_ime| (u polju se \v cuva ime i prezime studenta,
  npr. "Marko Markovic", maksimalna du\v zina polja je 100
  karaktera),
\item \verb|ocene| (sadr\v zi najvi\v se 10 ocena studenta)
\item \verb|broj_ocena| (ukupan broj ocena za studenata)
\item \verb|prosek| (prose\v cna ocena)
\end{itemize}
U datoteci čiji se naziv zadaje kao argument komandne linije
se nalaze podaci o studentima (prvo broj studenata, a zatim i i
informacije o svakom studentu). Za svakog studenta unosi se
ime i prezime razdvojeno razmakom, a potom ocene koje se zavr\v savaju sa
0. Napisati funkciju
\begin{center}{\tt int najveci\_prosek(STUDENT* niz, int n)} \end{center}
koja pronalazi studenta sa najvećim prosekom i vraća poziciju u nizu na kojoj se taj student nalazi. \\
Napisati funkciju
\begin{center}{\tt void ispisi(const STUDENT* s)} \end{center}
koja ispisuje podatke o studentu u formatu: ime prezime, prosek na dve decimale.
Testirati obe funkcije pozivom u glavnom programu, tako što prvo treba nači studenta
sa najvećim prosekom, a zatim ispisati informacije o tom studentu.
U slučaju greške na standardni izlaz za greške ispisati {\tt -1}. 
\begin{Verbatim}%
  [formatcom=\color{primeri}]
\textbf{Primer 1:                             Primer 2:           Primer 3:}
./a.out studenti.txt                  ./a.out             ./a.out prazna.dat
studenti.txt:                                             prazna.dat:
4
Marko Markovic 5 6 7 8 9 0            -1
Jelena Jankovic 10 10 10 0
Filip Viskovic 10 9 8 7 6 0
Jana Peric 10 10 9 9 8 8 7 7 0

Jelena Jankovic 10.00
\end{Verbatim}
\end{z}

\newpage

\section{Domaći zadaci}
\begin{z}
Napisati program koji sa standardnog ulaza učitava dimenziju niza celih brojeva, a zatim i njegove
elemente. Na standardni izlaz za svaki broj ispisati koliko ima brojeva koji se nalaze ispred njega u nizu, a koji su manji od tog broja.
U slučaju greške na standardni izlaz za greške ispisati {\tt -1}. 
\end{z}
\begin{Verbatim}%
  [formatcom=\color{primeri}]
\textbf{Primer1:                 Primer 2:                Primer 3:}
5                        7                        -5
1 2 3 4 5                12 10 56 -98 15 2 100    -1

0 1 2 3 4                0 0 2 0 3 1 6
\end{Verbatim}

\begin{z}
Ime datoteke dato je kao argument komandne linije. U datoteci se
nalaze otvorene i zatvorene zagrade i jo\v s nekakav tekst. Napisati program koji proverava
da li su zagrade pravilno uparene. Npr. \verb|ab( cd) ..| odgovor je
\verb|jesu|, a \verb|..)ba()| odgovor je \verb|nisu|.
U slučaju greške na standardni izlaz za greške ispisati {\tt -1}. 
\end{z}
\begin{Verbatim}%
  [formatcom=\color{primeri}]
\textbf{Primer 1:               Primer 2:                Primer 3:               Primer 4:}
./a.out zagrade.txt     ./a.out primer2.dat      ./a.out primer3.dat     ./a.out
zagrade.txt:            primer2.dat:             primer3.dat:
ab( cd) ..              (7+8                     )) 7 + 6 ((             -1
((3+4)*5+1)*9           nisu(
                        uparene                  nisu
jesu
                        nisu
\end{Verbatim}

\begin{z} Skupovi karaktera
\begin{description}
	\item{a)} Napisati C funkciju 
	\verb|int unesiSkup(char **s, FILE* f)|
	kojom se unosi skup elemenata iz datoteke F. Skup se predstavlja kao 
	niz karaktera, pri \v cemu su dozvoljeni elementi skupa mala i velika slova abecede, kao i cifre.
	Unos se prekida kada se nai\d e na znak za novi red ili nedozvoljeni karakter za skup 
	Maksimalan broj elemenata skupa nije poznat.
	Funkcija vra\' ca broj elemenata skupa koji su uspesno u\v citani.
	\item{b)} Napisati funkciju
	\verb|void prebroj(char *s, int *br_slova,int *br_cifara)|
	kojom se odre\d uje broj slovnih elemenata skupa (velikih ili malih slova) 
	kao i broj cifara u skupu.
	\item{c)} Napisati glavni program gde se unose podaci o skupu elemenata. Ime datoteke se zadaje kao argument komandne linije.
	Na stadardni izlaz ispisati informacije o broju slova i cifara (koristiti funkcije pod a) i b)).
      \end{description}
U slučaju greške na standardni izlaz za greške ispisati {\tt -1}. 
\end{z}
\begin{Verbatim}%
  [formatcom=\color{primeri}]
\textbf{Primer 1:                   Primer 2:          Primer 3:                 Primer 4:}
./a.out skup.txt            ./a.out            ./a.out skup2.txt         ./a.out skup3.txt
skup.txt:                                      skup2.txt:                skup3.txt:
abc56ighj9012hjFGHH         -1                 ovdeimamo\$dolar           broJ3
                                                                         broj5
broj slova: 13                                 broj slova: 9
broj cifara: 6                                 broj cifara: 0            broj slova: 4 
                                                                         broj cifara: 1
\end{Verbatim}

\begin{z}
Definisati strukturu 
\begin{verbatim}
typedef struct{
    int x;
    int y;
    int z;
} VEKTOR;
\end{verbatim}
kojom se opisuje trodimenzioni vektor. U datoteci \verb|vektori.txt|
nalazi se nepoznati broj vektora. Na standarni
izlaz ispisati koordinate vektora sa najve\' com du\v zinom. Du\v zina vektora se izra\v cunava
po formuli:
$$|v|= \sqrt{x^2+y^2+z^2}$$ U slučaju greške na standardni izlaz za greške ispisati {\tt -1}. 
\begin{Verbatim}%
  [formatcom=\color{primeri}]
\textbf{   Primer 1:           Primer 2:            Primer 3:           Primer 4:}
   2                   -9                   3                   4
   4 -1 7                                   0 0 0               3 0 1
   3  1 2                                   0 1 0               4 5 2
                                            1 0 0               1 0 0
                                                                2 -1 2
								
   4 -1 7              -1                   0 1 0               4 5  2

\end{Verbatim}
\end{z}

\end{document}

\begin{z}
Ovdde ide tekst zadatka.
\end{z}
\begin{Verbatim}%
  [formatcom=\color{primeri}]
\textbf{Primer1:                 Primer 2:}
nije bf             vlvlvl
nesto dalje         zzzz
\end{Verbatim}
